\documentclass[a4paper, 12pt]{article}

\usepackage[english]{babel}
\usepackage[margin=0.5in]{geometry}

\usepackage[utf8]{inputenc}
\usepackage[T1]{fontenc}
\usepackage{lmodern}
\usepackage{units}
\usepackage{eurosym}
\usepackage{graphicx}
\usepackage{wrapfig}
\usepackage{color}
%\usepackage{url}
\usepackage{hyperref}
\usepackage{enumerate}
\usepackage{enumitem}
\usepackage{pifont}
\usepackage[normalem]{ulem}

% packages
\usepackage{amsmath}
\usepackage{amssymb}
\usepackage{amsthm}
\usepackage{amsfonts}
\usepackage{mathtools}
\usepackage{tikz-cd}
\usetikzlibrary{babel}
\usepackage{adjustbox}
\usepackage{stmaryrd}

% commonly used math operators
\DeclareMathOperator{\diam}{diam}
\DeclareMathOperator{\diag}{diag}
\DeclareMathOperator{\rank}{rank}
\DeclareMathOperator{\tr}{tr}
\DeclareMathOperator{\im}{im}
\DeclareMathOperator{\dom}{dom}
\DeclareMathOperator{\coker}{coker}
\DeclareMathOperator{\codim}{codim}
\DeclareMathOperator{\pr}{pr}
\DeclareMathOperator{\rad}{rad}
\DeclareMathOperator{\chrs}{char}
\DeclareMathOperator{\len}{len}
\DeclareMathOperator{\Lin}{Lin}
\DeclareMathOperator{\Ann}{Ann}
\DeclareMathOperator{\Ass}{Ass}
\DeclareMathOperator{\Spec}{Spec}
\DeclareMathOperator{\mSpec}{mSpec}
\DeclareMathOperator{\Quot}{Quot}
\DeclareMathOperator{\Tor}{Tor}
\DeclareMathOperator{\Ext}{Ext}
\DeclareMathOperator{\Hom}{Hom}
\DeclareMathOperator{\End}{End}
\DeclareMathOperator{\Aut}{Aut}
\DeclareMathOperator{\Br}{Br}
\DeclareMathOperator{\Gal}{Gal}

% commonly used math objects
\newcommand{\F}{\mathbb{F}}
\newcommand{\A}{\mathbb{A}}
\newcommand{\D}{\mathbb{D}}
\renewcommand{\S}{\mathbb{S}}
\newcommand{\T}{\mathbb{T}}
\newcommand{\B}{\mathbb{B}}
\newcommand{\I}{\mathbb{I}}
\newcommand{\N}{\mathbb{N}}
\newcommand{\Z}{\mathbb{Z}}
\newcommand{\Q}{\mathbb{Q}}
\newcommand{\R}{\mathbb{R}}
\newcommand{\C}{\mathbb{C}}
\renewcommand{\H}{\mathbb{H}}
\renewcommand{\P}{\mathbb{P}}

% commonly used math relations
\newcommand{\iso}{\cong}
\newcommand{\homeo}{\approx}
\newcommand{\htpeq}{\simeq}
\newcommand{\hlgeq}{\sim}
\newcommand{\idtfy}{\longleftrightarrow}

% commonly used math symbols
\newcommand{\closure}[1]{\overline{#1}}
\newcommand{\subideal}{\vartriangleleft}
\newcommand{\supideal}{\vartriangleright}

% numbered environments
\theoremstyle{plain}
\newtheorem{theorem}{Theorem}[section]
\newtheorem{corollary}[theorem]{Corollary}
\newtheorem{exercise}[theorem]{Exercise}
\newtheorem{lemma}[theorem]{Lemma}
\newtheorem{proposition}[theorem]{Proposition}

\theoremstyle{definition}
\newtheorem{definition}[theorem]{Definition}

\theoremstyle{remark}
\newtheorem*{claim}{Claim}
\newtheorem*{remark}{Remark}

\newcounter{excounter}[section]
\newenvironment{Exercise}
    {\refstepcounter{excounter}\underline{\textbf{Ex. \theexcounter:}}}
    {\par\vspace{\baselineskip}}


\title{Introduction to Algebraic Geometry - $1^\text{st}$ homework}
\author{Benjamin Benčina}
\date{\today}

\begin{document}

\maketitle

\begin{Exercise}
    Let $A$ be a commutative ring with a unit element.
    Let us show the following statements:
    \begin{enumerate}[label=(\alph*)]
        \item \underline{\emph{the set of nilpotent elements in $A$ forms an ideal:}}

            We need to merely verify that the set $N$ of nilpotents in $A$ is an additive subgroup in $A$ which is closed under multiplication with elements from $A$.
            Indeed, take $a, b \in N$ and $n,m \in \N$ such natural numbers that $a^n = b^m = 0$.
            We calculate
            \begin{align*}
                (a+b)^{n+m}
                &= \sum_{k=0}^{n+m}\binom{n+m}{k}a^{k}b^{n+m-k} \\
                &= \underbrace{\sum_{k=0}^{n}\binom{n+m}{k}a^{k}b^{n+m-k}}_{\text{div. by $b^m$}} + \underbrace{\sum_{k=n+1}^{n+m}\binom{n+m}{k}a^{k}b^{n+m-k}}_{\text{div. by $a^n$}} \\
                &= 0
            \end{align*}
            Now, take $a \in N$ with $a^n = 0$ and $x \in A$. Since $A$ is commutative, clearly
            \[
                (xa)^n = x^na^n = 0
            \]
            Notice that the case $x = -1$ (where $1$ is the unit element) proves that $N$ is closed for additive inverses as well, and hence an ideal.
            This ideal is commonly called the \emph{nilradical} of $A$ and is in fact equal to the intersection of all prime ideals of $A$.\footnote{Last year's Commutative Algebra course covered this.}
        \item \underline{\emph{the sum of a nilpotent and a unit element is always a unit:}}

            Let $a \in N$ with $a^n = 0$ and $u \in A^{-1}$.
            Since $a$ is nilpotent, we get the following (informal) idea for the inverse
            \[
                \frac{1}{u + a} = \frac{u^{-1}}{1 + u^{-1}a} = u^{-1} \sum_{k=0}^{\infty}(-u^{-1}a)^k = u^{-1} \sum_{k=0}^{n-1}(-u^{-1}a)^k
            \]
            Let us check that the above is indeed the element $(u + a)^{-1}$:
            \begin{align*}
                ( a + u ) \cdot u^{-1}\sum_{k=0}^{n-1}(-u^{-1}a)^k
                &= u^{-1}\sum_{k=0}^{n-1}(-u^{-1})^ka^{k+1} + \sum_{k=0}^{n-1}(-u^{-1}a)^k \\
                &= \sum_{k=1}^{n-1}(-1)^{k-1}(u^{-1}a)^{k} + \sum_{k=0}^{n-1}(-1)^{k}(u^{-1}a)^k \\
                &= (-u^{-1}a)^0 = 1
            \end{align*}
            since only the element at power $0$ survives.
        \item \underline{\emph{$f \in A[x]$ is nilpotent $\iff$ all its coefficients are nilpotent:}}
            \begin{itemize}
                \item \underline{($\impliedby$):}
                    Let $f = a_0 + a_1x + a_mx^m \in A[x]$ with $a_0^{n_0} = \dots = a_m^{n_m} = 0$.
                    Then we can now use the multinomial formula to get $f^{n_0+\cdots+n_m} = 0$ the same way we used the binomial formula in (1.a).
                \item \underline{($\implies$):}
                    Suppose now that $f^n = 0$ for some $n$ (write $f$ by coefficients as in the converse).
                    We will do a sort of induction on the degree $\deg f$.
                    If $\deg f = 0$ then the statement trivially holds, as all higher coefficients are zero.
                    For the induction step suppose $\deg f = m$.
                    Since
                    \[
                        f^n(x) = x^{mn}a_m^n + \cdots + a_0^{n} = 0
                    \]
                    we get in particular that $a_0$ is nilpotent.
                    By (1.a), it follows that
                    \[
                        f - a_0 = a_1x + \cdots + a_mx^m = x(a_1 + a_2x + \cdots + a_mx^{m-1})
                    \]
                    is nilpotent, but this will happen precisely when $g(x) = a_1 + \cdots a_mx^{m-1}$ is nilpotent.
                    Notice that $\deg g < \deg f$, so by the induction hypothesis, all coefficients of $g$ are nilpotent.
                    By construction, all coefficient of $f$ are nilpotent and the proof is complete.
            \end{itemize}
        \item \underline{\emph{$f \in A[x]$ is a unit $\iff$ $a_0$ is a unit and the other coefficients are nilpotent:}}
            \begin{itemize}
                \item \underline{($\impliedby$):}
                    If $a_0$ is a unit and $a_1,\dots,a_m$ are nilpotent,
                    then by (1.c) the polynomial $g(x) = a_1x + \cdots + a_mx^m$ is nilpotent,
                    hence by (1.b) the polynomial $f = a_0 + g$ is a unit.
                \item \underline{($\implies$):}
                    Suppose $g = f^{-1}$ with $g(x) = b_0 + \cdots + b_rx^r$.
                    Again we prove the claim by induction on $\deg f$.
                    If $\deg f = 0$, then the statement trivially holds, as all higher coefficients are zero.
                    Suppose $\deg f = m$ and multiply
                    \[
                        1 = fg = \sum_{k=0}^{m+r}c_kx^k
                    \]
                    where
                    \[
                        c_k = \sum_{i+j=k}a_ib_j
                    \]
                    Clearly, $b_0 = a_0^{-1}$, so $a_0$ is a unit.
                    We now compare coefficients from the other end.
                    Since $a_mb_r = 0$, we get
                    \begin{align*}
                        a_{m-1}b_r + a_mb_{r-1} = 0 &\overset{a_m\cdot}{\implies} a_m^2b_{r-1} = 0 \\
                        a_{m-2}b_r + a_{m-1}b_{r-1} + a_mb_{r-2} = 0 &\overset{a_m^2\cdot}{\implies} a_m^3b_{r-2} = 0 \\
                        &\cdots \\
                        \sum_{i+j = k} a_ib_j &\overset{a_m^{r-k}\cdot}{\implies} a_m^{r+1-k}b_k = 0 \\
                    \end{align*}
                    and hence $a_m^{r+1}b_0 = 0$, but since $b_0$ is a unit, $a_m$ is nilpotent.
                    Then by (1.b), $h(x) = f(x) - a_mx^m$ is a unit and a polynomial with $\deg h < \deg f$.
                    By the induction hypothesis, we get that $a_1,\dots,a_{m-1}$ are nilpotent,
                    so the proof is complete.
            \end{itemize}
        \item \underline{\emph{$f \in A[x]$ is a zero divisor $\iff$ there exists a non-zero $a \in A$ with $af=0$:}}
            \begin{itemize}
                \item \underline{($\impliedby$):}
                    Obvious, since $A \hookrightarrow A[x]$ via constant polynomials.
                \item \underline{($\implies$):}
                    Let $fg = 0$ for non-zero polynomials $f$ and $g$, and denote the coefficients of $f$ and $g$ as above.
                    If $\deg f = 0$ the statement is again trivially true, as $f \in A$ via the identification from the converse.
                    Suppose $\deg f = m$.
                    Then clearly $a_mb_r = 0$, so $b_rf$ is a polynomial with $(b_rf)g = 0$ with $\deg b_rf < \deg f$.
                    By the induction hypothesis, there exists a non-zero constant $a \in A$ such that $ab_rf = 0$.
                    Then, by associativity, $a(b_rf) = (ab_r)f = 0$ and $a b_r \in A$ non-zero.
            \end{itemize}
    \end{enumerate}
\end{Exercise}

\begin{Exercise}
    Let $C = \left\{ (x, y) \in \A^2 ; \; y^2 - x^3 = 0 \right\}$.
    \begin{itemize}
        \item Is the map $\varphi \colon \A^1 \to C$, defined by $\varphi(t) = (t^2, t^3)$, an isomorphism of affine varieties?

            \underline{NO}.
            A map $\phi$ is an isomorphism of affine varieties if and only if the map $\phi^*$ is an algebra isomorphism.
            In our case we have $\im\varphi^* = F[t^3, t^2]$, which is not isomorphic to $F[t]$.
            Indeed, $t \notin \im\varphi^*$.

            Another way to see this is by noticing that
            \[
                \psi(x, y) = \varphi^{-1}(x, y) =
                \begin{cases}
                    \frac{y}{x} ; \; x \neq 0 \\
                    0 ; \; x = 0
                \end{cases}
            \]
            is not a morphism, since it is not a regular map.
            (This exercise is a typical example to show that isomorphism $\neq$ bijective morphism.)
        \item Is $\varphi$ a homeomorphism with respect to the Zariski topology?

            \underline{YES}.
            The map $\varphi$ is clearly Zariski-continuous as a polynomial map.
            It is also bijective (we have the inverse above), so what remains to show is that $\psi$ is also Zariski-continuous.
            Since $\psi \colon C \to \A^1$ and since the Zariski topology on $\A^1$ is precisely the finite-complements topology,
            it is enough to see that $\psi^{-1}(c)$ is Zariski-closed in $C$ (or $\A^2$) for every point $c \in A^1$,
            which is obvious as points are closed in $C$ ($\psi$ is bijective, hence $1-1$).
    \end{itemize}
\end{Exercise}

\begin{Exercise}
    We want to find the irreducible components of the affine variety
    \[
        V(x - yz, xz - y^2) \subset \A^3.
    \]
    We calculate
    \begin{align*}
        V(x - yz, xz - y^2)
        &= V(x - yz, yz^2 - y^2) \\
        &= V(x - yz, y(z^2 - y)) \\
        &= V(x - yz) \cap (V(y) \cup V(z^2 - y)) \\
        &= (V(x -yz) \cap V(y)) \cup (V(x - yz) \cap V(z^2 - y)) \\
        &= (V(x) \cap V(y)) \cup (V(x - z^3) \cap V(y - z^2)) \\
        &= V(x, y) \cup V(x - z^3, y - z^2)
    \end{align*}
    which are both clearly irreducible.
    The first component is the $z$-axis and the second component is the curve $t \mapsto (t^3, t^2, t)$.
    Alternatively, both associated ideals are clearly prime, since $F[x, y, z]/(x,y) \iso F[z]$ and $F[x, y, z]/(x-z^3,y-z^2) \iso F[z]$ are domains.
\end{Exercise}

\begin{Exercise}
    Let us determine the radical of the ideal $I = (x^3 - y^6, xy - y^3) \subideal \C[x, y]$.

    Let us first think of the solution informally:
    \begin{align*}
        &x^3 = y^6 \implies x = y^2 \implies y^2 - x = 0 \\
        &xy = y^3 \implies y(y^2 - x) = 0
    \end{align*}
    So we start suspecting that $(y^2 -x ) = \sqrt{I}$.\footnote{Calculating this ideal in Macaulay 2 (with $\Q$ coefficients) also gives this solution, which is a strong indicator.}
    \begin{itemize}
        \item \underline{($\subseteq$):}
            Take $f \in (y^2 - x)$, that is, $f = g \cdot (y^2 - x)$ for some $g \in \C[x, y]$.
            We calculate
            \begin{align*}
                f^3
                &= g^3\cdot (y^2 - x)^3 \\
                &= g^3 \cdot (y^6 - 3y^4x + 3y^2x^2 - x^3) \\
                &= g^3 \cdot (y^6 - x^3) + 3g^3 \cdot (y^2x^2 - y^4x) \\
                &= g^3 \cdot (y^6 - x^3) + 3g^3 \cdot xy \cdot (xy - y^3) \in I
            \end{align*}
            Hence, $f \in \sqrt{I}$.
        \item \underline{($\supseteq$):}
            Let $f \in I$, that is, $f = a \cdot (x^3 - y^6) + b \cdot (xy - y^3)$ for some $a, b \in \C[x, y]$.
            Further, we calculate
            \begin{align*}
                f &= a \cdot (x^3 - y^6) + b \cdot (xy - y^3) \\
                &= a \cdot (x - y^2)^3 + 3a \cdot xy \cdot (xy - y^3) + b \cdot (xy - y^3) \\
                &= a \cdot (x - y^2)^3 + (3axy + b)\cdot xy \cdot (x - y^2) \in (y^2 - x)
            \end{align*}
            Hence, $I \subseteq (y^2 - x)$, but $(y^2 - x)$ is clearly radical, e.g. $\C[x, y]/(y^2-x) \iso \C[y]$ is reduced.
            So, $\sqrt{I} \subseteq (y^2 - x)$.
    \end{itemize}
\end{Exercise}

\begin{Exercise}
    Let $X$ be the union of the three coordinate axes.
    We will compute the generators of the ideal $I(X)$ and show that $I$ cannot be generated by fewer than three elements.

    We write
    \[
        X = V(x,y) \cup V(y, z) \cup V(z, x)
    \]
    So we get
    \[
        I(X) = I(V(x, y)) \cap I(V(y, z)) \cap I(V(z, x)) = (x, y) \cap (y, z) \cap (z, x)
    \]
    since each of the ideals is clearly radical.
    Denote $I_1 = (x, y)$, $I_2 = (y, z)$, and $I_3 = (z, x)$.
    If we take, e.g., the ideals $I_1$ and $I_2$ and canonically gather coefficients at $y$,
    we get that the remaining coefficient of an element in $I_1 \cap I_2$ has to be divided by both $x$ and $z$, hence by $xz$.
    Taking all other combinations of ideals, we get that each element in $I_1 \cap I_2 \cap I_3$ can be written as a combination of elements $xy$, $yz$, and $xz$.
    Hence $I_1 \cap I_2 \cap I_3 \subseteq (xy, yz, xz)$, and clearly $(xy, yz, xz) \subseteq I_i$ for $i = 1, 2, 3$, so we get $I_1 \cap I_2 \cap I_3 = (xy, yz, xz)$.\footnote{We also get the same result if we calculate the intersection with Macaulay 2.}

    Now, suppose for contradiction that there exist $p, q \in F[x, y]$ such that $I = I(X) = (p, q)$.
    Notice, that $I$ is a homogeneous ideal, as we have seen it can be generated by three homogeneous elements.
    Take the homogeneous maximal ideal $M = (x, y, z)$ and consider the quotient $I/MI$ of $F$-modules, which is now an $F$-vector space.
    Then by assumption, $\overline{p}$ and $\overline{q}$ generate $I/MI$, so $\dim_F I/MI \leq 2$ (in the case where $p$ and $q$ are not linearly independent, we can get $1$).
    However, $\overline{xy}$, $\overline{yz}$, and $\overline{xz}$ are $F$-linearly independent in $I/MI$.
    Indeed, take $\alpha, \beta, \gamma \in F$ and consider
    \[
        \alpha \overline{xy} + \beta \overline{yz} + \gamma \overline{xz} = 0
    \]
    Since we are in the quotient, we get that
    \[
        \alpha xy + \beta yz + \gamma xz \in MI
    \]
    However, since the product of ideals can be obtained by simply multiplying all generators,
    $MI$ is generated by homogeneous monomials of degree $3$, that is,
    take the product of generating sets $\lbrace x, y, z \rbrace$ and $\lbrace xy, yz, xz \rbrace$ (minus redundancy).
    Hence, no element in $MI$ has non-zero terms of degree $2$, so all $\alpha, \beta, \gamma$ must be zero.
    So $\dim_F I/MI \geq 3$, a contradiction.
\end{Exercise}

\begin{Exercise}
    Let $Y$ be a non-empty irreducible subvariety of an affine variety $X$ and denote $U = X \setminus Y$.
    We assume that the coordinate ring $F[X] = \mathcal{O}_X(X)$ of $X$ is a unique factorization domain.
    We will show that $\mathcal{O}_X(U) = \mathcal{O}_X(X)$ if and only if $\codim Y \geq 2$.

    Firstly, recall that $F[X]$ is a unique factorization domain precisely when every prime ideal of codimension $1$ is principal.
    Furthermore, every irreducible variety of codimension $1$ is then defined by a single irreducible polynomial.
    Now, suppose $\codim Y < 2$. Then $Y = V(f)$, hence $U = D(f)$.
    Another theorem from the lectures then tells us that $\mathcal{O}_X(U) = F[X]_f$, which is not isomorphic to $F[X]$.
    Conversely, let $\codim Y \geq 2$.
    We want to show that every regular function on $U$ extend to a regular function on $X$.
    If that is indeed the case, the extension is unique by a Corollary from lectures, as the extensions would match on the open set $U$.
    Write $Y = V(I)$, where $I$ is the associated prime ideal to the irreducible subvariety $Y$.
    Then $U = \bigcup_{f \in I} D(f)$.
    Furthermore, there exist independent irreducible polynomials $f_1,\dots,f_r$ for some $r \geq \codim Y$ such that $U = \bigcup_{i = 1}^{r} D(f_i)$
    (the important part is not how many there are per se, but rather that $r \geq 2$ and that they do not divide each other).
    Indeed, since $\codim I \geq 2$, there exist prime ideals $P_0, P_1, P_2 \subideal F[X]$ such that
    \[
        (0) \subsetneq P_0 \subsetneq P_1 \subsetneq P_2 \subseteq I
    \]
    Then there exist polynomials $f_1$ and $f_2$ such that $f_1 \in P_1$ but $f_1 \notin P_0$, and $f_2 \in P_2$ but $f_2 \notin P_1$.
    Since the ideals are prime, $f_1$ and $f_2$ must necessarily be independent, and clearly $D(f_1),D(f_2) \subseteq U$.
    We work in Noetherian rings so we can always find finitely many, but as we have seen, at least two.
    Now, take $\varphi \in \mathcal{O}_X(U)$.
    On $D(f_i)$ we have that $\varphi = \frac{g_i}{f_i^{k_i}}$ for some $g_i \in F[X]$ and $k_i \in \N_0$, where $g_i$ is not divisible by $f_i$.
    Since $r \geq 2$, we look at intersections.
    On $D(f_i) \cap D(f_j)$ we have
    \[
        \frac{g_i}{f_i^{k_i}} = \frac{g_j}{f_j^{k_j}} \implies g_if_j^{k_j} = g_jf_i^{k_i}
    \]
    hence $k_i = 0$ for all $i = 1,\dots, r$ and all $g_i$ are equal, as they pairwise match on open intersections.
    Since $D(f_i)$ cover $U$, we have found an extension of $\varphi$ to $X$ as a regular function, hence $\varphi \in \mathcal{O}_X(X)$ and the proof is complete.

    Lastly, we will find a counter-example if $\mathcal{O}_X(X)$ is not a unique factorization domain.
    We take, e.g., the $3$-dimensional affine variety $X = V(x_1x_4 - x_2x_3) \subset \A^4$,
    where the element $x_1x_4 = x_2x_3$ has two ways of factoring in $F[X] = F[x_1,x_2,x_3,x_4]/(x_1x_4 - x_2x_3)$.
    For $Y$ we take the irreducible subvariety $V(x_1x_4-x_2x_3, x_1, x_2) = V(x_1, x_2)$.
    Indeed, we know from tutorials that $J = (x_1, x_2)$ is a prime ideal in $F[X]$ of $\codim J = 1$, which is, importantly, not principal.
    We therefore get that $U = D(x_1) \cup D(x_2)$ and continue as above to obtain $\mathcal{O}_X(U) = \mathcal{O}_X(X)$, even though $\codim Y = \codim J = 1$.
\end{Exercise}

\end{document}
