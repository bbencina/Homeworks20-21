\documentclass[a4paper, 12pt]{article}

\usepackage[english]{babel}
\usepackage[margin=0.5in]{geometry}

\usepackage[utf8]{inputenc}
\usepackage[T1]{fontenc}
\usepackage{lmodern}
\usepackage{units}
\usepackage{eurosym}
\usepackage{graphicx}
\usepackage{wrapfig}
\usepackage{color}
%\usepackage{url}
\usepackage{hyperref}
\usepackage{enumerate}
\usepackage{enumitem}
\usepackage{pifont}
\usepackage[normalem]{ulem}

% packages
\usepackage{amsmath}
\usepackage{amssymb}
\usepackage{amsthm}
\usepackage{amsfonts}
\usepackage{mathtools}
\usepackage{tikz-cd}
\usetikzlibrary{babel}
\usepackage{adjustbox}
\usepackage{stmaryrd}

% commonly used math operators
\DeclareMathOperator{\diam}{diam}
\DeclareMathOperator{\diag}{diag}
\DeclareMathOperator{\rank}{rank}
\DeclareMathOperator{\tr}{tr}
\DeclareMathOperator{\im}{im}
\DeclareMathOperator{\dom}{dom}
\DeclareMathOperator{\coker}{coker}
\DeclareMathOperator{\codim}{codim}
\DeclareMathOperator{\pr}{pr}
\DeclareMathOperator{\rad}{rad}
\DeclareMathOperator{\chrs}{char}
\DeclareMathOperator{\len}{len}
\DeclareMathOperator{\Lin}{Lin}
\DeclareMathOperator{\Ann}{Ann}
\DeclareMathOperator{\Ass}{Ass}
\DeclareMathOperator{\Spec}{Spec}
\DeclareMathOperator{\mSpec}{mSpec}
\DeclareMathOperator{\Quot}{Quot}
\DeclareMathOperator{\Tor}{Tor}
\DeclareMathOperator{\Ext}{Ext}
\DeclareMathOperator{\Hom}{Hom}
\DeclareMathOperator{\End}{End}
\DeclareMathOperator{\Aut}{Aut}
\DeclareMathOperator{\Br}{Br}
\DeclareMathOperator{\Gal}{Gal}

% commonly used math objects
\newcommand{\F}{\mathbb{F}}
\newcommand{\A}{\mathbb{A}}
\newcommand{\D}{\mathbb{D}}
\renewcommand{\S}{\mathbb{S}}
\newcommand{\T}{\mathbb{T}}
\newcommand{\B}{\mathbb{B}}
\newcommand{\I}{\mathbb{I}}
\newcommand{\N}{\mathbb{N}}
\newcommand{\Z}{\mathbb{Z}}
\newcommand{\Q}{\mathbb{Q}}
\newcommand{\R}{\mathbb{R}}
\newcommand{\C}{\mathbb{C}}
\renewcommand{\H}{\mathbb{H}}
\renewcommand{\P}{\mathbb{P}}

% commonly used math relations
\newcommand{\iso}{\cong}
\newcommand{\homeo}{\approx}
\newcommand{\htpeq}{\simeq}
\newcommand{\hlgeq}{\sim}
\newcommand{\idtfy}{\longleftrightarrow}

% commonly used math symbols
\newcommand{\closure}[1]{\overline{#1}}
\newcommand{\subideal}{\vartriangleleft}
\newcommand{\supideal}{\vartriangleright}

% numbered environments
\theoremstyle{plain}
\newtheorem{theorem}{Theorem}[section]
\newtheorem{corollary}[theorem]{Corollary}
\newtheorem{exercise}[theorem]{Exercise}
\newtheorem{lemma}[theorem]{Lemma}
\newtheorem{proposition}[theorem]{Proposition}

\theoremstyle{definition}
\newtheorem{definition}[theorem]{Definition}

\theoremstyle{remark}
\newtheorem*{claim}{Claim}
\newtheorem*{remark}{Remark}

\newcounter{excounter}[section]
\newenvironment{Exercise}
    {\refstepcounter{excounter}\underline{\textbf{Ex. \theexcounter:}}}
    {\par\vspace{\baselineskip}}


\title{Introduction to Algebraic Geometry - $2^{\text{nd}}$ homework}
\author{Benjamin Benčina}
\date{\today}

\begin{document}

\maketitle

\begin{Exercise}
    \begin{enumerate}[label=(\roman*)]
        \item Let $R = \C[x,y]$ and $f(x, y) = x^4 + y^3 + 4y^2 + 6y + 3$ a polynomial in $R$.
            Let us check that $J = (f)$ is a prime ideal.
            
            Indeed, write $\C[x,y] = \C[y][x]$ and
            \[
                f(x, y) = 1 \cdot x^4 + (y^3 + 4y^2 + 6y + 3) \cdot 1
                = 1 \cdot x^4 + (y + 1)(y^2 + 3y + 3) \cdot 1
            \]
            so if we take $P = (y + 1) \subideal \C[y]$ we see that,
            of course, $a_4 = 1 \notin P$, $a_0 \in P$ and clearly $a_0 \notin P^2$.
            Hence, by Eisenstein's criterion, $f$ is irreducible and hence prime,
            since $\C[y]$ and $\C[x,y]$ are unique factorization domains.
        \item Let us now find all maximal ideals that contain $J$.

            Since the above equation defines a curve,
            maximal ideals containing $J$ are precisely those given by points on the curve.
            That is, for any $b \in \C$ we get $4$ maximal ideals $M_i = (x - a_i, y - b)$ for $i = 1,\dots,4$,
            where
            \begin{align*}
                a_1 &= \sqrt[4]{-b^3 - 4b^2 - 6b - 3} \\
                a_2 &= -\sqrt[4]{-b^3 - 4b^2 - 6b - 3} \\
                a_3 &= i\sqrt[4]{-b^3 - 4b^2 - 6b - 3} \\
                a_4 &= -i\sqrt[4]{-b^3 - 4b^2 - 6b - 3}
            \end{align*}
            some of which may be the same.
        \item We homogenize the above polynomial with respect to coordinates $[x, y, z]$ in $\P^2$.
            \[
                f^h(x, y, z) = x^4 + y^3z + 4y^2z^2 + 6yz^3 + 3z^4
            \]
            This is now a homogeneous polynomial of degree $4$ (we added the
            new coordinate at the end instead of the beginning, which is
            equivalent).
        \item Let us find the homogeneous ideals that contain $J' = (f^h)$.

            We know that $J'$ defines a curve in $\P^2$ (represented by a conic surface in $\A^3$.
            By the Projective Nullstelensatz, radical homogeneous ideals that contain $J'$
            are precisely given by projective subvarieties of the above projective curve,
            and these will in turn be given by finite intersections of maximal ideals $M_p$
            of polynomials vanishing at a point $p$ on the above curve (given by a line in $\A^3$ through $p$ and the origin).
            Note that these ideals are not actually maximal, since they are all properly contained in the irrelevant ideal $I_0$,
            they are merely maximal in their respective local rings at $p$.

            To find all of these point ideals for every $p$ on the curve, we consider the affine covering.
            In $U_2$ where $z = 1$, we immediately see that the points on the curve are $[a, b, 1]$ for $(a, b)$ zeros of $f$.
            From the procedure from tutorials we get point ideals
            \[
                M_{[a, b, 1]} = (x - az, y - bz)
            \]
            In $U_1$ where $y = 1$, we get a similar result.
            We are searching for solutions of
            \[
                x^4 + z + 4z^2 + 6z^3 + 3z^4 = 0
            \]
            which are as above given by
            \begin{align*}
                a_1 &= \sqrt[4]{-c(1 + 4c + 6c^2 + 3c^3)} \\
                a_2 &= -\sqrt[4]{-c(1 + 4c + 6c^2 + 3c^3)} \\
                a_3 &= i\sqrt[4]{-c(1 + 4c + 6c^2 + 3c^3)} \\
                a_4 &= -i\sqrt[4]{-c(1 + 4c + 6c^2 + 3c^3)}
            \end{align*}
            for any $c \in \C$, where some solutions may be the same.
            We then similarly get point ideals
            \[
                M_{[a,1,c]} = (x - ay, z - by)
            \]
            In $U_0$ where $x = 1$, we get the equation
            \[
                1 + y^3z + 4y^2z^2 + 6yz^3 + 3z^4 = 0
            \]
            and denote by $(b, c)$ its solutions.\footnote{I solved the equation in \texttt{Mathematica}.}
            We then obtain point ideals
            \[
                M_{[1, b, c]} = (y - bx, z - cx)
            \]
            This last step is luckily not necessary,
            since the only possible zero not covered by $U_1$ and $U_2$ is $[1, 0, 0]$,
            which is not even on the curve.

            Lastly, these are all contained in $I_0$ which is also a radical ideal.
    \end{enumerate}
\end{Exercise}

\begin{Exercise}
    Let $C \subset \A^2$ be an algebraic curve given by the equation
    \[
        x^4 - x^2y - y^3 = 0
    \]
    \begin{enumerate}[label=(\roman*)]
        \item Let us find all the singular points of $C$.

            We are solving the system of equations
            \[
                F(p) = \frac{\partial F}{\partial x}(p) = \frac{\partial F}{\partial y}(p) = 0
            \]
            In our case, we are solving
            \begin{align*}
                x^4 - x^2y - y^3 &= 0 \\
                2x^3 - xy &= 0 \\
                x^2 + 3y^2 &= 0
            \end{align*}
            If $x = 0$ it immediately follows that $y = 0$.
            If $x \neq 0$, the second equation yields $y = 2x^2$ and hence
            \[
                x^4 - 2x^4 - 8x^6 = 0
                \implies x^4(1 + 8x^2) = 0
                \overset{x\neq 0}{\implies} x^2 = -\frac{1}{8}
            \]
            We then have $y = -\frac{1}{4}$ and inputting into the third equation
            \[
                -\frac{1}{8} + \frac{3}{16} = \frac{1}{16} \neq 0
            \]
            We therefore have the one singularity $p = (0, 0)$.
        \item We show that $C$ is rational by parametrizing it.
            Indeed, let us use the usual method of lines through the origin.
            Write $y = tx$ and calculate
            \[
                x^4 - tx^3 -t^3x^3 = 0
                \implies x^4 - (t+t^3)x^3 = 0
                \implies x^3\left( x - (t + t^3) \right) = 0
            \]
            Since without loss of generality $x \neq 0$, we get the parametrization
            \[
                x = t + t^3, \quad\quad y = t^2 + t^4
            \]
            The birational map $\A^1 \to C$ is then given by
            \[
                t \mapsto (t + t^3, t^2 + t^4)
            \]
            For the corresponding projective curve $\overline{C}$ we simply
            homogenize and get the birational map $\P^1 \to \overline{C}$ as
            \[
                [t, s] \mapsto [ts^3 + t^3s, t^2s^2 + t^4, s^4]
            \]
        \item Now consider the blow-up $\pi \colon \widetilde{\A^2} \to \A^2$
            at the singular point $p = (0, 0)$.
            Denote by $E$ its exceptional line.
            Let us explicitly describe $\overline{\pi^{-1}(C \setminus (0, 0))} \cap E$.

            We know that
            \[
                \pi^{-1}(C)
                = \left\{ ((x, y), [t, s]) ; \; x^4 - x^2y - y^3 = 0, \; ty = sx \right\}
            \]
            Now take the lowest degree terms and calculate the tangents at $p$ as
            \[
                x^2y + y^3 = y(x^2 + y^2) = y(y + ix)(y - ix) = 0
            \]
            hence the tangents at $p$ are precisely
            \[
                y = 0, \quad\quad y = ix, \quad\quad y = -ix.
            \]
            From the $ty = sx$ we then get that the intersection contains $3$ points,
            namely
            \[
                \overline{\pi^{-1}(C \setminus (0, 0))} \cap E
                = \left\{ [1, 0], [1, i], [1, -i] \right\}.
            \]
    \end{enumerate}
\end{Exercise}
 
 \begin{Exercise}
     The curve $C_k \subset \P^2$ is given by
     \[
         y^2z - x(x - z)(x - kz) = 0
     \]
     for a parameter $k \in \C$.
     \begin{enumerate}[label=(\roman*)]
         \item We first find all of the values of $k$ such that $C_k$ is smooth,
             that is, without singularities.

             We are solving the system of equations
             \begin{align*}
                 y^2z - x(x-z)(x-kz) &= 0 \\
                 (x-z)(x-kz) + x(x-kz) + x(x-z) &= 0 \\
                 yz &= 0 \\
                 y^2 + x(x-kz) + kx(x-z) &= 0
             \end{align*}
             Note that this curve clearly has a singularity in $(0, 0, 0)$, but
             this does not interest us, since we are looking at it in the
             projective space.  We follow the third equation and suppose $z=0$.
             Then from the first equation $x=0$ and from the fourth $y=0$, so
             this yields nothing.  Now suppose $y = 0$.  If $x = z$, then $x =
             kz$, hence $k=1$.  If $x \neq z$, then $kz =(2-k)x$.  If $k \neq
             0$, then it follows that either $k=1$ or $x = z = 0$, both leading
             nowhere.  The only case to check is $k=0$, which gives us a
             solution for all $z \in \C$.  We have therefore obtained that
             $C_0$ has a singularity in $[0,0,1]$ and $C_1$ has a singularity
             in $[1,0,1]$.  For all other values of $k$ the curve $C_k$ is
             smooth in $\P^2$.

             From now on assume that $C_k$ is smooth, that is, $k \neq 0, 1$.
         \item Let the map $\Phi \colon \P^2 \to \P^2$ be given by $[x, y, z] \mapsto [x, -y, z]$.
             Notice that this map has the property $\Phi(C_k) \subseteq C_k$,
             since $C_k$ is purely quadratic in $y$.  Let us show that there
             exist $p, q, r, s \in C_k$ such that they are all fixed by $\Phi$.

             By (3.i) we immediately get three points
             \[
                 p = [0, 0, 1], \quad\quad q = [1, 0, 1], \quad\quad r = [k, 0, 1],
             \]
             which are clearly fixed by $\Phi$.
             But the projective transformation $\Phi$ is of course equivalently
             given by $[x, y, z] \mapsto [-x, y, -z]$, which yields the fourth
             point
             \[
                 s = [0, 1, 0]
             \]
             also on the curve $C_k$.
         \item We will now prove that the automorphisms of $\P^1$ are given by
             \[
                 [t,s] \mapsto [at+bs, ct+ds]
             \]
             and write down the condition on $a, b, c, d \in \C$.

             We recall that $\P^1$ is actually the Riemann sphere $\C\P^1$.
             Indeed, denote $\infty := [0, 1]$, then identify $[t,s] \idtfy [1,\frac{s}{t}]$.
             Automorphisms of $\P^1$ must then correspond to automorphisms of $\C\P^1$,
             which are precisely non-degenerate M\"obius transformations, that is,
             functions of the form
             \[
                 z \mapsto \frac{cz + d}{az + b}
             \]
             where $ad - bc \neq 0$.
             Since $z \idtfy \frac{s}{t}$, multiplying by $t$ yields the desired result.

             If this feels like cheating, we can approach the problem more
             algebraically.  As above denote $\infty := [0, 1]$ and consider
             the embedding $\C = U_0 \to \P^1$ via $s \mapsto [1, s]$.  Suppose
             $f$ is an automorphism of $\P^1$ and assume without loss of
             generality that $f(\infty) = \infty$. Indeed, once we have our
             automorphisms, if $f(\infty) = a = [1, a]$, consider the
             automorphism $g \colon [t, s] \mapsto [t - as, s]$.  Then $(g\circ
             f)(\infty) = \infty$, so the assumption really preserves
             generality.
             Consider now the restriction $f_0 = f|_{U_0}$.
             Since $f$ is an automorphism and hence bijective, $f_0$ is a polynomial bijection $\C \to \C$.
             But by the Fundamental Theorem of Algebra, $f_0$ must have degree $1$,
             say $f_0(s) = \alpha + \beta s$.
             Repeat this for $f_1 = f|_{U_1}$ and obtain $f_1(t) = \gamma t + \delta$.
             To go back to the projective setting, we need to merely homogenize the above polynomials
             with appropriate variables and we obtain that $f$ must be of the form
             \[
                 [t, s] \mapsto [ at + bs, ct + ds]
             \]
             The relation on the complex coefficients is also obtained locally.
             If we restrict the above to $U_1$ and without loss of generality assume $s = 1$,
             then we obtain that
             \[
                 [t, 1] \mapsto \frac{a t + b}{ct + d}
             \]
             must be an automorphism of $\C$, so we get the condition from above $ad - bc \neq 0$.
         \item Next we show that every automorphism of $\P^1$ with more than $2$ fixed points
             is equal to the identity.

             One solution is again from Complex Analysis.
             Since automorphisms of $\P^1$ correspond to automorphisms of $\C\P^1$,
             we get that
             \[
                 \frac{az + b}{cz + d} = z
                 \iff az + b = cz^2 + dz
                 \iff cz^2 + (d-a)z - b = 0
             \]
             which is a degree $2$ polynomial and hence has at most $2$ different zeros.

             Algebraically, we know that $[t, s]$ is a fixed point of an automorphism $f$ of $\P^1$,
             if
             \begin{align*}
                 a t + b s &= \lambda t \\
                 c t + d s &= \lambda s
             \end{align*}
             From Linear Algebra, we know that $\lambda$ must be some eigenvalue of
             \[
                 \begin{bmatrix}
                     a & b \\
                     c & d
                 \end{bmatrix}
             \]
             and the solutions must be its eigenvectors.
             Therefore we can have at most $2$ projective solutions,
             unless $f = id$.
             Furthermore, since by (3.iii) this matrix is non-degenerate,
             we always get precisely $2$ solutions.
             %Algebraically, let $[t_0, s_0]$ and $[t_1, s_1]$ be fixed points of
             %a generic automorphism $f$ of $\P^1$.
             %Then
             %\begin{align*}
                 %(a-1)t_0 + bs_0 &= 0 \\
                 %ct_0 + (d-1)s_0 &= 0 \\
                 %(a-1)t_1 + bs_1 &= 0 \\
                 %ct_1 + (d-1)s_1 &= 0
             %\end{align*}
             %is a non-degenerate system of linear equations that uniquely determines the coefficients $a, b, c, d$.
             %Any other fixed point will then either violate the system or already be one of the solutions,
             %except if $f = id$.
         \item Lastly, let us show that $C_k$ is not a rational curve.

             This is of course just a consequence of (3.ii-iv).
             Indeed, $C_k$ is by definition rational if there exists a birational map
             \[
                 \varphi \colon \P^1 \to C_k
             \]
             which is equivalent to them containing open dense $\varphi$-isomorphic subsets.
             But this is a problem, since $\Phi$ from (3.ii) induces an automorphism on $C_k$ with $4$ fixed points.
             Let $U \subseteq \P^1$ and $V \subseteq C_k$ be some $\varphi$-isomorphic dense open sets and consider

             \adjustbox{scale=1, center}{
                 \begin{tikzcd}
                     U \arrow[r, "f"] \arrow[d, "\varphi"] & U \arrow[d, "\varphi"] \\
                     V \arrow[r, "\Phi"] & V
                 \end{tikzcd}
             }
             where $f$ is such a function that makes the above diagram commute.
             By density and since compositions of morphisms are again
             morphisms, $f$ is induced by some automorphism on $\P^1$ (also
             denoted by $f$).  But then $f$ has more than $2$ fixed points,
             hence $f = id$, but $\Phi \neq id$, a contradiction.  Thus, the
             curve $C_k$ cannot be rational.

             Alternatively, we can recall a theorem from Tutorials,
             stating that any birational equivalence between smooth curves is an isomorphism.
             Then we clearly get the induced automorphism $f$ given by

             \adjustbox{scale=1, center}{
                 \begin{tikzcd}
                     \P^1 \arrow[r, "f"] \arrow[d, "\varphi"] & \P^1 \arrow[d, "\varphi"] \\
                     C_k \arrow[r, "\Phi"] & C_k
                 \end{tikzcd}
             }
             which clearly has $4$ fixed points, implying $f = id$, contradicting $\Phi \neq id$.
     \end{enumerate}
 \end{Exercise}

\end{document}
