\documentclass[a4paper, 12pt]{article}

\usepackage[english]{babel}
\usepackage[margin=0.5in]{geometry}

\usepackage[utf8]{inputenc}
\usepackage[T1]{fontenc}
\usepackage{lmodern}
\usepackage{units}
\usepackage{eurosym}
\usepackage{titlesec}
\usepackage{graphicx}
\usepackage{wrapfig}
\usepackage{color}
%\usepackage{url}
\usepackage{hyperref}
\usepackage{enumerate}
\usepackage{enumitem}
\usepackage{pifont}
\usepackage[normalem]{ulem}

% packages
\usepackage{amsmath}
\usepackage{amssymb}
\usepackage{amsthm}
\usepackage{amsfonts}
\usepackage{mathtools}
\usepackage{tikz-cd}
\usetikzlibrary{babel}
\usepackage{adjustbox}
\usepackage{stmaryrd}

% commonly used math operators
\DeclareMathOperator{\diam}{diam}
\DeclareMathOperator{\diag}{diag}
\DeclareMathOperator{\rank}{rank}
\DeclareMathOperator{\tr}{tr}
\DeclareMathOperator{\im}{im}
\DeclareMathOperator{\dom}{dom}
\DeclareMathOperator{\coker}{coker}
\DeclareMathOperator{\codim}{codim}
\DeclareMathOperator{\pr}{pr}
\DeclareMathOperator{\rad}{rad}
\DeclareMathOperator{\chrs}{char}
\DeclareMathOperator{\len}{len}
\DeclareMathOperator{\Lin}{Lin}
\DeclareMathOperator{\Ann}{Ann}
\DeclareMathOperator{\Ass}{Ass}
\DeclareMathOperator{\Spec}{Spec}
\DeclareMathOperator{\mSpec}{mSpec}
\DeclareMathOperator{\Quot}{Quot}
\DeclareMathOperator{\Tor}{Tor}
\DeclareMathOperator{\Ext}{Ext}
\DeclareMathOperator{\Hom}{Hom}
\DeclareMathOperator{\End}{End}
\DeclareMathOperator{\Aut}{Aut}
\DeclareMathOperator{\Br}{Br}
\DeclareMathOperator{\Gal}{Gal}

% commonly used math objects
\newcommand{\F}{\mathbb{F}}
\newcommand{\A}{\mathbb{A}}
\newcommand{\D}{\mathbb{D}}
\renewcommand{\S}{\mathbb{S}}
\newcommand{\T}{\mathbb{T}}
\newcommand{\B}{\mathbb{B}}
\newcommand{\I}{\mathbb{I}}
\newcommand{\N}{\mathbb{N}}
\newcommand{\Z}{\mathbb{Z}}
\newcommand{\Q}{\mathbb{Q}}
\newcommand{\R}{\mathbb{R}}
\newcommand{\C}{\mathbb{C}}
\renewcommand{\H}{\mathbb{H}}
\renewcommand{\P}{\mathbb{P}}

% commonly used math relations
\newcommand{\iso}{\cong}
\newcommand{\homeo}{\approx}
\newcommand{\htpeq}{\simeq}
\newcommand{\hlgeq}{\sim}
\newcommand{\idtfy}{\longleftrightarrow}

% commonly used math symbols
\newcommand{\closure}[1]{\overline{#1}}
\newcommand{\subideal}{\vartriangleleft}
\newcommand{\supideal}{\vartriangleright}

% numbered environments
\theoremstyle{plain}
\newtheorem{theorem}{Theorem}[section]
\newtheorem{corollary}[theorem]{Corollary}
\newtheorem{exercise}[theorem]{Exercise}
\newtheorem{lemma}[theorem]{Lemma}
\newtheorem{proposition}[theorem]{Proposition}

\theoremstyle{definition}
\newtheorem{definition}[theorem]{Definition}

\theoremstyle{remark}
\newtheorem*{claim}{Claim}
\newtheorem*{remark}{Remark}

\newcounter{excounter}[section]
\newenvironment{Exercise}
    {\refstepcounter{excounter}\underline{\textbf{Ex. \theexcounter:}}}
    {\par\vspace{\baselineskip}}


\title{Differential Geometry - $2^{\text{nd}}$ homework}
\author{Benjamin Benčina}
\date{\today}

\begin{document}

\maketitle

\begin{Exercise}
    Let $\xi \colon E \to F$ be a smooth morphism of vector bundles $\pi_E \colon E \to M$ and $\pi_F \colon F \to M$ and denote $\xi_p = \xi|_{E_p} \colon E_p \to F_p$.
    We will show that $\ker(\xi)$ and $\im(\xi)$ are vector subbundles of $E$ and $F$, respectively, precisely when the morphism $\xi$ has constant rank.
    \begin{itemize}
        \item \underline{($\implies$):}
            This implication follows directly from the definition of the vector bundle,
            as all fibres must have the same dimension.
            Since the fibres of our subbundles are $(\ker\xi)_p = \ker\xi_p$ and $(\im\xi)_p = \im\xi_p$, which are all of the same dimension,
            then necessarily $\dim\im\xi_p$ is also the same for all $p \in M$.
        \item \underline{($\impliedby$):}
            To prove the converse implication, we will use The Subbundle Characterization Lemma (SCL) from tutorials.
            Firstly, since $\xi_p$ is a vector homomorphism for every $p \in M$,
            we have that $\ker\xi_p$ and $\im\xi_p$ are indeed vector subspaces of $E_p$ and $F_p$, respectively.
            Denote $k = \rank E$ and $l = \rank F$, and assume that $\xi$ has constant rank $r$.

            \underline{\emph{The image:}}

            Take $p \in M$ and a local frame $(\sigma_1,\dots,\sigma_k)$ for the vector bundle $E$ over some neighbourhood $U$ of point $p$.
            Then for all $i = 1,\dots,k$ we have that $\xi\circ\sigma_i \colon U \to F$ are a smooth local sections of $F$ that span $(\im\xi)|_{U}$.
            Since $\dim\im\xi_p = r$, w.l.o.g. the set $\left\{ \xi\circ\sigma_1(p),\dots,\xi\circ\sigma_r(p) \right\}$ is a basis for the vector space $\im\xi_p \leq F_p$,
            and by continuity, these vectors remain linearly independent also in some neighbourhood $U_0$ of $p$ (i.e., replacing $p$ with points in $U_0$).
            We can take $U_0 \subseteq U$, otherwise replace it with $U \cap U_0$.
            Now, since $\xi$ has constant rank $r$, the collection $(\xi\circ\sigma_1,\dots,\xi\circ\sigma_r)$ forms a smooth local frame for $\xi$ over $U_0$.
            Because the point $p$ was chosen arbitrarily, we can do the same at every $p \in M$.
            By SCL, $\im\xi$ is a smooth subbundle of $F$.

            \underline{\emph{The kernel:}}

            Keep $U_0$ and $(\sigma_i)_i$ as above, and denote $V \subseteq E|_{U_0}$ the smooth subbundle spanned by $\left\{ \sigma_1,\dots,\sigma_r \right\}$.
            By the argument for the image, the smooth bundle homomorphism $\xi|_V \colon V \to (\im\xi)|_{U_0}$ is bijective (map $\sigma_i(q) \mapsto \xi\circ\sigma_i(q)$, basis to basis) and thus a smooth vector bundle isomorphism.
            We now define a smooth bundle homomorphism $\psi \colon E|_{U_0} \to E|_{U_0}$ by
            \[
                \psi(v) = v - (\xi|_V)^{-1}\circ\xi(v)
            \]
            If $v \in V$, then $\xi(v) = \xi|_V(v)$, so
            \[
                \xi\circ\psi(v) = \xi(v) - \xi\circ(\xi|_V)^{-1}\circ(\xi|_V)(v) = \xi(v) - \xi(v) = 0
            \]
            If $v \in \ker\xi$, then $\psi(v) = v$, so $\psi|_{\ker\xi} = id$ and again $\xi\circ\psi(v) = \xi(v) = 0$.
            Since $(\ker\xi)|_{U_0}$ and $V$ together span $E|_{U_0}$ (by definitions of linear maps $\xi_p$ for $p \in U_0$),
            it follows that $\psi$ takes values on $(\ker\xi)|_{U_0}$, where it restricts to $id$, so $\im\psi = (\ker\xi)|_{U_0}$.
            Since for every $p$, we have $\dim\im\xi_p = r$ and by linear algebra $\dim\ker\xi_p = k - r$ is also constant,
            the map $\psi$ has constant rank.
            Hence, by the argument for the image, $(\ker\xi)|_{U_0} = \im\psi$ is a smooth submanifold of $E|_{U_0}$.
            Again by SCL, since we can do this at each point $p \in M$, $\ker\xi$ is a smooth vector subbundle of $E$.
    \end{itemize}
\end{Exercise}

\begin{Exercise}
    A basis for the Lie algebra $\mathfrak{sl}(2, \R)$ of the Lie group $SL(2, \R)$ is given by the matrices
    \[
        E =
        \begin{bmatrix}
            0 & 1 \\
            0 & 0
        \end{bmatrix},
        \quad
        F =
        \begin{bmatrix}
            0 & 0 \\
            1 & 0
        \end{bmatrix},
        \quad
        H =
        \begin{bmatrix}
            1 & 0 \\
            0 & -1
        \end{bmatrix}.
    \]
    We will compute the expressions of their left-invariant vector fields $E^L, F^L, H^L$ on $SL(2, \R)$, their commutators and flows.

    Recall from Tutorials that
    \[
        A^L\Big|_x = d(L_x)_I(A) = xA
    \]
    so we get local expressions
    \begin{align*}
        E^L &=
        \begin{bmatrix}
            x & y \\
            z & w
        \end{bmatrix}
        \begin{bmatrix}
            0 & 1 \\
            0 & 0
        \end{bmatrix}
        =
        \begin{bmatrix}
            0 & x \\
            0 & z
        \end{bmatrix}
        = x \partial_y + z \partial_w \\
        F^L &=
        \begin{bmatrix}
            x & y \\
            z & w
        \end{bmatrix}
        \begin{bmatrix}
            0 & 0 \\
            1 & 0
        \end{bmatrix}
        =
        \begin{bmatrix}
            y & 0 \\
            w & 0
        \end{bmatrix}
        = y \partial_x + w \partial_z \\
        H^L &=
        \begin{bmatrix}
            x & y \\
            z & w
        \end{bmatrix}
        \begin{bmatrix}
            1 & 0 \\
            0 & -1
        \end{bmatrix}
        =
        \begin{bmatrix}
            x & -y \\
            z & -w
        \end{bmatrix}
        = x \partial_x - y \partial_y + z \partial_z - w \partial_w
    \end{align*}
    We can then compute the commutators
    \begin{align*}
        [E^L, F^L]
        &= (E^Ly - 0) \partial_x + (0 - F^L(x)) \partial_y + (E^L(w) - 0) \partial_z + (0 - F^L(z)) \partial_w \\
        &= x \partial_x - y \partial_y + z \partial_z - w \partial_w = H^L \\
        [F^L, H^L]
        &= (F^L(x) - H^L(y)) \partial_x + (0 - 0) \partial_y + (F^L(z) - H^L(w)) \partial_z + (0 - 0) \partial_w \\
        &= (y + y) \partial_x + (w + w) \partial_z = 2F^L \\
        [H^L, E^L]
        &= (0 - 0) \partial_x + (H^L(x) - E^L(-y)) \partial_y + (0 - 0) \partial_z + (H^L(z) - E^L(-w)) \partial_w \\
        &= (x + x) \partial_y + (z + z) \partial_w ) = 2E^L
    \end{align*}
    but we could have done this more elegantly by using the fact that the Lie bracket corresponds to the usual commutator in matrix groups.
    Indeed, we get
    \begin{align*}
        [E, F] &= H \\
        [F, H] &= 2F \\
        [H, E] &= 2E
    \end{align*}
    as matrices.
    Regarding the flows, recall from Tutorials that
    \[
        \phi_t^{A^L} = R_{e^{tA}}
    \]
    We will therefore consider the exponentials of our matrices $E, F, H$.
    We calculate
    \begin{align*}
        E^2 &= 0 \\
        F^2 &= 0 \\
        H^2 &= I
    \end{align*}
    so we get
    \begin{align*}
        e^{tE}
        &= I + tE=
        \begin{bmatrix}
            1 & t \\
            0 & 1
        \end{bmatrix} \\
        e^{tF}
        &= I + tF =
        \begin{bmatrix}
            1 & 0 \\
            t & 1
        \end{bmatrix} \\
        e^{tH}
        &= I + tH + \frac{t^2}{2}I + \frac{t^3}{6}H + \cdots =
        \begin{bmatrix}
            e^t & 0 \\
            0 & e^{-t}
        \end{bmatrix}
    \end{align*}
    and hence the flows of our left-invariant vector fields equal right multiplications with the respective matrices above.
\end{Exercise}

\begin{Exercise}
    Let $G$ be a matrix Lie group, i.e. a closed subgroup and an embedded submanifold of $GL(n, \F)$.
    \begin{enumerate}[label=(\roman*)]
        \item A one-parametric subgroup of $G$ is a smooth homomorphism $\alpha \colon \R \to G$ of Lie groups.
            \begin{enumerate}[label=(\alph*)]
                \item We show that given $v \in \mathfrak{g}$, $t \mapsto e^{tv}$ is a one-parametric subgroup of $G$.
                    
                    Recall from Tutorials that the map $t \mapsto e^{tA}$ is smooth as a map $\R \to GL(n, \F)$.
                    Since $G$ is an embedded submanifold in $GL(n, \F)$, we know from Homework 1 that $t \mapsto e^{tv}$ is smooth as a map $\R \to G$.
                    Denote the above map $\varphi_v$.
                    Then clearly $v = \frac{d\varphi_v}{dt}(0) \in \mathfrak{g}$, so the map is well-defined.
                    Since $e^{(t + s)v} = e^{tv}e^{sv}$, the map $\varphi_v$ is a smooth homomorphism as a map $\R \to G$, and hence a one-parametric subgroup of $G$.
                \item Conversely, given a one-parametric subgroup $\alpha$ of $G$, we show that there holds
                    \[
                        \alpha(t) = e^{t\alpha'(0)}.
                    \]
                    Indeed, we calculate
                    \[
                        \frac{d}{dt}\Big|_{t=s} \alpha(t) = \frac{d}{dt}\Big|_{r=0}\alpha(s+r) = \frac{d}{dt}\Big|_{r=0}\alpha(s)\alpha(r) = \alpha(s)\alpha'(0)
                    \]
                    so if we define $\gamma_x(t) = x\alpha(t)$, then
                    \[
                        \gamma_x'(s) = \gamma_x(s)\alpha'(0) = \alpha'(0)^L\Big|_{\gamma_x(s)}
                    \]
                    Hence, $\gamma_x(t)$ is an integral curve for $\alpha'(0)^L$.
                    We know from Tutorials that also $xe^{t\alpha'(0)}$ is an integral curve for $\alpha'(0)^L$.
                    By uniqueness, it follows that $\alpha(t) = e^{t\alpha'(0)}$.
            \end{enumerate}
        \item Let $H$ be another matrix Lie group and let $\phi \colon G \to H$ be a homomorphism of Lie groups.
            We show that the diagram
            
            \adjustbox{scale=1, center}{
                \begin{tikzcd}
                    G \arrow[r, "\phi"] & H \\
                    \mathfrak{g} \arrow[u, "\text{exp}"] \arrow[r, "d\phi_I"] & \mathfrak{h} \arrow[u, "\text{exp}"]
                \end{tikzcd}
            }
            commutes.
            We define
            \[
                \alpha(t) = \phi(e^{tv})
            \]
            for $v \in \mathfrak{g}$.
            By (3i) and since $\phi$ is a Lie group homomorphism, $\alpha$ is a one-parametric subgroup in $H$.
            Then, by (3i.b), we get
            \[
                \phi(e^{tv}) = e^{t\alpha'(0)}
            \]
            We calculate
            \[
                \alpha'(0) = \frac{d}{dt}\Big|_{t=0}\phi(e^{tv}) = d\phi_I\circ (e^{tv})'(0) = d\phi_I v
            \]
            By inserting $t = 1$ in the above equation, we get
            \[
                \phi(e^{v}) = e^{\alpha'(0)} = e^{d\phi_Iv}
            \]
            which is what we wanted to prove.

            We now use this result on the conjugation map $C_g \colon G \to G$, $C_g(h) = ghg^{-1}$, where $g \in G$.
            We get
            \[
                e^{d(C_g)_Iv}) = ge^{v}g^{-1}
            \]
            We calculate
            \begin{align*}
                d(C_g)_I v &= \frac{d}{dt}\Big|_{t=0}C_g(I + tv) = \frac{d}{dt}\Big|_{t=0}g(I+tv)g^{-1} \\
                &= \frac{d}{dt}\Big|_{t=0}(gIg^{-1} + tgvg^{-1}) = gvg^{-1} = C_g(v)
            \end{align*}
            and thusly obtain
            \[
                e^{gvg^{-1}} = ge^{v}g^{-1}
            \]
        \item Let $\phi \colon G \to H$ be a homomorphism of matrix Lie groups.
            Let us show that $\phi$ is an immersion $\iff$ $\ker\phi$ is discrete.
            \begin{itemize}
                \item \underline{($\implies$):}
                    Since $\phi$ is an immersion, for every $g \in G$ we have that $d\phi_g$ is injective.
                    In particular, the function $d\phi_I$ is injective and so bijective onto its image.
                    Now take $I \in \ker\phi$.
                    We know that $\phi$ is a local embedding at every point,
                    so in particular there exists a neighbourhood $U_I$ of $I$ such that $U_I \homeo \phi(U_I) \subseteq H$.
                    Also, since $\text{exp}$ is a local diffeomorphism at $0 \in \mathfrak{g}$,
                    there exists a neighbourhood $V$ of $0$ in $\mathfrak{g}$ such that $V \iso \text{exp}(V)$.
                    Note that $I \in \phi(U_I)$ and $I \in \text{exp}(V)$.
                    Let now $W = U_I \cap \text{exp}(V)$ be a neighbourhood of $I$ in $G$.

                    \adjustbox{scale=1, center}{
                        \begin{tikzcd}
                            W \arrow[r, "\phi"] & \phi(W) \\
                            \text{exp}^{-1}(W) \arrow[u, "\text{exp}"] \arrow[r, "d\phi_I"] & \text{exp}^{-1}(\phi(W)) \arrow[u, "\text{exp}"]
                        \end{tikzcd}
                    }
                    Since the diagram commutes and the left, right, and top arrows are bijective maps,
                    so is the bottom one, i.e., the set in the bottom-right corner is indeed in the image of $d\phi_I$.
                    It now follows from the commutativity of the above diagram that $W \cap \ker\phi = \lbrace I \rbrace$.
                    Since $L_g$ is a diffeomorphism of $G$ by the definition of a Lie group,
                    $L_p(W)$ is a neighbourhood for $p \in \ker\phi$ with the property $L_p(W) \cap \ker\phi = \lbrace p \rbrace$.
                    Hence every point of $\ker\phi$ is open in $\ker\phi$, so $\ker\phi$ is discrete.
                \item \underline{($\impliedby$):}
                    Let us first prove that $d\varphi_g$ is injective for a $g \in G$ precisely when $d\varphi_I$ is injective.
                    Indeed, consider the following diagram

                    \adjustbox{scale=1, center}{
                        \begin{tikzcd}
                            T_IG \arrow[r, "d\phi_I"] \arrow[d, "d(L_g)_I"] & T_IH \arrow[d, "d(L_{\phi(g)})_I"] \\
                            T_gG \arrow[r, "d\phi_g"] & T_{\phi(g)}H
                        \end{tikzcd}
                    }
                    where both vertical arrows are vector space isomorphisms.
                    This diagram commutes because of the chain rule:
                    \begin{align*}
                        &d(L_{\phi(g)})_I \circ d\phi_I = d(L_{\phi(g)})_{\phi(I)}\circ d\phi_I = d(L_{\phi(g)} \circ \phi)_I \\
                        & d\phi_g \circ d(L_g)_I = d\phi_{L_g(I)} \circ d(L_g)_I = d(\phi \circ L_g)_I
                    \end{align*}
                    which are equal since $\phi$ is a homomorphism, i.e., $\phi(gx) = \phi(g)\phi(x)$.
                    
                    So, it is enough to prove that $d\phi_I$ is injective.
                    Take $I \in \ker\phi$ and let $U_I$ be a neighbourhood of $I$ such that $\ker\phi \cap U_I = \lbrace I \rbrace$.
                    Since $\text{exp}$ is a local diffeomorphism at $0$, there exists a neighbourhood $V$ for $0$ in $\mathfrak{g}$ such that $V \iso \text{exp}(V)$.
                    Let $W = U_I \cap \text{exp}(V)$ be an open neighbourhood of $I$ in $G$.
                    Now, suppose for contradiction that $\dim\ker d\phi_I > 0$.
                    Then there exists a $1$-dimensional vector subspace $A$ in $\mathfrak{g}$ with $A \leq \ker d\phi_I$.
                    Denote $A_0 = A \cap \text{exp}^{-1}(W)$.
                    Then for every point $a \in A_0$ (and there is infinitely many of them) we get
                    \[
                        \text{exp}(d\phi_I(a)) = \text{exp}(0) = I
                    \]
                    but by (3ii) also
                    \[
                        \text{exp}(d\phi_I(a)) = \phi(\text{exp}(a)) = I
                    \]
                    In other words, each element $a \in A_0$ gives us an element $\text{exp}(a) \in \ker\phi$ with $\text{exp}(a) \in U_I$,
                    a contradiction with the choice of $U_I$.

                    Hence, $\dim\ker d\phi_I = 0$, i.e., $d\phi_I$ is injective and by above so is $d\phi_g$ for every $g \in G$.
                    By definition, $\phi$ is an immersion.
            \end{itemize}
    \end{enumerate}
\end{Exercise}

\begin{Exercise}
    Let $\sigma_1, \sigma_2, \sigma_3$ be the Pauli matrices and denote $\sigma(\vec{s}) = is^j\sigma_j$ for $\vec{s} \in \R^3$.
    \begin{enumerate}[label=(\roman*)]
        \item Let us show that
            \[
                e^{t\sigma(\vec{s})} = \cos(t)I + \sin(t)\sigma(\vec{s})
            \]
            for all $\vec{s} \in \S^2$.

            We simply calculate
            \begin{align*}
                \sigma(\vec{s})^2
                &= \left( i \sum_{j=1}^{3}s_j\sigma_j \right)^2
                = -\left( (s_1^2 + s_2^2 + s_3^3)I + s_1s_2\left\{ \sigma_1,\sigma_2 \right\} + s_2s_3\left\{ \sigma_2,\sigma_3 \right\} + s_1s_3\left\{ \sigma_3,\sigma_1 \right\} \right) \\
                &= -(I + 0 + 0 + 0) = -I
            \end{align*}
            Furthermore, we get $\sigma(\vec{s})^{2n} = (-1)^nI$.
            It then follows that $\sigma(\vec{s})^{2n+1} = (-1)^n\sigma(\vec{s})$.
            Therefore
            \begin{align*}
                e^{t\sigma(\vec{s})}
                &= I + t\sigma(\vec{s}) + \frac{t^2}{2}\sigma(\vec{s})^2 + \frac{t^3}{6}\sigma(\vec{s})^3 + \cdots \\
                &= I\cdot\left( 1 - \frac{t^2}{2} + \frac{t^4}{24} - \cdots \right) + \sigma(\vec{s}) \cdot \left( t - \frac{t^3}{6} + \frac{t^5}{120} - \cdots \right) \\
                &= I \cdot \cos(t) + \sigma(\vec{s}) \cdot \sin(t)
            \end{align*}
        \item Let us now prove that for every $\vec{s} \in \R^3$ we have
            \[
                e^{\sigma(\vec{s})} \in SU(2) = \left\{ A \in \C^{2\times 2} ; \; AA^* = A^*A = I, \; \det A = 1 \right\}
            \]
            Firstly, we know from algebra that $(e^{A})^* = e^{A^*}$,
            so we want to show that $\sigma(\vec{s})^* = -\sigma(\vec{s})$
            We write
            \[
                \sigma(\vec{s}) =
                \begin{bmatrix}
                    0 & is_1 \\
                    is_1 & 0
                \end{bmatrix}
                +
                \begin{bmatrix}
                    0 & s_2 \\
                    -s_2 & 0
                \end{bmatrix}
                +
                \begin{bmatrix}
                    is_3 & 0 \\
                    0 & -is_3
                \end{bmatrix}
            \]
            and calculate
            \[
                \sigma(\vec{s})^*
                =
                \begin{bmatrix}
                    0 & -is_1 \\
                    is_1 & 0
                \end{bmatrix}
                +
                \begin{bmatrix}
                    0 & -s_2 \\
                    s_2 & 0
                \end{bmatrix}
                +
                \begin{bmatrix}
                    -is_3 & 0 \\
                    0 & is_3
                \end{bmatrix}
                = -\sigma(\vec{s})
            \]
            Secondly, the Jacobi formula for $2\times 2$ complex matrices gives us
            \[
                \det(e^{A}) = e^{\tr(A)}
            \]
            so we get
            \[
                \det(e^{\sigma(\vec{s})}) = e^{\tr(\sigma(\vec{s}))} = e^{is_3 - is_3} = e^{0} = 1
            \]
            which proves the statement.
        \item We define the map $\pi \colon SU(2) \to SO(3)$ by
            \[
                \pi\left( e^{\frac{\varphi}{2}\sigma(\vec{s})} \right) = R_{\vec{s},\varphi}
            \]
            for $\vec{s} \in \S^2$ and $\varphi \in \R$.
            This map is smooth and surjective. We will show that
            \[
                \ker\pi = \left\{ I, -I \right\}
            \]
            Let us extend this map as
            \[
                \S^2 \times \R \xrightarrow{(4ii)} SU(2) \xrightarrow{\pi} SO(3), \; (\vec{s}, \varphi) \mapsto R_{\vec{s},\varphi}
            \]
            (the first map is actually slightly modified as $(\vec{s}, \varphi) \mapsto e^{\frac{\varphi}{2}\sigma(\vec{s})}$).
            Since both of these maps are surjective, so is their composition.
            We get the $I$ on the right for every $\vec{s} \in \S^2$ precisely when $\varphi = 2\pi k$ for some $k \in \Z$.
            In the middle, we get that for elements in the kernel we have
            \[
                e^{\frac{\varphi}{2}\sigma(\vec{s})}
                = e^{\pi k \sigma(\vec{s})}
                \overset{(4i)}{=} \cos(\pi k) I + \sin(\pi k) \sigma(\vec{s}) = \pm I
            \]
            
            Now, since $\ker \pi$ is finite and hence discrete, $\pi$ is by (3iii) an immersion.
            Then $\pi$ is locally an embedding, so a local homeomorphism.
            Since $\pi$ is smooth, it is a local diffeomorphism.
            Alternatively, since $\pi$ is an immersion, all derivative maps are injective vector homomorphisms between tangent spaces.
            But since the dimensions of the two manifolds match, the derivative maps are also surjective and hence vector isomorphisms.
            By the characterization theorem, $\pi$ is a local diffeomorphism.
    \end{enumerate}
\end{Exercise}

\end{document}
