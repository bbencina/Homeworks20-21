\documentclass[a4paper, 12pt]{article}

\usepackage[slovene]{babel}
\usepackage[margin=0.5in]{geometry}

\usepackage[utf8]{inputenc}
\usepackage[T1]{fontenc}
\usepackage{lmodern}
\usepackage{units}
\usepackage{eurosym}
\usepackage{graphicx}
\usepackage{wrapfig}
\usepackage{color}
%\usepackage{url}
\usepackage{hyperref}
\usepackage{enumerate}
\usepackage{enumitem}
\usepackage{pifont}
\usepackage[normalem]{ulem}

% packages
\usepackage{amsmath}
\usepackage{amssymb}
\usepackage{amsthm}
\usepackage{amsfonts}
\usepackage{mathtools}
\usepackage{tikz-cd}
\usetikzlibrary{babel}
\usepackage{adjustbox}
\usepackage{stmaryrd}

% commonly used math operators
\DeclareMathOperator{\diam}{diam}
\DeclareMathOperator{\diag}{diag}
\DeclareMathOperator{\rank}{rank}
\DeclareMathOperator{\tr}{tr}
\DeclareMathOperator{\im}{im}
\DeclareMathOperator{\dom}{dom}
\DeclareMathOperator{\coker}{coker}
\DeclareMathOperator{\codim}{codim}
\DeclareMathOperator{\pr}{pr}
\DeclareMathOperator{\rad}{rad}
\DeclareMathOperator{\chrs}{char}
\DeclareMathOperator{\len}{len}
\DeclareMathOperator{\Lin}{Lin}
\DeclareMathOperator{\Ann}{Ann}
\DeclareMathOperator{\Ass}{Ass}
\DeclareMathOperator{\Spec}{Spec}
\DeclareMathOperator{\mSpec}{mSpec}
\DeclareMathOperator{\Quot}{Quot}
\DeclareMathOperator{\Tor}{Tor}
\DeclareMathOperator{\Ext}{Ext}
\DeclareMathOperator{\Hom}{Hom}
\DeclareMathOperator{\End}{End}
\DeclareMathOperator{\Aut}{Aut}
\DeclareMathOperator{\Br}{Br}
\DeclareMathOperator{\Gal}{Gal}

% commonly used math objects
\newcommand{\F}{\mathbb{F}}
\newcommand{\A}{\mathbb{A}}
\newcommand{\D}{\mathbb{D}}
\renewcommand{\S}{\mathbb{S}}
\newcommand{\T}{\mathbb{T}}
\newcommand{\B}{\mathbb{B}}
\newcommand{\I}{\mathbb{I}}
\newcommand{\N}{\mathbb{N}}
\newcommand{\Z}{\mathbb{Z}}
\newcommand{\Q}{\mathbb{Q}}
\newcommand{\R}{\mathbb{R}}
\newcommand{\C}{\mathbb{C}}
\renewcommand{\H}{\mathbb{H}}
\renewcommand{\P}{\mathbb{P}}

% commonly used math relations
\newcommand{\iso}{\cong}
\newcommand{\homeo}{\approx}
\newcommand{\htpeq}{\simeq}
\newcommand{\hlgeq}{\sim}
\newcommand{\idtfy}{\longleftrightarrow}

% commonly used math symbols
\newcommand{\closure}[1]{\overline{#1}}
\newcommand{\subideal}{\vartriangleleft}
\newcommand{\supideal}{\vartriangleright}

% numbered environments
\theoremstyle{plain}
\newtheorem{theorem}{Theorem}[section]
\newtheorem{corollary}[theorem]{Corollary}
\newtheorem{exercise}[theorem]{Exercise}
\newtheorem{lemma}[theorem]{Lemma}
\newtheorem{proposition}[theorem]{Proposition}

\theoremstyle{definition}
\newtheorem{definition}[theorem]{Definition}

\theoremstyle{remark}
\newtheorem*{claim}{Claim}
\newtheorem*{remark}{Remark}

\newcounter{excounter}[section]
\newenvironment{Exercise}
    {\refstepcounter{excounter}\underline{\textbf{Ex. \theexcounter:}}}
    {\par\vspace{\baselineskip}}


\title{Differential Geometry - $4^{\text{th}}$ homework}
\author{Benjamin Benčina}
\date{\today}

\begin{document}

\maketitle

\begin{Exercise}
    Let $\nabla$ be a covariant derivative on $TM$.
    % Lee - Riemannian pp.95-97
    \begin{enumerate}[label=(\roman*)]
        \item Let us show that the map $\nabla \colon \mathfrak{X}(M)\times\Omega^1(M) \to \Omega^1(M)$,
            given by
            \[
                (X, \omega) \mapsto (\nabla_X\omega)(Y) = X(\omega(Y)) - \omega(\nabla_XY)
            \]
            determines a covariant derivative on $T^*M$.

            We first check that the defining expression is $C^\infty(M)$-multilinear in $X$.
            Take $f_1, f_2 \in C^(M)$ and $X_1, X_2 \in \mathfrak{X}(M)$ and calculate
            \begin{align*}
                (\nabla_{f_1X_1 + f_2X_2}\omega)(Y)
                &= (f_1X_1 + f_2X_2)(\omega(Y)) - \omega(\nabla_{f_1X_1 + f_2X_2}Y) \\
                &= f_1X_1(\omega(Y)) + f_2X_2(\omega(Y)) - \omega(f_1\nabla_{X_1}Y + f_2\nabla_{X_2}Y) \\
                &= f_1X_1(\omega(Y)) - f_1\omega(\nabla_{X_1}Y) + f_2X_2(\omega(Y)) - f_2\omega(\nabla_{X_2}Y) \\
                &= f_1(\nabla_{X_1}\omega)(Y) + f_2(\nabla_{X_2}\omega)(Y)
            \end{align*}
            Next, we check $\R$-linearity in $\omega$.
            Take $a_1, a_2 \in \R$ and $\omega_1, \omega_2 \in \Omega^1(M)$ and calculate
            \begin{align*}
                (\nabla_X(a_1\omega_1 + a_2\omega_2))(Y)
                &= X((a_1\omega_1 + a_2\omega_2)(Y)) - (a_1\omega_1 + a_2\omega_2)(\nabla_XY) \\
                &= a_1X(\omega_1(Y)) - a_1\omega_1(\nabla_XY) + a_2X(\omega_2(Y)) - a_2\omega_2(\nabla_XY) \\
                &= a_1(\nabla_X\omega_1)(Y) + a_2(\nabla_X\omega_2)(Y)
            \end{align*}
            We need to check the expression satisfies the product rule.
            Take $f \in C^\infty(M)$ and calculate
            \begin{align*}
                (\nabla_X(f\omega))(Y)
                &= X(f\omega(Y)) - f\omega(\nabla_XY) \\
                &= X(f)\omega(Y) + fX(\omega(Y)) - f\omega(\nabla_XY) \\
                &= X(f)\omega(y) + f(\nabla_X\omega)(Y)
            \end{align*}
            Lastly, we want to see that this formula is $C^\infty(M)$-linear in $Y$.
            Indeed, the addition part is trivial, and the new terms that arise due to the Leibniz and product rule, respectively, cancel out.
            \begin{align*}
                (\nabla_X\omega)(fY)
                &= X(f\omega(Y)) - \omega(\nabla_X(fY)) \\
                &= X(f)\omega(Y) + fX(\omega(Y)) - \omega(f\nabla_XY + X(f)Y) \\
                &= X(f)\omega(Y) + fX(\omega(Y)) - f\omega(\nabla_XY) - X(f)\omega(Y) \\
                &= f(\nabla_X\omega)(Y)
            \end{align*}
            
            We will now prove that in any local coordinates $(x^i)$,
            the $1$-form $\nabla_X\omega$ is given by
            \[
                \nabla_X\omega = (X(\omega_k) - X^i\omega_j\Gamma_{ik}^{j})dx^k
            \]
            
            Let $(\partial_i)$ be the frame associated to $(x^i)$ and let $(dx^i)$ be its dual frame.
            We know that $\Gamma_{ik}^{j}$ are the coefficients of $\nabla$ (on $TM$) in this frame.
            We write $X = X^i\partial_i$, $\omega = \omega_jdx^j$, and $\nabla_{\partial_i}\partial_j = \Gamma_{ij}^{k}\partial_k$.
            We then get
            \begin{align*}
                (\nabla_X\omega)(\partial_k)
                &= X(\omega(\partial_k)) - \omega(\nabla_X\partial_k) \\
                &= X(\omega_k) - \omega(X^i\Gamma_{ik}^{j}\partial_j) \\
                &= X(\omega_k) - X^i\omega_j\Gamma_{ik}^j
            \end{align*}
            Repeat this for all $\partial_k$ and use that $dx^i(\partial_j) = \delta_{ij}$, then extend by linearity.
        \item We now explain why the map $\mathfrak{X}(M)\times\Gamma^\infty(T^{(k,l)}TM) \to \Gamma^\infty(T^{(k,l)}TM)$,
            given by
            \begin{align*}
                (\nabla_XA)(\omega^1,\dots,\omega^k,Y_1,\dots,Y_l)
                &= X(A(\omega^1,\dots,\omega^k,Y_1,\dots,Y^l)) \\
                &- \sum_{i = 1}^{k}A(\omega^1,\dots,\nabla_X\omega^i,\dots,\omega^k,Y_1,\dots,Y_l) \\
                &- \sum_{j = 1}^{l} A(\omega^1,\dots,\omega^k,Y_1,\dots,\nabla_XY_j,\dots,Y_l)
            \end{align*}
            determines a covariant derivative on $T^{(k,l)}TM$.

            Indeed, this is nothing but a generalization of (1i).
            For $C^\infty(M)$-multilinearity in $\omega$'s and $Y$'s,
            we merely insert $f\omega^i$ or $fY_j$, and use (1i) in the former case.
            To prove the three defining properties of a connection,
            we also proceed as above and use $(1i)$ whenever we are dealing with $\nabla_X\omega^i$.
    \end{enumerate}
\end{Exercise}

\begin{Exercise}
    Let $(M, g)$ be a Riemannian manifold.
    \begin{enumerate}[label=(\alph*)]
        \item Let us first prove that the map $\overline{g} \colon \mathfrak{X}(M) \to \Omega^1(M)$,
            given by $\overline{g}(X) = g(X, \cdot)$ is an isomorphism of vector spaces,
            which induces an isomorphism of vector bundles $TM$ and $T^*M$.

            Indeed, this is easy as $g$ provides an inner product at every point.
            It then immediately follows that $\overline{g}$ is a homomorphism, since $g$ is linear in the first component.
            The map $\overline{g}$ is then uniquely determined by where it maps the basis elements of any local $(v_i)$.
            Since clearly $\ker\overline{g} = 0$, by dimension count, $\overline{g}$ is bijective and hence a vector space isomorphism at every point.
            Since $g$ locally operates on frames, $\overline{g}$ extends to an isomorphism of vector bundles.
            Take now a local frame $(E_i)$ and denote by $(\varepsilon^j)$ its dual frame.
            We get the local expression
            \[
                \overline{g}(X) = (g_{ij}X^i)\varepsilon^j
            \]
            where $[g_{ij}]$ is the matrix of $g$.
            If we denote by $[g^{ij}]$ its inverse matrix (the inverse of the map will clearly be given by the inverse of the matrix for $g$),
            we get the local expression for the inverse
            \[
                \overline{g}^{-1}(\omega) = (g^{ij}\omega_j)E_i
            \]
        \item We now show that the definition of the trace of a $(1,1)$-tensor field is independent of the chosen coordinates.

            Let $(E_i)$ and $(F_i)$ be two local frames given by two sets of local coordinates around a point $p \in M$,
            and let $(\varepsilon^j)$ and $(f^j)$ be their dual frames, respectively.
            Write locally $A = A_j^iE_i\otimes\varepsilon^j = B_j^iF_i\otimes f^j$.
            For every sensible point $p$ we would like to then have
            \[
                \sum_{i = 1}^{n} A_i^i(p) = \sum_{i = 1}^{n}B_i^i(p)
            \]
            but this is of course true, since the trace is independent of the basis in any (finite dimensional) vector space.
    \end{enumerate}
    Now, let $\nabla$ denote the Levi-Civita connection on $(M, g)$.
    \begin{enumerate}[label=(\roman*)]
        \item We define the \emph{gradient} of a function $f \in C^\infty(M)$ as the vector field
            \[
                \nabla f = \overline{g}^{-1}(df)
            \]
            Let us express $\nabla f$ in local coordinates.

            Let $(x^i)$ be some local coordinates on $M$.
            We know that
            \[
                df = (\partial_i f) dx^i
            \]
            so we get
            \[
                \nabla f = (g^{ij}(\partial_j f)) \partial_i
            \]
            
            We would also like to express $\nabla f$ in spherical coordinates on $\R^3$ for $f \in C^\infty(\R^3)$.
            Recall from Tutorials that in $\R^3$ we have
            \[
                [g_{ij}] =
                \begin{bmatrix}
                    1 & & \\
                    & r^2 & \\
                    & & r^2\sin^2\theta
                \end{bmatrix}
            \]
            in spherical coordinates $(r, \theta, \varphi)$.
            This is a diagonal matrix, so obtaining its inverse is trivial.
            By the above local formula we get
            \[
                \nabla f
                = \sum_{i = 1}^{3} g^{ii} (\partial_if) \partial_i
                = \frac{\partial f}{\partial r} \partial_r + \frac{1}{r^2}\frac{\partial f}{\partial\theta} \partial_\theta + \frac{1}{r^2\sin^2\theta}\frac{\partial f}{\partial\varphi}\partial_\varphi
            \]
        \item Let $X \in \mathfrak{X}(M)$ be a vector field and $(x^i)$ some local coordinates on $M$.
            Let us show that there holds
            \[
                \tr(\nabla X) = (\nabla_{\partial_k}X)^k = \frac{1}{\sqrt{\det[g_{ij}]}} \partial_k\left( \sqrt{\det[g_{ij}]} X^k \right)
            \]
            
            Since the trace operator is independent of the choice of the local coordinates,
            let without loss of generality $(x^i)$ be some normal coordinates on $M$,
            centered at $p \in M$.
            Recall from Tutorials that normal coordinates define an orthonormal frame,
            hence $g_{ij}(p) = \delta_{ij}$.
            But this property makes the RHS equal to $\partial_k(X^k)$,
            which is precisely the LHS.

            We will now express $\text{div}(X) = \tr(\nabla X)$ in spherical coordinates on $\R^3$.
            \[
                \text{div}(X)
                = \frac{1}{r^2\sin\theta}\sum_{i = 1}^{3} \partial_i (r^2\sin\theta X^i)
                = \frac{1}{r^2} (r^2 \frac{\partial X^r}{\partial r} + 2rX^r) + \frac{1}{\sin\theta} (\sin\theta \frac{\partial X^\theta}{\partial \theta} + \cos\theta X^\theta) + \frac{\partial X^\varphi}{\partial \varphi}
            \]
        \item Define the covariant Hessian of a function $f \in C^\infty(M)$ as $H(f)$, given by
            \[
                H(f)(X,Y) = (\nabla_X df)(Y)
            \]
            
            This defines a $2$-tensor field, since by the properties of a covariant derivative (or a connection),
            the definition is bilinear.
            We know from (1i) that for a $1$-form $df$, its covariant derivative in the direction of $X$ can be computed as
            \[
                (\nabla_X df)(Y) = X(df(Y)) - df(\nabla_XY) = X(Y(f)) - (\nabla_XY)(f)
            \]
            and we also know how to express this locally.
            So let $(x^i)$ be some local coordinates on $M$.
            Then, writing $X = X^i\partial_i$ and $Y = Y^j\partial_j$, we get
            \[
                H(f) = (\partial_i\partial_j f - \Gamma_{ij}^k\partial_k f)dx^i \otimes dx^j
            \]
            It is clearly symmetric, since $\partial_i\partial_j = \partial_j\partial_i$,
            and because $\nabla$ is Levi-Civita and hence symmetric, i.e. $\Gamma_{ij}^k = \Gamma_{ji}^k$.
        \item Lastly, we define the Laplacian of a function $f \in C^\infty(M)$ by
            \[
                \Delta f = \text{div}(\nabla f)
            \]
            
            In local coordinates $(x^i)$, we can express this as
            \[
                \Delta f
                = \text{div}((g^{ij}(\partial_j f))\partial_i)
                = \frac{1}{\sqrt{\det[g_{ij}]}} \partial_i \left( \sqrt{\det[g_{ij}]} (g^{ij}\partial_j f) \right)
            \]
            Now write
            \[
                \Delta f = \text{div}(\nabla f)
                \overset{(2ii)}{=} \tr(\nabla\nabla f)
            \]
            since the trace operator is independent of the choice of coordinates,
            choose normal coordinates $(x^i)$.
            In this case $g_{ij} = g^{ij} = \delta_{ij}$,
            hence $\nabla f = (\partial_j f) \partial_i$.
            We also have $\Gamma_{ij}^k = 0$, so $H(f) = (\partial_i\partial_j f)dx^i\otimes dx^j$.
            Hence $H(f)^\sharp = (\partial_i\partial_k f) \partial_j \otimes dx^k$
            and its trace is precisely $\partial_i \partial_k f$,
            but $\Delta f$ in these coordinates is also given by $\partial_i\partial_j f$,
            which are the same expressions.

            Furthermore, since we are in normal coordinates, this frame is orthonormal and we get the usual formula
            \[
                \Delta f = \partial_i \partial_i f
            \]
            we know from analysis.
    \end{enumerate}
\end{Exercise}

\begin{Exercise}
    Let $M$ and $\tilde{M}$ be smooth manifolds.
    \begin{enumerate}[label=(\roman*)]
        \item Let $f \colon M \to \tilde{M}$ be a diffeomorphism and $\tilde{\nabla}$ a connection on $\tilde{M}$.
            Let $f^*\tilde{\nabla} \colon \mathfrak{X}(M) \times \mathfrak{X}(M) \to \mathfrak{X}(M)$, given by
            \[
                \left( f^*\tilde{\nabla} \right)_X Y = (f^{-1})_* \left( \tilde{\nabla}_{f_*X}f_*Y \right)
            \]
            be the pullback of connection $\tilde{\nabla}$.
            Let us prove that $f^*\tilde{\nabla}$ is a connection on $M$.

            Recall, that $f_*X$ is the unique vector field over $\tilde{M}$ such that $df_p(X_p) = (f_*X)_{f(p)}$ for all $p \in M$.
            It is then obvious that the above definition is $\R$-linear in $Y$, since $\tilde{\nabla}$, $df$ and $df^{-1}$ are all $\R$-linear.
            Next, we take $g \in C^\infty(M)$ and prove $C^\infty(M)$-linearity in $X$.
            Denote $\tilde{g} = g \circ f^{-1}$, so we have $f_*(gX) = \tilde{g} f_*X$.
            We calculate
            \[
                \left( f^*\tilde{\nabla} \right)_{gX} Y
                = (f^{-1})_* (\tilde{\nabla}_{\tilde{g}f_*X}(f_*Y))
                = (f^{-1})_* (\tilde{g}\tilde{\nabla}_{f_*X}(f_*Y))
                = g \left( f^*\tilde{\nabla} \right)_X Y
            \]
            Lastly, we need to show the product rule holds.
            Let $g \in C^\infty(M)$ and denote $\tilde{g}$ as above.
            Notice that the definition of the push-forward of a vector field coincides with that of the two vector fields being $f$-related.
            We then know from Tutorials that this implies (actually equivalent) that
            $(f_*X)(\tilde{g}) = (Xg) \circ f^{-1}$.
            We then calculate
            \[
                \left( f^*\tilde{\nabla} \right)_X(fY)
                = (f^{-1})_* (\tilde{\nabla}_{f_*X}(\tilde{g}f_*Y))
                = (f^{-1})_* (\tilde{g}\tilde{\nabla}_{f_*X}(f_*Y) + (f_*X)(\tilde{g})f_*Y)
                = g(f_*\tilde{\nabla})_XY + (Xg)Y
            \]
            which finishes the proof.
        \item Denote now $\nabla = f^*\tilde{\nabla}$.
            Let $\gamma$ be a smooth path in $M$ and $V$ a vector field along $\gamma$.
            We will prove that
            \[
                df \circ D_\gamma V = \tilde{D}_{f\circ\gamma}(df \circ V)
            \]

            Recall that we have
            \[
                D_\gamma V (t)
                = \nabla_{\gamma'(t)}V
                = (f^*\tilde{\nabla})_{\gamma'(t)}V
                = (f^{-1})_*(\tilde{\nabla}_{f_*\gamma'(t)} f_*V)
                = df^{-1}\circ(\tilde{\nabla}_{(f\circ\gamma)'(t)} (df\circ V))
            \]
            and
            \[
                \tilde{D}_{f\circ\gamma}(df\circ V)
                = \tilde{\nabla}_{(f\circ\gamma)'} (df\circ V)
            \]
            so composing the former with $df$ finishes the proof.
            This then shows that $f$ maps $\nabla$-geodesics to $\tilde{\nabla}$-geodesics,
            since inputting $V = \gamma'$ in the LHS for $\gamma$ a geodesic makes the LHS vanish,
            so $D_{f\circ\gamma}((f\circ\gamma)') = 0$ by the above equality, i.e., $f\circ\gamma$ is a geodesic.
        \item Suppose $f \colon M \to \tilde{M}$ is an isometry between Riemannian manifolds $M$ and $\tilde{M}$.
            Denote by $\nabla$ and $\tilde{\nabla}$ the Levi-Civita connections on $M$ and $\tilde{M}$, respectively.
            Let us show that $f^*\tilde{\nabla} = \nabla$.

            Since \underline{the} Levi-Civita connection is unique on a Riemannian manifold,
            it suffices to show that $f^*\tilde{\nabla}$ is metric-compatible with $g$ and symmetric.
            Since $f$ is an isometry, $df_p$ is a linear isometry at every $p \in M$,
            so we have for any two vector fields $X, Y$ and $p \in M$
            \[
                \langle X_p, Y_p \rangle
                = \langle df_p(X_p), df_p(Y_p) \rangle = \langle (f_*X)_{f(p)}, (f_*Y)_{f(p)}\rangle
            \]
            and hence
            \begin{align*}
                Z \langle X, Y \rangle
                &= Z(\langle f_*X, f_*Y \rangle \circ f) \\
                &= ((f_*Z) \langle f_*X, f_*Y \rangle) \circ f \\
                &= (\langle \tilde{\nabla}_{f_*Z}(f_*X), f_*Y \rangle + \langle f_*X, \tilde{\nabla}_{f_*Z}(f_*Y) \rangle )\circ f \\
                &= \langle (f^{-1})_* \tilde{\nabla}_{f_*Z}(f_*X), Y \rangle + \langle X, (f^{-1})_*\tilde{\nabla}_{f_*Z}(f_*Y) \rangle \\
                &= \langle (f^*\tilde{\nabla})_Z X, Y \rangle + \langle X, (f^* \tilde{\nabla})_Z Y \rangle
            \end{align*}
            For symmetricity, we calculate
            \[
                (f^*\tilde{\nabla})_X Y - (f^*\tilde{\nabla})_Y X
                = (f^{-1})_* (\tilde{\nabla}_{f_*X}(f_*Y) - \tilde{\nabla}_{f_*Y}(f_*X))
                = (f^{-1})_* [f_*X, f_*Y ]
                = [X, Y]
            \]
        \item We will now prove that an isometry of Riemannian manifolds maps geodesics to geodesics.

            This is of course a direct consequence of (3iii) and the conclusion of (3ii).
        \item A local isometry of Riemannian manifolds also maps geodesics to geodesics,
            since being a geodesic is a local property.
            That is, a geodesic is defined as a solution of a system of ordinary differential equations,
            which we know has a unique solution defined locally in a neighbourhood of the given initial condition.
    \end{enumerate}
\end{Exercise}

\begin{Exercise}
    Let $(\H, g_\H)$ denote the hyperbolic half-plane, that is
    \[
        \H = \R \times (0, \infty), \quad g_\H = \frac{1}{y^2}(dx^2 + dy^2)
    \]
    \begin{enumerate}[label=(\roman*)]
        \item Let us calculate the Christoffel symbols of the Levi-Civita connection $\nabla$ on $\H$ with respect to the standard coordinates $(x, y)$.
            Recall from Tutorials the equation for Christoffel symbols of the Levi-Civita connection
            \[
                \Gamma_{ij}^k = \frac{1}{2}g^{kl} (\partial_i g_{jl} + \partial_j g_{il} - \partial_l g_{ij})
            \]
            The matrix $g$ is clearly given by
            \[
                [g_{ij}] =
                \begin{bmatrix}
                    \frac{1}{y^2} & \\
                    & \frac{1}{y^2}
                \end{bmatrix}
                ,\quad \quad
                [g^{ij}] =
                \begin{bmatrix}
                    y^2 & \\
                    & y^2
                \end{bmatrix}
            \]
            so $\partial_x g_{ij} = \partial_x g^{ij} = 0$ for all $i,j$.
            We get
            \begin{align*}
                &\Gamma_{xx}^x = 0
                &\Gamma_{xx}^y = \frac{1}{2} y^2 (-\frac{-2}{y^3}) = \frac{1}{y} \\
                &\Gamma_{xy}^x = \Gamma_{yx}^x = \frac{1}{2} y^2 (\frac{-2}{y^3}) = -\frac{1}{y}
                &\Gamma_{xy}^y = \Gamma_{yx}^y = 0 \\
                &\Gamma_{yy}^x = 0
                &\Gamma_{yy}^y = \frac{1}{2} y^2 (\frac{-2}{y^3}) = -\frac{1}{y}
            \end{align*}
        \item We will now write down the connection $1$-form $\omega$ of the
            Levi-Civita connection $\nabla$ with respect to the standard
            coordinates $(x, y)$ and then calculate the curvature $2$-form $F$.

            By definition, we get
            \[
                \omega_j^k(X)\partial_k
                = \nabla_X\partial_j
                = X^i\nabla_{\partial_i}\partial_j
                = X^i\Gamma_{ij}^k\partial_k
            \]
            and hence $\omega_j^k(X) = X^i\Gamma_{ij}^k$.
            For any vector field $X = X^1 \partial_x + X^2\partial_y$ we therefore have
            \[
                \omega(X) =
                \begin{bmatrix}
                    -\frac{X^2}{y} & -\frac{X^1}{y} \\
                    \frac{X^1}{y} & -\frac{X^2}{y}
                \end{bmatrix}
            \]
            so if we consider $\omega$ acting on $X$ by each $(i, j)$-component separately, we can write
            \[
                \omega =
                \begin{bmatrix}
                    -\frac{1}{y}dy & -\frac{1}{y} dx \\
                    \frac{1}{y} dx & -\frac{1}{y} dy
                \end{bmatrix}
            \]
            The curvature $2$-form $F$ is then expressed in the same local coordinates as
            \[
                F = d\omega + \omega\wedge\omega
            \]
            where
            \[
                (\omega\wedge\omega)_{ij} = \sum_{k=1}^{n}\omega_{ik}\wedge\omega_{kj}
            \]
            and $d\omega$ is calculated component-wise.
            Since
            \[
                d\left( \frac{1}{y} \right) = 0\cdot dx - \frac{1}{y^2}dy = -\frac{1}{y^2}dy
            \]
            we get
            \[
                d\omega =
                \begin{bmatrix}
                    0 & -\frac{1}{y^2} dx \wedge dy \\
                    \frac{1}{y^2} dx\wedge dy & 0
                \end{bmatrix}
            \]
            and by the above formula
            \begin{align*}
                (\omega\wedge\omega)_{11}
                &= \omega_{11}\wedge_{11} + \omega_{12}\wedge\omega_{21}
                = 0 \\
                (\omega\wedge\omega)_{12}
                &= \omega_{11}\wedge\omega_{12} + \omega_{12}\wedge\omega_{22}
                = -\frac{1}{y^2}dx\wedge dy + \frac{1}{y^2}dx \wedge dy
                = 0 \\
                (\omega\wedge\omega)_{21}
                &= \omega_{21}\wedge\omega_{11} + \omega_{22}\wedge\omega_{21}
                = -\frac{1}{y^2}dx \wedge dy + \frac{1}{y^2} dx \wedge dy
                = 0 \\
                (\omega\wedge\omega)_{22}
                &= \omega_{21}\wedge\omega_{12} + \omega_{22}\wedge\omega_{22}
                = 0
            \end{align*}
            hence
            \[
                F =
                \begin{bmatrix}
                    0 & -\frac{1}{y^2} dx \wedge dy \\
                    \frac{1}{y^2} dx \wedge dy & 0
                \end{bmatrix}
            \]
        \item Let $(\D, g_\D)$ be the hyperbolic disk, that is
            \[
                \D = D(0, 1), \quad g_\D = \frac{4}{(1 - x^2 - y^2)^2}(dx^2 + dy^2)
            \]
            Let us show that the M\"obius transformation $\tau \colon \H \to \D$, given by
            \[
                \tau(z) = \frac{1 + iz}{z + i}
            \]
            is an isometry.

            We already know from Complex Analysis that $\tau$ is a diffeomorphism.
            Indeed, this is the M\"obius transformation that maps
            \[
                i \mapsto 0 \quad 1 \mapsto 1 \quad ``\infty'' \mapsto i
            \]
            (this makes more sense looking at the inverse map,
            we open up the edge of the disk into the $x$-axis).
            We only need to check that it is indeed metric-preserving,
            that is, $g_\H = \tau^* g_\D$.

            Take a more general case;
            let $\tau \colon (M, g) \to (N, h)$ be a smooth map between Riemannian manifolds
            Let $p \in M$ and $\partial_i$ an orthogonal frame around $p$.
            Then at point $p$ we have by definition of the pull-back that
            the equality $\tau^* h = g$ implies
            \[
                h_{\tau(p)} \cdot d\tau_p = d\tau_p \cdot g_p
            \]
            Since our local frame is orthogonal,
            the inverse of the differential on the RHS is given by its transposed matrix,
            so we get
            \[
                d\tau_p^T \cdot h_{\tau(p)} \cdot d\tau_p = g_p
            \]
            Now identify
            \[
                x + iy \idtfy
                \begin{bmatrix}
                    x & -y \\
                    y & x
                \end{bmatrix}
            \]
            which is precisely the form of the Jacobi matrix of a function
            that solves the Cauchy-Riemann equations.
            Indeed, since M\"obius transformations are biholomorphisms of the Riemann sphere,
            in particular $\tau$,
            the Cauchy-Riemann system of equations holds for $\tau$.
            Also notice that in the above identification,
            complex conjugation corresponds to matrix transposition,
            and taking the square of the absolute value corresponds to the determinant.
            Now, since the coefficients in front of the inner products at each point $z$ are
            $\frac{1}{\text{Im}(z)^2}$ for $g_\H$ and $\frac{4}{(1 - |z|^2)^2}$ for $g_\D$,
            we can rewrite the above condition as
            \[
                \overline{\tau'(z)} \frac{4}{(1 - |\tau(z)|^2)^2} \tau'(z) = \frac{1}{\text{Im}(z)^2}
            \]
            We now merely verify that this equality holds.
            We calculate the derivative
            \[
                \tau'(z)
                = (\frac{1 + iz}{z + i})'
                = \frac{i(z + i) - (1 + iz)}{(z + i)^2}
                = \frac{-2}{z^2 + 2iz - 1}
            \]
            hence
            \[
                \overline{\tau'(z)}\tau'(z)
                = |\tau'(z)|^2
                = \frac{4}{(x^2 - y^2 - 2y -1)^2 + (2xy + 2x)^2}
                = \frac{4}{(x^2 + y^2 + 2y + 1)^2}
            \]
            and similarly
            \[
                1 - |\tau(z)|^2
                = 1 - \frac{(1 - y)^2 + x^2}{x^2 + (1 + y)^2}
                = \frac{4y}{x^2 + y^2 + 2y + 1}
            \]
            so the equality indeed holds.
        \item Let $v = \partial_x|_{(0, 0)}$ and $w = \partial_y|_{(0,0)}$ be
            two vector fields in $(\D, g_\D)$.  Let us describe the maximal
            geodesics $\gamma_v$ and $\gamma_w$ of the space $(\D,g_\D)$.

            We know from Tutorials that the geodesics in $\H$ are precisely
            half-circles with their centers on the $x$-axis, and also vertical lines.
            We know from (3iv) that $\tau$ maps geodesics to geodesics,
            and we know from Complex Analysis that $\tau$ is a conformal map.
            Hence all geodesics on $\D$ must be precisely circle arcs that touch the
            edge of $\D$ in a perpendicular manner, and diameter lines.
            In particular, the maximal geodesics $\gamma_v$ and $\gamma_w$ must be
            the horizontal and the vertical diameter, respectively.
            Indeed, these are precisely the unit half-circle through $i$ and the
            vertical line through $i$ in $\H$, respectively,
            since $\tau$ maps $1 \mapsto 1$ and $-1 \mapsto -1$.
        \item Let $A = \begin{bmatrix} a & b \\ c & d \end{bmatrix} \in SL(2, \R)$ and
            let $\tau_A \colon \H \to \H$ be the M\"obius transformation
            \[
                \tau_A(z) = \frac{a z + b}{c z + d}
            \]
            We will show that $\tau_A$ is an isometry of $(\H, g_\H)$.

            We know already that $\tau_A$ is a biholomorphism (and hence a diffeomorphism) of the upper half-plane.
            We therefore have to prove that $\tau_A$ is metric-preserving,
            that is, at every point $z \in \H$ we want that
            \[
                g_z(X_z, Y_z) = g_{\tau_A(z)}(\tau_A'(z)X_z, \tau_A'(z)Y_z)
            \]
            holds for any two vector fields $X, Y$ on $\H$.
            We calculate
            \[
                \tau_A'(z)
                = \frac{a(cz + d) - (az + b)c}{(cz + d)^2}
                = \frac{1}{(cz + d)^2}
            \]
            and for the new coefficient
            \[
                \text{Im}(\tau_A(z))
                = \text{Im}\left( \frac{(az + b)\overline{(cz + d)}}{|cz + d|^2} \right)
                = \frac{\text{Im}(z)}{|cz + d|^2}
            \]
            It then follows that
            \begin{align*}
                g(\tau_A '(z) X_z, \tau_A '(z) Y_z)
                &= \frac{|cz+d|^4}{\text{Im}(z)^2} g(\frac{1}{(cz+d)^2}X_z, \frac{1}{(cz+d)^2}Y_z) \\
                &= \frac{|cz+d|^4}{\text{Im}(z)^2} \frac{1}{(cz+d)^2} \frac{1}{\overline{(cz+d)}^2} g(X_z, Y_z) \\
                &= \frac{1}{\text{Im}(z)^2} g(X_z, Y_z)
            \end{align*}
            so $\tau_A$ is indeed an isometry.
    \end{enumerate}
\end{Exercise}

\end{document}
