\documentclass[a4paper, 12pt]{article}

\usepackage[english]{babel}
\usepackage[margin=0.5in]{geometry}

\usepackage[utf8]{inputenc}
\usepackage[T1]{fontenc}
\usepackage{lmodern}
\usepackage{units}
\usepackage{eurosym}
\usepackage{titlesec}
\usepackage{graphicx}
\usepackage{wrapfig}
\usepackage{color}
%\usepackage{url}
\usepackage{hyperref}
\usepackage{enumerate}
\usepackage{enumitem}
\usepackage{pifont}
\usepackage[normalem]{ulem}

% packages
\usepackage{amsmath}
\usepackage{amssymb}
\usepackage{amsthm}
\usepackage{amsfonts}
\usepackage{mathtools}
\usepackage{tikz-cd}
\usetikzlibrary{babel}
\usepackage{adjustbox}
\usepackage{stmaryrd}

% commonly used math operators
\DeclareMathOperator{\diam}{diam}
\DeclareMathOperator{\diag}{diag}
\DeclareMathOperator{\rank}{rank}
\DeclareMathOperator{\tr}{tr}
\DeclareMathOperator{\im}{im}
\DeclareMathOperator{\dom}{dom}
\DeclareMathOperator{\coker}{coker}
\DeclareMathOperator{\codim}{codim}
\DeclareMathOperator{\pr}{pr}
\DeclareMathOperator{\rad}{rad}
\DeclareMathOperator{\chrs}{char}
\DeclareMathOperator{\len}{len}
\DeclareMathOperator{\Lin}{Lin}
\DeclareMathOperator{\Ann}{Ann}
\DeclareMathOperator{\Ass}{Ass}
\DeclareMathOperator{\Spec}{Spec}
\DeclareMathOperator{\mSpec}{mSpec}
\DeclareMathOperator{\Quot}{Quot}
\DeclareMathOperator{\Tor}{Tor}
\DeclareMathOperator{\Ext}{Ext}
\DeclareMathOperator{\Hom}{Hom}
\DeclareMathOperator{\End}{End}
\DeclareMathOperator{\Aut}{Aut}
\DeclareMathOperator{\Br}{Br}
\DeclareMathOperator{\Gal}{Gal}

% commonly used math objects
\newcommand{\F}{\mathbb{F}}
\newcommand{\A}{\mathbb{A}}
\newcommand{\D}{\mathbb{D}}
\renewcommand{\S}{\mathbb{S}}
\newcommand{\T}{\mathbb{T}}
\newcommand{\B}{\mathbb{B}}
\newcommand{\I}{\mathbb{I}}
\newcommand{\N}{\mathbb{N}}
\newcommand{\Z}{\mathbb{Z}}
\newcommand{\Q}{\mathbb{Q}}
\newcommand{\R}{\mathbb{R}}
\newcommand{\C}{\mathbb{C}}
\renewcommand{\H}{\mathbb{H}}
\renewcommand{\P}{\mathbb{P}}

% commonly used math relations
\newcommand{\iso}{\cong}
\newcommand{\homeo}{\approx}
\newcommand{\htpeq}{\simeq}
\newcommand{\hlgeq}{\sim}
\newcommand{\idtfy}{\longleftrightarrow}

% commonly used math symbols
\newcommand{\closure}[1]{\overline{#1}}
\newcommand{\subideal}{\vartriangleleft}
\newcommand{\supideal}{\vartriangleright}

% numbered environments
\theoremstyle{plain}
\newtheorem{theorem}{Theorem}[section]
\newtheorem{corollary}[theorem]{Corollary}
\newtheorem{exercise}[theorem]{Exercise}
\newtheorem{lemma}[theorem]{Lemma}
\newtheorem{proposition}[theorem]{Proposition}

\theoremstyle{definition}
\newtheorem{definition}[theorem]{Definition}

\theoremstyle{remark}
\newtheorem*{claim}{Claim}
\newtheorem*{remark}{Remark}

\newcounter{excounter}[section]
\newenvironment{Exercise}
    {\refstepcounter{excounter}\underline{\textbf{Ex. \theexcounter:}}}
    {\par\vspace{\baselineskip}}


\begin{document}

\begin{titlepage}
\centering
\textbf{\Huge{The de Rham Theorem}}
%\vfill
%\textbf{\LARGE{with applications to algebraic topology}}
\vfill\vfill
\textsc{\Large{Benjamin Benčina}}
\vfill\vfill
\textsc{\large{University of Ljubljana}}

\textsc{\large{Faculty of Mathematics and Physics}}

\textsc{\large{Department of Mathematics}}
\vfill\vfill\vfill

{\large\today}

\end{titlepage}

\tableofcontents
\newpage

\section{Introduction}

The ability to assign simple algebraic objects to more complicated objects from
various disciplines of mathematics and then deduce information about the object
is an exceedingly powerful one, and this strongly becomes apparent when
assigning homology groups to manifolds.  Homology and its dual, cohomology,
have established themselves in the past century as important tools in the study
of manifolds.  In this essay, we take a closer look at one of the prime
examples of a cohomology theory that arises naturally from the study of smooth
manifolds, in particular while studying differential forms.  What starts as a
seemingly simple question -- is every closed form exact? -- turns into a
beautiful theory that recontextualizes what we know about (co)homology in a
concrete geometric setting.

After discussing the necessary preliminary topics, integration on manifolds and
singular (co)homology, we will dive right into the topic at hand and define the
de Rham cohomology groups.  We will then see that these groups are not only
invariant with respect to diffeomorphisms, but also with respect to
homeomorphisms, which begs the question whether there exists a merely
topological way of viewing these groups.  To answer this question we will
construct an isomorphism between the de Rham cohomology groups and the singular
cohomology groups, yielding the famous \emph{de Rham theorem}.  Finally, after
performing some simple calculation, we will try to view the two products on
singular (co)homology, cup and cap, in a new geometric light, and then restate
the \emph{Poincar\'e duality}.

\section{Preliminaries}

This section will cover the two major necessary topics one needs to start
working with the de Rham cohomology, namely differential topology and algebraic
topology. The differential topology subsection will discuss integration of
differential forms over manifolds and Stokes' Theorem.  We will then generalize
this to \emph{manifolds with corners}, which will help us later when defining
the \emph{de Rham homomorphism}.  The algebraic topology subsection will briefly
discuss singular homology and cohomology, and some of its properties.  We
intentionally omit proofs in this section, since Stokes' Theorem is not the
topic of this essay, but sources will be provided.

\subsection{Differential topology}

A key concept involved in construction the de Rham homomorphism is integration of differential forms over manifolds.
We first start with the real case.
Let $D$ be a domain of integration in $\R^n$ and let $\omega$ be a (continuous) $n$-form on $\closure{D}$ written as
$\omega = f dx^1\wedge\cdots dx^n$ for some continuous function $f \colon \closure{D} \to \R$.
We define the \emph{integral of $\omega$ over $D$} to be
\[
    \int_{D} \omega = \int_{D} f \; dV
\]
That is, we compute the integral of such a form by simply erasing the wedges.
The definition for the real half-space $\H^n$ is identical.

To generalize this definition to the case of manifolds, we have to consider the integral of a pull-back of a differential form.
The following proposition is proved in e.g. \cite[Prop. 16.1]{Lee2012}.

Suppose $D$ and $E$ are open domains of integrations in $\R^n$ or $\H^n$,
and $G \colon \closure{D} \to \closure{E}$ a smooth map that restricts to an orientation-preserving or orientation-reversing diffeomorphism from $D$ to $E$.
If $\omega$ is an $n$-form on $\closure{E}$, then
\[
    \int_{D} G^*\omega =
    \begin{cases}
        + \int_{E} \omega &\text{if $G$ is orientation-preserving,} \\
        - \int_{E} \omega &\text{if $G$ is orientation-reversing.}
    \end{cases}
\]

Using the above proposition we can now define the integral of a differential form over an oriented manifold.
Let $M$ be an oriented manifold with or without boundary and let $\omega$ be an $n$-form on $M$.
Suppose $\omega$ is compactly supported in the domain of a single smooth chart $(U,\varphi)$ that is either positively or negatively oriented.
We then define the \emph{integral of $\omega$ over $M$} to be
\[
    \int_{M} \omega = \pm \int_{\varphi(U)} \left( \varphi^{-1} \right)^*\omega
\]
with the positive sign for a positively oriented chart, and the negative sign otherwise.
It is of course easy to see that this definition does not depend on the choice of chart containing the support of $\omega$.

To integrate over the entire manifold, we combine this definition with a partition of unity.
Let $M$ be an oriented manifold with or without boundary and let $\omega$ be a compactly supported $n$-form on $M$.
Let $\left\{ U_i \right\}$ be a finite open cover of the support of $\omega$ by domains of positively or negatively oriented smooth charts,
and let $\left\{ \psi_i \right\}$ be a subordinate smooth partition of unity.
We define the \emph{integral of $\omega$ over $M$} to be
\[
    \int_{M} \omega = \sum_{i} \int_{M} \psi_i\omega
\]
This definition is independent of the choice of the finite cover or the partition of unity.
For the zero-dimensional case the integral of a compactly supported $0$-form, i.e., a real valued function,
is merely the sum of values of the function, adjusting for orientation with signs.
As expected, this integral has the usual ``nice'' properties (\cite[Prop. 16.6]{Lee2012}).
%As expected, this integral has the usual ``nice'' properties (proof in \cite[Prop. 16.6]{Lee2012}).
%\begin{proposition}
    %Suppose $M$ and $N$ are non-empty oriented smooth $n$-manifolds with or without boundary,
    %and $\omega,\eta$ are compactly supported $n$-forms on $M$.
    %\begin{enumerate}[label=(\alph*)]
        %\item If $a, b \in \R$ then $\int_{M} a\omega + b\eta = a\int_{M} ;Gw + b\int_{M} \eta$.
        %\item If $-M$ denotes $M$ with the opposite orientation then $\int_{-M}\omega = - \int_{M} \omega$.
        %\item If $\omega$ is a positively oriented volume form then $\int_{M}\omega > 0$.
        %\item If $F \colon N \to M$ is an orientation-preserving or orientation-reversing diffeomorphism then
            %\[
                %\int_{M} \omega =
                %\begin{cases}
                    %+\int_{N} F^*\omega &\text{if $F$ is orientation-preserving,} \\
                    %-\int_{N} F^*\omega &\text{if $F$ is orientation-reversing.}
                %\end{cases}
            %\]
    %\end{enumerate}
%The last property gives us an easier method of actually calculating this integral via local parametrizations.
%\end{proposition}

We can now finally state Stokes' Theorem, one of the most important theorems in the theory of smooth manifolds (proof in \cite[Thm. 16.11]{Lee2012}).
\begin{theorem}
    Let $M$ be an oriented smooth $n$-manifold with boundary and let $\omega$ be a compactly supported smooth $(n-1)$-form on $M$.
    Then
    \[
        \int_{M} d\omega = \int_{\partial M} \omega
    \]
\end{theorem}
\begin{remark}
    The manifold $\partial M$ is understood to have the induced orientation,
    and the $\omega$ on the right-hand side is to be interpreted as $(\iota_{\partial M})^*\omega$,
    i.e., pull-back with respect to the inclusion map $\iota \colon \partial M \to M$.
\end{remark}

Sadly, this construction alone is not enough, as the de Rham theorem will require us to integrate over simplices.
To that effect, we define the so-called manifolds with corners.
Denote
\[
    \overline{\R}_+^n = \left\{ (x^1,\dots,x^n) \in \R^n ; \; x^i \geq 0 \right\}
\]
Suppose $M$ is a (topological) $n$-manifold with boundary.
A \emph{chart with corners} for $M$ is a pair $(U, \varphi)$ where $U \subseteq M$ is open and $\varphi$ is a homeomorphism $U \to \hat{U}\subseteq \overline{\R}_+^n$.
We define a \emph{smooth structure with corners} and by extension a \emph{smooth manifold with corners} in the usual way.
\begin{remark}
Note that by definition any closed rectangle in $\R^n$ is a smooth $n$-manifold with corners,
including all standard $n$-simplices (which are domains of singular $n$-simplices).
\end{remark}

One problem that arises when trying to integrate over the boundary of a smooth manifold with corners is that the boundary itself is not a smooth manifold with corners.
However, this problem is easily circumvented.
Denote by
\[
    H_i = \left\{ (x^1,\dots,x^n) \in \overline{\R}_+^n ; \; x^i = 0 \right\}
\]
the $i$-th $(n-1)$-dimensional smooth manifold with corners.
Let $M$ be an oriented smooth $n$-manifold with corners, and suppose $\omega$ is an $(n-1)$-form on $\partial M$ compactly supported in the domain of a single oriented smooth chart with corners.
We define the integral of $\omega$ over $\partial M$ by
\[
    \int_{\partial M} \omega = \sum_{i = 1}^{n}\int_{H_i} \left( \varphi^{-1} \right)^*\omega
\]
The Stokes' Theorem for smooth manifolds with corners can now be stated identically as above.

\subsection{Algebraic topology}

Let us now take a quick detour into the world of algebraic topology and recall singular homology.
Let $X$ be any topological space and let us denote by $\Delta^n = [e_0,\dots,e_n]$ the standard $n$-simplex.
A \emph{singular $n$-simplex} is a continuous map
\[
    \sigma \colon \Delta^n \to X
\]
For any $i = 0,\dots, n$ let $\varphi_i^n \colon \Delta^{n-1} \to \Delta^n$ be the $i$-th face map, defined by
\[
    \varphi_i^n(\Delta^{n-1}) = [e_0,\dots,\hat{e_i},\dots,e_n]
\]
For any $n\geq 0$, the \emph{$n$-th singular chain group} is the free Abelian group $C_n(X)$ generated by singular $n$-simplices in $X$.
We define a \emph{boundary homomorphism} between chain groups $\partial = \partial_n \colon C_n(X) \to C_{n-1}(X)$ by
\[
    \partial_n(\sigma) = \sum_{i=0}^{n}(-1)^i\sigma\circ\varphi_i^n
\]
Since $\partial_n \circ \partial_{n+1} = 0$ for every $n \leq 0$, the collection $(C_\cdot, \partial_\cdot)$ is the \emph{singular chain complex of $X$}.
A singular $n$-chain $c$ is called a \emph{cycle} if $\partial c = 0$,\footnote{cf. closed differential form.}
and a \emph{boundary} of $c = \partial b$ for some singular $(n+1)$-chain $b$.\footnote{cf. exact differential form.}
We define the \emph{$n$-th singular homology group of $X$} as the quotient group
\[
    H_n(X) = \frac{Z_n(X)}{B_n(X)} = \frac{\ker\partial_n}{\im\partial_{n+1}}
\]
Homology thus ``measures'' the amount of cycles of a particular dimension that are not boundaries.
This construction allows the use of quite a few useful algebraic tools such as the Mayer-Vietoris sequence (\cite[p. 149]{Hatcher2002}),
but we will not discuss them further.
The only thing we note is that by taking an Abelian group $G$ and constructing $C_n^G(X) = C_n(X) \otimes G$ and $\partial^G = \partial \otimes id_G$,
and then proceeding as above, we get \emph{homology groups with coefficients in $G$} by defining
\[
    H_n(X, G) = \frac{\ker\partial_n^G}{\im\partial_{n+1}^G}
\]

For us, the much more interesting notion is that of \emph{singular cohomology groups of $X$ with coefficients in $G$}.
We define by $C_G^n(X) = \Hom(C_n^G(X), G)$ the \emph{$n$-th cochain group} and by $\delta^{n+1} \colon C_G^n(X) \to C_G^{n+1}(X)$,
\[
    \delta^{n+1}\alpha = \alpha\circ\partial_{n+1}
\]
for every $n\geq 0$ the \emph{$n$-th coboundary map}.
We of course again have $\delta\circ\delta = 0$, so we define by analogy the \emph{$n$-th cohomology group of $X$}
\[
    H^n(X; G) = \frac{\ker\delta_G^{n+1}}{\im\delta_G^n}
\]

Many algebraic properties of homology transfer to cohomology, some of which we may need later.
It is worth noting that the cohomology groups of $X$ do not tell us anything fundamentally new about our space,
but rather give the same information in a more structured way.
Namely, the collection of cohomology groups can be made into a graded ring.

Of particular interest to us is the special case when $G = \R$.
It is then easy to see that $H^n(X; \R)$ is a real vector space that is naturally isomorphic to the space $\Hom(H_n(X), \R)$.

While this subsection may have been (overly) abstract, let us now approach a similar idea through a completely different, more concrete and geometric lens.

\section{De Rham cohomology}

Let us recall that a smooth differential form $\omega$ is
\begin{itemize}
    \item \emph{closed} if $d\omega = 0$, and
    \item \emph{exact} if it can be written as $w = d\eta$,
\end{itemize}
where $d \colon \Omega^k(M) \to \Omega^{k+1}(M)$ is the exterior derivative on a smooth manifold $M$.
We consider this immediate consequence of the Stokes' Theorem.
\begin{corollary}
    Let $M$ be a compact oriented smooth manifold and $\omega$ a form on $M$.
    \begin{enumerate}[label=(\alph*)]
        \item Suppose $M$ is without boundary and $\omega$ is exact,
            then
            \[
                \int_{M} d\omega = 0.
            \]
        \item Suppose $M$ can have a boundary and $\omega$ is closed,
            then
            \[
                \int_{\partial M} \omega = 0
            \]
    \end{enumerate}
\end{corollary}
Because $d \circ d = 0$, every exact form is closed.
Let us now explore the converse question: Is every closed form exact?
The following example confirms our suspicions that the answer is indeed negative.
\begin{example}
    \label{exa:piercedplane}
    Let $X = \R^2\setminus\lbrace 0 \rbrace$ be the pierced real plane, and consider
    \[
        \omega = \frac{x\; dy - y\; dx}{x^2 + y^2}
    \]
    Let $\gamma \colon [0,2\pi] \to X$ be the curve segment defined by $\gamma(t) = (\cos t, \sin t)$.
    We integrate
    \[
        \int_{\gamma}\omega
        = \int_{[0,2\pi]} \frac{\cos t(\cos t \; dt) - \sin t(-\sin t \; dt)}{\sin^2t + \cos^2t}
        = \int_{0}^{2\pi}dt
        = 2\pi
    \]
    Hence, $\omega$ cannot be exact by the above corollary.
\end{example}
A better question might then be whether we can somehow ``measure'' the non-exactness of closed forms,
which immediately suggests this might have something to do with homology.
Indeed, since $d \circ d = 0$ and $d$ is $\R$-linear, we obtain a complex
\[
    0 \xrightarrow{} \Omega^0(M) \xrightarrow{d} \Omega^1(M) \xrightarrow{d} \cdots \xrightarrow{d} \Omega^n(M) \xrightarrow{d} 0
\]
and consider linear subspaces
\begin{align*}
    Z^k(M) &= \ker\left( d \colon \Omega^k(M) \to \Omega^{k+1}(M) \right) = \left\{ \text{closed $k$-forms on $M$} \right\} \\
    B^k(M) &= \im\left( d \colon \Omega^{k-1}(M) \to \Omega^k(M) \right) = \left\{ \text{exact $k$-forms on $M$} \right\} \\
\end{align*}
Since $B^k(M) \leq Z^k(M)$ for every $k$, we define the \emph{$k$-th de Rham cohomology group} to be the quotient vector space (which we call a group out of convention)
\[
    H_{\text{dR}}^k(M) = \frac{Z^k(M)}{B^k(M)}
\]
Clearly, $H_{\text{dR}}^k(M) = 0$ for $k < 0$ and $k > n = \dim M$, but more importantly,
$H_{\text{dR}}^k(M) = 0$ if and only if  every closed $k$-form on $M$ is exact.
\begin{example}
    The previous example implies that $H_{\text{dR}}^1(\R^2\setminus\lbrace 0\rbrace) \neq 0$.
\end{example}
One of the main features of singular (co)homology groups is homeomorphism invariance, i.e., homeomorphic topological spaces have isomorphic (co)homology groups.
Since we are working with smooth manifolds, let us first show diffeomorphism invariance.
This will as in the singular case follow from the fact that induced maps are well-behaved.
\begin{proposition}
    For any smooth map $F \colon M \to N$ between smooth manifolds with or without boundary,
    the pull-back $F^* \colon \Omega^k(N) \to \Omega^k(M)$ carries closed forms into closed forms and exact forms into exact forms.
    It thus descends to a linear map $F^* \colon H_{\text{dR}}^k(N) \to H_{\text{dR}}^k(M)$, called the \emph{induced cohomology map},
    with the following properties
    \begin{enumerate}[label=(\alph*)]
        \item If $G \colon N \to P$ is another smooth map, then $(G\circ F)^* = F^* \circ G^*$.
        \item If $id$ denotes the identity map on $M$, then $id^*$ is the identity map of $H_{\text{dR}}^k(M)$.
    \end{enumerate}
\end{proposition}
\begin{proof}
    If $\omega$ is a closed form, then $d(F^*\omega) = F^*(d\omega) = 0$, so $F^*\omega$ is also closed.
    If $\omega = d\eta$ is exact, then $F^*\omega = F^*(d\eta) = d(F^*\eta)$, so $F^*\omega$ is also exact.

    We define the induced map on the cohomology groups in the obvious way: $F^*[\omega] = [F^*\omega]$.
    If $\omega$ and $\omega'$ are cohomologous, i.e., they differ by an exact form $d\eta$,
    then $[F^*\omega'] = [F^*\omega + d(F^*\eta)] = [F^*\omega]$, so this map is indeed well-defined.
    Then both properties follow from the properties of the pull-back map.
\end{proof}
Notice that the properties above remind us of the definition of a functor.
Indeed, for any $k = 0,\dots,n$, the assignment $M \mapsto H_{\text{dR}}^k(M)$ and $F \mapsto F^*$
is a contravariant functor from the category of smooth manifolds with boundary to the category of smooth vector spaces.
The next corollary is then immediate.
\begin{corollary}
    Diffeomorphic smooth manifolds with or without boundary have isomorphic de Rham cohomology groups.
\end{corollary}
\begin{proof}
    By functoriality, isomorphisms in one category get mapped to isomorphisms in another.
\end{proof}
It turns out, however, that as in the case of singular (co)homology, we can do better.
\begin{theorem}
    If $M$ and $N$ are homotopy equivalent smooth manifolds with or without boundary,
    then $H_{\text{dR}}^k \iso H_{\text{dR}}^k$ for each $k$.
    The isomorphisms are induced by a smooth homotopy equivalence $F \colon M \to N$.
\end{theorem}
\begin{proof}
    See \cite[Thm. 17.11]{Lee2012} and surrounding propositions.
\end{proof}
The upshot of this theorem is that the de Rham cohomology groups of a smooth manifolds $M$ are
in fact topological invariants, i.e., homeomorphic manifolds have isomorphic de Rham groups.
This is quite remarkable, since we defined the de Rham groups with the use of the exterior derivative map,
which is strongly connected to the smooth structure of $M$.

\section{De Rham Theorem}

The above paragraph hints to the idea that there may be a way to compute the de
Rham cohomology groups in a purely topological way, ignoring its smooth
structure all together. In this section we formalize this notion and prove it
is indeed possible. Even more, we will construct an explicit isomorphism between
the de Rham cohomology groups of a manifold and its singular cohomology groups
with real coefficients.
But first, a quick detour back to simplicial homology.

\subsection{Integration over simplices}

The connection between the de Rham and singular cohomology groups will be
established by somehow integrating differential forms over singular chains.
Explicitly, let $M$ be a smooth manifold, $\sigma$ a singular $k$-simplex and
$\omega$ a $k$-form on $M$.  We would like to integrate the pull-back of
$\omega$ along $\sigma$, which is a standard $k$-simplex $\Delta^k$ in $\R^n$.
However, we defined simplices to be continuous maps, whereas we can only pull
back forms along smooth maps.
Let us use this subsection to resolve this issue by introducing \emph{smooth
simplicial homology} and showing it is computationally equivalent to the one
defined above, that is, we obtain isomorphic groups.

\begin{definition}
    Let $M$ be a smooth manifold.
    A \emph{smooth $k$-simplex in $M$} is a map $\sigma \colon \Delta^k \to M$
    that has a smooth extension to a neighbourhood of each point.
    Denote by $C_k^\infty(M)$ the subgroup of $C_k(M)$ generated only by smooth simplices.
    Since the boundary of a smooth simplex is a smooth chain, i.e., a sum of smooth simplices,
    we define the \emph{$k$-th smooth singular homology group of $M$} to be the quotient
    \[
        H_k^\infty(M)
        = \frac{Z_k^\infty(M)}{B_k^\infty(M)}
        = \frac{\ker(\partial \colon C_k^\infty(M) \to C_{k-1}^\infty(M))}{\im(\partial \colon C_{k+1}^\infty(M) \to C_k^\infty(M))}
    \]
\end{definition}
Because the inclusion homomorphism $\iota \colon C_k^\infty(M) \to C_k(M)$ clearly commutes with the boundary operator,
it induces a map in homology $\iota_* \colon H_k^\infty(M) \to H_k(M)$ by $\iota_*[z] = [\iota(z)]$.
The following theorem proves the equivalence.
\begin{theorem}
    For any smooth manifold $M$ the map $\iota_* \colon H_k^\infty(M) \to H_k(M)$ is an isomorphism.
\end{theorem}
\begin{proof}
    See \cite[Thm. 18.7 and Lemma 18.8]{Lee2012}.
\end{proof}

We intentionally omit the rather long proof since we plan to merely use singular simplices to solve the above problem.
Indeed, let us go ahead and define aforementioned integration.
\begin{definition}
    Let $M$ be a smooth manifold, $\omega$ a closed $k$-form on $M$ and $\sigma$ a smooth $k$-simplex in $M$.
    We define the \emph{integral of $\omega$ over $\sigma$} to be
    \[
        \int_{\sigma} \omega = \int_{\Delta^k} \sigma^*\omega
    \]
    If $z = \sum_{i = 1}^{l}z_i \sigma_i$ is a smooth $k$-chain, the integral of $\omega$ over $z$ is defined as
    \[
        \int_{z} \omega = \sum_{i = 1}^{l} z_i\int_{\sigma_i} \omega
    \]
\end{definition}
\begin{remark}
    Recall that $\Delta^k$ is a smooth $k$-manifold with corners in $\R^k$,
    so the above definition indeed makes sense.
\end{remark}
What follows is the natural generalization of Stokes' Theorem from what we already know.
\begin{theorem}
    Let $z$ be a smooth $k$-chain in a smooth manifold $M$, and let $\omega$ be a smooth $(k-1)$-form on $M$.
    Then
    \[
        \int_{\partial z} \omega = \int_{z} d\omega
    \]
\end{theorem}
\begin{proof}
    See \cite[Thm. 18.12]{Lee2012}.
\end{proof}

We are now finally equipped with the knowledge we need to draw the equivalence.

\subsection{The de Rham homomorphism}

Using the above Stokes' Theorem for Chains we define the following map.
\begin{definition}
    Let $M$ be a smooth manifold.
    The \emph{de Rham homomorphism} is the linear map $\ell \colon H_{dR}^k(M) \to H^k(M; \R)$,
    defined as follows. For $[\omega] \in H_{dR}^k(M)$ and $[z] \in H_k(M) \iso H_k^\infty(M)$ (recall, cohomology classes are maps on homology groups), we define
    \[
        \ell [\omega] [z] = \int_{\tilde{z}} \omega
    \]
    where $\tilde{z}$ is some smooth $k$-cycle representing the homology class $[z]$.
\end{definition}
Let us quickly verify that this map is well defined.
If $\tilde{z}$ and $\tilde{z}'$ are homologous smooth cycles, i.e., $\tilde{z} = \partial\tilde{b} + \tilde{z}'$,
then
\[
    \int_{\tilde{z}} \omega - \int_{\tilde{z}'} \omega
    = \int_{\partial\tilde{b}} \omega
    = \int_{\tilde{b}} d\omega
    = 0
\]
and if $\omega = d\eta$ is exact then
\[
    \int_{\tilde{z}} \omega
    = \int_{\tilde{z}} d\eta
    = \int_{\partial\tilde{z}} \eta
    = 0
\]
One of the most important properties of the de Rham homomorphism is its naturality, as the following proposition states.
\begin{proposition}
    Let $M$ be a smooth manifold, $k$ a natural number, and let $\ell$ denote the de Rham homomorphism as defined above.
    \begin{enumerate}[label=(\alph*)]
        \item If $F \colon M \to N$ is a smooth map between smooth manifolds, then

            \adjustbox{scale=1, center}{
                \begin{tikzcd}
                    H_{dR}^k(N) \arrow[r, "F^*"] \arrow[d, "\ell"] & H_{dR}^k(M) \arrow[d, "\ell"] \\
                    H^k(N; \R) \arrow[r, "F^*"] & H^k(M; \R)
                \end{tikzcd}
            }
            
            is a commutative diagram.
        \item If $U, V \subset M$ are open sets such that $M = U \cup V$, them

            \adjustbox{scale=1, center}{
                \begin{tikzcd}
                    H_{dR}^{k-1}(U\cap V) \arrow[r, "\delta"] \arrow[d, "\ell"] & H_{dR}^k(M) \arrow[d, "\ell"] \\
                    H^k(U \cap V; \R) \arrow[r, "\partial^*"] & H^k(M; \R)
                \end{tikzcd}
            }
            
            is a commutative diagram,
            where $\delta$ and $\partial^*$ are the connecting homomorphisms
            of the Mayer-Vietoris sequences for the de Rham\footnote{See \cite[Thm. 17.20]{Lee2012}.}
            and singular\footnote{See \cite[pp. 149, 203]{Hatcher2002}.} cohomology, respectively.
    \end{enumerate}
\end{proposition}
\begin{proof}
    The first point is immediate.
    Let $\sigma$ be a smooth $k$-simplex in $M$ and let $\omega$ be a smooth $k$-form on $N$.
    Then
    \[
        \int_{\sigma} F^*\omega
        = \int_{\Delta^k} \sigma^*F^*\omega
        = \int_{\Delta^k} (F \circ \sigma)^* \omega
        = \int_{F\circ\sigma} \sigma
    \]
    which implies
    \[
        \ell(F^*[\omega])[\sigma]
        = \ell[\omega][F\circ\sigma]
        = \ell[\omega](F_*[\omega])
        = F^*(\ell[\omega])[\sigma]
    \]
    For the second point, the commutativity of the diagram in question is equivalent to
    \[
        \ell(\delta[\omega])[z] = (\partial^*\ell[\omega])[z]
    \]
    for any $[\omega] \in H_{dR}^{k-1}(U\cap V)$ and any $[z] \in H_k(M)$.
    Recall from the section on preliminaries that in real coefficients $H^k(M; \R) \iso \Hom(H_k(M), \R)$,
    so with this identification we rewrite the above as
    \[
        \ell(\delta[\omega])[z] = \ell([\omega])(\partial_*[z])
    \]
    Now, if $\sigma$ is a smooth $k$-form representing the class $\delta[\omega]$
    and $y$ is a smooth $(k-1)$-chain representing the class $\partial_*[z]$,
    by definition of $\ell$ this is the same as
    \[
        \int_{y} \sigma = \int_{z} \omega
    \]
    By the Mayer-Vietoris sequence for singular homology,
    we can let $y = \partial x$, where $x$ and $x'$ are smooth $k$-chains in $U$ and $V$,
    respectively, such that $x + x'$ is homologous to $z$.
    By properties of the Mayer-Vietoris sequence, we can choose $\eta \in \Omega^{k-1}(U)$ and $\eta' \in \Omega^{k-1}(V)$
    such that $\omega = \eta|_{U \cap V} - \eta'|_{U\cap V}$,
    and then let $\sigma$ be the $k$-form that is equal to $d\eta$ on $U$ and $d\eta'$ on $V$.
    Since $\partial x + \partial x' = \partial z = 0$ as $z$ is a cycle,
    and $d\eta|_{U\cap V} - d\eta'|_{U\cap V} = d\omega = 0$,
    we have
    \[
        \int_{y} \omega
        = \int_{\partial x} \omega
        = \int_{\partial x} \eta - \int_{\partial x} \eta'
        = \int_{\partial x} \eta + \int_{\partial x'} \eta'
        = \int_{x} d\eta + \int_{x'} d\eta'
        = \int_{x} \sigma + \int_{x'} \sigma
        = \int_{z} \sigma
    \]
    so the diagram indeed commutes.
\end{proof}
We are now ready to state and prove the de Rham Theorem.
\begin{theorem}
    For every smooth manifold $M$ and natural number $k$, the de Rham homomorphism $\ell$ is an isomorphism.
\end{theorem}
\begin{proof}
    We say that $M$ is a \emph{de Rham manifold} if the de Rham homomorphism is
    an isomorphism for each $k$.  Since by naturality $\ell$ commutes with the
    cohomology maps induced by smooth maps between smooth manifolds, any
    manifold diffeomorphic to a de Rham manifold also satisfies the property of
    being de Rham.  We call an open cover $\mathcal{U} = \left\{ U_i \right\}$
    of $M$ a \emph{de Rham cover} if each $U_i$ and all finite intersections of
    sets in $\mathcal{U}$ is de Rham. If $\mathcal{U}$ is also a basis for the
    topology on $M$, we call it a \emph{de Rham basis}.
    Let us now prove the theorem by showing that every smooth manifold is de
    Rham in $6$ steps.

    \underline{\emph{Step 1:}}
    Firstly, let us show that if $\left\{ M_j \right\}$ is any countable
    collection of de Rham manifolds, then their disjoint union $\bigsqcup_j
    M_j$ is de Rham.
    This trivially follows from properties of both singular and de Rham cohomology.
    Indeed, the inclusion maps $\iota_j \colon M_j \to \bigsqcup_j M_j$ induce
    isomorphisms between the cohomology groups of the above disjoint union and
    the direct product of cohomology groups of the manifolds $M_j$. Then by
    naturality, $\ell$ commutes with this isomorphisms.

    \underline{\emph{Step 2:}}
    Secondly, we show that every convex open subset of $\R^n$ is de Rham.  By
    the Poincar\'e lemma,\footnote{For more topological proof, consider
    \cite[Thm. 17.14]{Lee2012} and the surrounding chapter on homotopy
    invariance. For those preferring direct calculation, read
    \cite[4.18]{Warner1983}.} for any such subset $U$ we have $H_{\text{dR}}^k(U) = 0$ for all $k \neq 0$.
    Since $U$ is clearly homotopy equivalent to a singleton,
    by homotopy invariance and properties of singular cohomology,
    the singular cohomology groups are also trivial for $k \neq 0$.
    In the case where $k = 0$, we have by definition that $H_{\text{dR}}^0(U)$ is the $1$-dimensional real vector space consisting of constant functions,
    and $H^0(U ; \R) \iso \Hom(H_0(U), \R)$ is also clearly one dimensional, since it is generated by any one singular $0$-simplex (all of them are homologous).
    Now, if $\sigma \colon \Delta^0 \to M$ is a singular $0$-simplex (which is trivially smooth),
    and $f$ is the constant function $1$,
    then
    \[
        \ell[f][\sigma]
        = \int_{\Delta^0} \sigma^*f
        = (f \circ \sigma)(0)
        = 1
    \]
    Thus $\ell \colon H_{\text{dR}}^0(U) \to H^0(U; \R)$ is a non-zero linear
    map between $1$-dimensional real vector spaces and hence an isomorphism.

    \underline{\emph{Step 3:}}
    Thirdly, we prove that if $M$ has a finite de Rham cover, then $M$ is de Rham.
    Suppose $M = U_1 \cup \cdots \cup U_n$ constitutes a de Rham cover.
    As expected, we utilize induction on the number $n$ of sets in the cover.
    The base step for $n = 1$ trivially holds true.
    The first induction step for $n = 2$ will give us the idea of what is going on.
    Suppose next that $M$ has a de Rham cover consisting of two sets $U$ and $V$,
    i.e., the manifolds $U$, $V$, and $U \cap V$ are de Rham.
    We join together the Mayer-Vietoris sequences for the de Rham and singular cohomology with the given de Rham isomorphism

    \adjustbox{scale=0.9, center}{
        \begin{tikzcd}
            H_{\text{dR}}^{k-1}(U) \oplus H_{\text{dR}}^{k-1}(V) \arrow[r] \arrow[d, "\ell"]
            & H_{\text{dR}}^{k-1}(U \cap V) \arrow[r] \arrow[d, "\ell"]
            & H_{\text{dR}}^k(M) \arrow[r] \arrow[d, "\ell"]
            & H_{\text{dR}}^k(U) \oplus H_{\text{dR}}^k(V) \arrow[r] \arrow[d, "\ell"]
            & H_{\text{dR}}^k(U \cap V) \arrow[d, "\ell"] \\
            H^{k-1}(U; \R) \oplus H^{k-1}(V; \R) \arrow[r]
            & H^{k-1}(U \cap V) \arrow[r]
            & H^k(M) \arrow[r]
            & H^k(U) \oplus H^k(V) \arrow[r]
            & H^k(U \cap V)
        \end{tikzcd}
    }
    
    This diagram clearly commutes by naturality of $\ell$.
    By the induction hypothesis, the first, second, fourth, and fifth $\ell$ are isomorphisms,
    hence by the Five Lemma\footnote{See \cite[p. 129]{Hatcher2002}.} so is the middle one.
    In other words, $M$ is de Rham.
    Now, for the general induction step,
    suppose $M = U_1 \cup \cdots \cup U_n$,
    and denote $U = U_1 \cup \cdots \cup U_{n-1}$ and $V = U_n$, then use the argument above.

    \underline{\emph{Step 4:}}
    Next, we prove that if $M$ has a de Rham basis, then $M$ is de Rham.
    Suppose $\left\{ U_\alpha \right\}$ is a de Rham basis for $M$,
    and let $f \colon M \to \R$ be an exhaustion function on $M$.\footnote{Recall the Analysis on Manifolds course, or see \cite[p. 46]{Lee2012}.}
    For each integer $m$ let us define the subsets $A_m, A_m' \subseteq M$ by
    \begin{align*}
        A_m &= \left\{ x \in M ; \; m \leq f(x) \leq m + 1 \right\} \\
        A_m'&= \left\{ x \in M ; \; m - \frac{1}{2} < f(x) < m + \frac{3}{2} \right\}
    \end{align*}
    For each $x \in A_m$ there is a basis open subset containing $x$ that is contained in $A_m'$.
    The collection of all such basis sets is of course an open cover of $A_m$.
    Since $f$ is an exhaustion function, $A_m$ is compact and is therefore covered by finitely many of these basis sets.
    Denote by $B_m$ the union of the above finite collection of sets.
    We now have a finite de Rham cover of $B_m$, so by \emph{Step 3}, $B_m$ is de Rham.
    Notice that $B_m \subseteq A_m'$, so $B_m$ can have non-empty intersection with $B_n$ only when $n \in \left\{ m-1, m, m+1 \right\}$.
    We therefore define
    \[
        U = \bigcup_{m \text{ odd}} B_m, \quad\quad V = \bigcup_{m \text{ even}} B_m
    \]
    Clearly, $U$ and $V$ are disjoint unions of de Rham manifolds, so they are both de Rham by \emph{Step 1}.
    The intersection $U \cap V$ is de Rham as well, because it is the disjoint union $\bigsqcup_m B_m \cap B_{m+1}$,
    and each of $B_m \cap B_{m+1}$ has a finite de Rham cover $\left\{ U_\alpha \cap U_{\beta} \right\}$,
    where $U_\alpha$ and $U_\beta$ are the basis sets used to define $B_m$ and $B_{m+1}$, respectively.
    By \emph{Step 3}, $M = U \cup V$ is de Rham.

    \underline{\emph{Step 5:}}
    We now prove that every open subset of $\R^n$ is de Rham.
    Indeed, let $U \subseteq \R^n$ be an open set.
    Then is a union of Euclidean open balls (basis sets for the topology on $\R^n$),
    each of which is of course convex and hence de Rham.
    Since every finite intersection of open balls is again convex, $U$ has a de Rham basis and is therefore de Rham by \emph{Step 4}.

    \underline{\emph{Step 6:}}
    Lastly, we prove our theorem.
    Since any smooth manifold has a basis of smooth coordinate domains and any
    smooth coordinate domain is diffeomorphic to an open subset of $\R^n$, and
    the same holds true for their intersections, this constitutes a de Rham
    basis. Then $M$ is de Rham by \emph{Step 4}, which completes the proof.
\end{proof}
We have finally proven that the de Rham cohomology groups can be calculated using only topological means,
and are in fact isomorphic to the singular cohomology groups with real coefficients.
In the following section we consider a few applications of this result.

\section{Applications}

There are on the surface two obvious way we can use the equivalence of
cohomology groups given by the de Rham Theorem above.  We can use the
information we have about the topology on our manifold and its singular
(co)homology groups in order to draw conclusions about, e.g., solution to
differential equations on $M$, such as $d\eta = \omega$. The other way is of
course to use the smooth structure and explicit definition of the de Rham
cohomology to then gain information about the topology on $M$.
Let us now consider a few of the uses for the de Rham cohomology and theorem.

\subsection{Calculation}

Let us compute de Rham cohomology groups for some example manifolds.
But first, two elementary propositions that will help us greatly.
\begin{proposition}
    Let $\left\{ M_j \right\}$ be a countable collection of smooth
    $n$-manifolds with or without boundary, and let $M = \bigsqcup_j M_j$. For
    each $k$, the inclusion maps $\iota_j \colon M_j \to M$ induce an
    isomorphism from $H_{\text{dR}}^k(M) \to \bigsqcap_j H_{\text{dR}}^k(M_j)$.
\end{proposition}
\begin{proof}
    The pullback maps $\iota_j^* \colon \Omega^k(M) \to \Omega^k(M_j)$ already induce an isomorphism $\Omega^k(M) \to \bigsqcap_j \Omega^k(M_j)$, defined by
    \[
        \omega \mapsto (\iota_1^*\omega, \iota_2^*\omega,\dots) = (\omega|_{M_1}, \omega|_{M_2},\dots)
    \]
    which then induced an isomorphism in cohomology.
\end{proof}
\begin{proposition}
    Let $M$ be a connected smooth manifold with or without boundary.
    The zeroth de Rham cohomology space $H_{\text{dR}}^0(M)$ is equal to
    the space of constant functions and is therefore $1$-dimensional.
\end{proposition}
\begin{proof}
    Since $B_{\text{dR}}^0(M) = 0$, we have that $H_{\text{dR}}^0(M) = \left\{ f \colon M \to \R ; \; df = 0 \right\}$.
    As $M$ is connected, $df = 0$ precisely when $f$ is constant.
\end{proof}
\begin{corollary}
    Let $M$ be a manifold of dimension $0$.
    Then $H_{\text{dR}}^0(M)$ is a direct product of $1$-dimensional vector spaces and the rest are trivial.
\end{corollary}
\begin{proof}
    Use the two propositions above.
\end{proof}
\begin{corollary}
    For $M = \R^n$ we have $H_{\text{dR}}^0(\R^n) = \R$ and the rest are trivial.
\end{corollary}
\begin{proof}
    $\R^n$ is homotopy equivalent to a single point.
    Now use the previous corollary.
\end{proof}
We next consider the example of the sphere.
\begin{proposition}
    Let $M = \S^n$. Then $H_{\text{dR}}^0(\S^n) = H_{\text{dR}}^n(\S^n) = \R$ and the rest are trivial.
\end{proposition}
\begin{proof}
    We know that by the Mayer-Vietoris exact sequence for singular homology,
    we have $H_0(\S^n;\; \R) = H_n(\S^n; \; \R) = \R$ and the rest are trivial.\footnote{See e.g. \cite[Example 2.46]{Hatcher2002}.}
    Then the same holds for singular cohomology groups with coefficients in $\R$, so the de Rham theorem finishes the proof.
\end{proof}
This implies that all closed $k$-forms on $\S^n$ for $k = 1,\dots,n-1$ are exact, and also yields the following corollary.
\begin{corollary}
    An $n$-form $\omega$ on $\S^n$ is exact if and only if $\int_{\S^n} \omega = 0$.
\end{corollary}
\begin{proof}
    Note that $\omega$ is clearly closed, as all $(n+1)$-forms are zero.
    If $\omega$ is exact, then $\int_{\S^n} \omega = 0$ by Stokes' theorem.
    Conversely, we know that $H_{\text{dR}}^n(\S^n) = \R$ and recall that $\S^n$ is compact, hence all forms on $\S^n$ have compact support.
    This means that for a fixed $n$-form $\omega$ with $\int_{\S^n} \sigma \neq 0$,
    then for any $n$-form $\omega'$ there exists $a \in \R$ such that $\omega' - a\omega$ is exact, that is
    \[
        \omega' - a\omega = d\sigma
    \]
    for some $(n-1)$-form $\sigma$.
    Then by Stokes' theorem
    \[
        \int_{\S^n} \omega' - a \int_{\S^n} \omega = 0
    \]
    and so
    \[
        a = \frac{\int_{\S^n} \omega'}{\int_{\S^n} \omega}
    \]
    If $\int_{\S^n} \omega = 0$, we can now assign $a = 0$, in other words,
    $\omega$ must be exact.
\end{proof}
\begin{remark}
    Furthermore notice that for any compact orientable manifold the statement
    $H_{\text{dR}}^n(M) \iso \R$ where $n = \dim M$ is equivalent to saying
    that $[\omega] \mapsto \int_{M}\omega$ is an isomorphism.\footnote{A simple introduction to this can be found in \cite[Volume 1, pp. 268--271]{Spivak1999Comp}, and also in \cite{Lee2012}.}
\end{remark}
Since $\S^n$ is homotopy equivalent to the punctured $(n+1)$-space
$\R^{n+1}\setminus\left\{ 0 \right\}$, this yields that a closed $n$-form
$\eta$ on $\R^{n+1}\setminus\left\{ 0 \right\}$ is exact if and only if
$\int_{S} \eta = 0$ for some sphere $S\subset M$.  Recall now Example
\ref{exa:piercedplane}, where we used an integral over $\S^1$ to determine
non-exactness of the form on the punctured real plane.  This corollary shows
that the condition used in the example was not only sufficient but also
necessary.

\subsection{Product structure}

As already discussed, singular cohomology does not tell us anything
fundamentally new about our manifold when compared to singular homology, it
merely presents the same information in a more structured way.  By this we of
course mean that the collection of singular cohomology groups can be made into
a graded ring with the use of the \emph{cup product}, which is neatly and
graphically explained in \cite[Section 3.2]{Hatcher2002}.
In this subsection we approach this concept from a different angle.

Recall that the properties of the \emph{wedge product}\footnote{Summed up in \cite[Prop. 14.11]{Lee2012}.}
make the space of differential forms $\Omega(M) = \bigoplus_{i=0}^n\Omega^i(M)$ into a graded ring.
Indeed, for $\omega \in \Omega^k(M)$ and $\eta \in \Omega^l(M)$ we have $\omega\wedge\eta \in \Omega^{k+l}(M)$,
and $\wedge$ is bilinear (implies distributivity of product), associative, and anticommutative, that is
\[
    \omega\wedge\eta = (-1)^{kl}\eta\wedge\omega
\]
If we additionally assume that $\omega$ and $\eta$ are closed forms, we have
\[
    d(\omega\wedge\eta) = d\omega\wedge\eta + (-1)^k\omega\wedge d\eta = 0
\]
so $\omega\wedge\eta$ is also closed and represents some de Rham cohomology class.
To show that this cohomology class depends solely on the cohomology classes of $\omega$ and $\eta$ it is enough to take an exact form $\omega = d\sigma$ and check that $\omega\wedge\eta$ is also exact.
Indeed, we have
\[
    d\sigma\wedge\eta
    = d(\sigma\wedge\eta) - (-1)^{k-1}\sigma\wedge d\eta
    = d(\sigma\wedge\eta)
\]
Hence the wedge product descends to cohomology and $H_{\text{dR}}(M) = \bigoplus_{i=0}^n H_{\text{dR}}^i(M)$ is a graded ring with $[\omega]\wedge[\eta] = [\omega\wedge\eta]$.
We can now use the de Rham theorem to induce a product structure on singular cohomology by
\[
    \alpha \cup \beta = \ell(\ell^{-1}(\alpha)\wedge\ell^{-1}(\beta))
\]
In the way the wedge product of two forms acts on vector fields we see that these two products are the same.
In fact, we have the following theorem.
\begin{theorem}
    The cup product on a smooth manifold $M$ is induced by the wedge product.
\end{theorem}

\subsection{The cap product and Poincar\'e duality}

There is another (external) product on singular homology called the \emph{cap
product} that later gives us the famous \emph{Poincar\'e duality}, drawing an
isomorphism between singular cohomology and singular homology.  In this
subsection we consider how this abstract duality materializes in the world of
de Rham cohomology.

We suppose additionally that our manifold $M$ is
orientable\footnote{(Co)homology in fact gives us a more general notion of
\emph{orientability in coefficients}. Since we are dealing with real
coefficients, we always assume our manifolds to be $\R$-orientable.}, connected,
and compact.
Recall\footnote{Or read \cite[pp. 236--241]{Hatcher2002}.} that we can define the cap product on singular (co)homology as a map
$\cap \colon H_{k+l}(M) \times H^k(M) \to H_l(M)$
that maps elements as follows
\[
    \sigma \cap \alpha = \alpha(\sigma\circ[e_0,\dots,e_k]) \cdot \sigma\circ[e_k,\dots,e_{k+l}]
\]
and it is easy to verify that
\[
    \partial(\sigma \cap \alpha) = (-1)^k (\partial\sigma \cap \alpha - \sigma \cap \delta\alpha)
\]
One of the principal properties of the cap product is its connection to the cup product via
\[
    \beta (\sigma \cap \alpha) = (\alpha \cup \beta) (\sigma)
\]
For $M$ as above there exists a non-zero element in the top singular homology group $[M] \in H_n(X; \; \R)$ called the \emph{fundamental class}.\footnote{Sometimes also orientation class, see \cite[p. 236]{Hatcher2002}.}
We can now state the Poincar\'e duality.
\begin{theorem}
    Let $M$ be a connected, closed, $A$-orientable manifold of dimension $n$, for a ring $A$.
    Then the homomorphism $D \colon H^k(M; \; A) \to H_{n-k}(M ; \; A)$, defined by
    \[
        D(\alpha) = [M] \cap \alpha
    \]
    is an isomorphism for all $k \leq n$.
\end{theorem}

When trying to transfer this to the context of the de Rham cohomology is that we have not defined the \emph{de Rham homology groups}.
In the spirit of duality, we therefore denote $H_k^{\text{dR}}(M) = \left( H_{\text{dR}}^k(M) \right)^*$.
The Poincar\'e duality in the de Rham cohomology setting is then given by integration.
\begin{theorem}
    Let $M$ be an $\R$-orientable, connected, and compact smooth manifold of dimension $n$.
    Then the map $D_{\text{dR}} \colon \Omega^k(M) \to \Omega^{n-k}(M)^*$, defined by
    \[
        D_{\text{dR}}(\omega)(\eta) = \int_{M} \omega \wedge \eta
    \]
    descends to an isomorphism $H_{\text{dR}}^k(M) \to H_{n-k}^{\text{dR}}(M)$.
\end{theorem}
\begin{proof}
    That this map descends into a homomorphism is obvious from the properties
    of the wedge product (and integration) that we discussed in the previous
    section.  Indeed, the very properties that make the wedge product descend
    to a cup product on the de Rham cohomology can be reused here.
    
    For the proof that this is in fact an isomorphism for every $k \leq n$, we
    can just trace the proof of the de Rham theorem and replace the property of
    \emph{being a de Rham manifold} with \emph{being a Poincar\'e manifold}.
    Since the proof of the de Rham theorem is quite lengthy, we omit the specifics here.
\end{proof}
This theorem and the above connection of the cap and the cup product in singular (co)homology gives us an idea on how to realize the cap product itself.
Indeed, let $M$ be as above and let $\iota \colon N \to M$ be an embedding of a submanifold $N$ of dimension $k+l$.
Then the cap product of $[N] \in H_{k+l}^{\text{dR}}(M)$ and a $k$-form $[\omega] \in H_{\text{dR}}^k(M)$ is an element $\alpha \in H_{l}^{\text{dR}}(M)$ given by
\[
    \alpha(\eta) = \int_{N} \iota^*\omega \wedge \iota^*\eta
\]
where $\eta \in H_{\text{dR}}^l(M)$.
It is up to the reader to use the aforementioned properties to verify that this is indeed the cap product on $M$ with real coefficients.

\section{Conclusion}

To conclude, let us just mention that this theory opens up a wide array of directions one could study.
Examples range from harmonic differential forms and Hodge theory, to once again proving old theorems in new ways.
We challenge the reader to, e.g., prove the topological invariance of dimension, i.e., two topological manifolds of different dimensions cannot be homeomorphic, using only knowledge obtained in this essay.

If nothing else, the de Rham theorem holds great aesthetic value, similar to that of the Gauss-Bonnet theorem, once again proving how an ostensibly smooth definition turns out to be nothing but mere topology.

\bibliographystyle{plain}
\bibliography{uni}

\end{document}
