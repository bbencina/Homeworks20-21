\documentclass[a4paper, 12pt]{article}

\usepackage[english]{babel}
\usepackage[margin=0.5in]{geometry}

\usepackage[utf8]{inputenc}
\usepackage[T1]{fontenc}
\usepackage{lmodern}
\usepackage{units}
\usepackage{eurosym}
\usepackage{titlesec}
\usepackage{graphicx}
\usepackage{wrapfig}
\usepackage{color}
%\usepackage{url}
\usepackage{hyperref}
\usepackage{enumerate}
\usepackage{enumitem}
\usepackage{pifont}
\usepackage[normalem]{ulem}

% packages
\usepackage{amsmath}
\usepackage{amssymb}
\usepackage{amsthm}
\usepackage{amsfonts}
\usepackage{mathtools}
\usepackage{tikz-cd}
\usetikzlibrary{babel}
\usepackage{adjustbox}
\usepackage{stmaryrd}

% commonly used math operators
\DeclareMathOperator{\diam}{diam}
\DeclareMathOperator{\diag}{diag}
\DeclareMathOperator{\rank}{rank}
\DeclareMathOperator{\tr}{tr}
\DeclareMathOperator{\im}{im}
\DeclareMathOperator{\dom}{dom}
\DeclareMathOperator{\coker}{coker}
\DeclareMathOperator{\codim}{codim}
\DeclareMathOperator{\pr}{pr}
\DeclareMathOperator{\rad}{rad}
\DeclareMathOperator{\chrs}{char}
\DeclareMathOperator{\len}{len}
\DeclareMathOperator{\Lin}{Lin}
\DeclareMathOperator{\Ann}{Ann}
\DeclareMathOperator{\Ass}{Ass}
\DeclareMathOperator{\Spec}{Spec}
\DeclareMathOperator{\mSpec}{mSpec}
\DeclareMathOperator{\Quot}{Quot}
\DeclareMathOperator{\Tor}{Tor}
\DeclareMathOperator{\Ext}{Ext}
\DeclareMathOperator{\Hom}{Hom}
\DeclareMathOperator{\End}{End}
\DeclareMathOperator{\Aut}{Aut}
\DeclareMathOperator{\Br}{Br}
\DeclareMathOperator{\Gal}{Gal}

% commonly used math objects
\newcommand{\F}{\mathbb{F}}
\newcommand{\A}{\mathbb{A}}
\newcommand{\D}{\mathbb{D}}
\renewcommand{\S}{\mathbb{S}}
\newcommand{\T}{\mathbb{T}}
\newcommand{\B}{\mathbb{B}}
\newcommand{\I}{\mathbb{I}}
\newcommand{\N}{\mathbb{N}}
\newcommand{\Z}{\mathbb{Z}}
\newcommand{\Q}{\mathbb{Q}}
\newcommand{\R}{\mathbb{R}}
\newcommand{\C}{\mathbb{C}}
\renewcommand{\H}{\mathbb{H}}
\renewcommand{\P}{\mathbb{P}}

% commonly used math relations
\newcommand{\iso}{\cong}
\newcommand{\homeo}{\approx}
\newcommand{\htpeq}{\simeq}
\newcommand{\hlgeq}{\sim}
\newcommand{\idtfy}{\longleftrightarrow}

% commonly used math symbols
\newcommand{\closure}[1]{\overline{#1}}
\newcommand{\subideal}{\vartriangleleft}
\newcommand{\supideal}{\vartriangleright}

% numbered environments
\theoremstyle{plain}
\newtheorem{theorem}{Theorem}[section]
\newtheorem{corollary}[theorem]{Corollary}
\newtheorem{exercise}[theorem]{Exercise}
\newtheorem{lemma}[theorem]{Lemma}
\newtheorem{proposition}[theorem]{Proposition}

\theoremstyle{definition}
\newtheorem{definition}[theorem]{Definition}

\theoremstyle{remark}
\newtheorem*{claim}{Claim}
\newtheorem*{remark}{Remark}

\newcounter{excounter}[section]
\newenvironment{Exercise}
    {\refstepcounter{excounter}\underline{\textbf{Ex. \theexcounter:}}}
    {\par\vspace{\baselineskip}}


\title{Differential Geometry - $3^\text{rd}$ homework}
\author{Benjamin Benčina}
\date{\today}

\begin{document}

\maketitle

\begin{Exercise}
    Let $M$ be a smooth manifold.
    \begin{enumerate}[label=(\roman*)]
        \item Let us show that any connected orientable manifold $M$ admits precisely two orientations.

            Suppose $\mathcal{A} = (U_i, \varphi_i)_i$ and $\mathcal{B} = (V_j, \psi_j)_j$ are two compatible and oriented atlases on $M$.
            Suppose in addition that there exists a point $p \in M$ and two neighbourhood charts $(U_i, \varphi_i) \in \mathcal{A}$ and $(V_j, \psi_j) \in \mathcal{B}$
            such that $\det(d(\psi_j\circ\varphi_i^{-1})_{\varphi_i(p)}) > 0$.
            We want to show that $\mathcal{A} \cup \mathcal{B}$ is an oriented atlas.

            Indeed, first take any other possible neighbourhood chart $(U_k, \varphi_k) \in \mathcal{A}$ that contains $p$.
            Then we have
            \begin{align*}
                \det(d(\psi_j \circ \varphi_i)_{\varphi_i(p)})
                &= \det(d(\psi_j \circ \varphi_k^{-1} \circ \varphi_k \circ \varphi_i)_{\varphi_i(p)}) \\
                &= \det(d(\psi_j \circ \varphi_k^{-1})_{\varphi_k(p)}\cdot d(\varphi_k \circ \varphi_i^{-1})_{\varphi_i(p)}) \\
                &= \det(d(\psi_j \circ \varphi_k^{-1})_{\varphi_k(p)})\cdot \underbrace{\det(d(\varphi_k \circ \varphi_i^{-1})_{\varphi_i(p)})}_{> 0} > 0
            \end{align*}
            hence we may take any chart in $\mathcal{A}$ that contains $p$.
            Symmetrically, we may also take any chart in $\mathcal{B}$ that contains $p$.

            Secondly, since $\det$ is a continuous map, the above determinant is greater than zero in the entire (connected) intersection of chart neighbourhoods.
            If it was negative in some $p'$ in the intersection, then there would exist a point on some (compact) path between $p$ and $p'$ where the $\det$ would be zero,
            which would contradict the fact that the derivative there is actually a transition matrix between two consistently oriented local frames.

            Thirdly, since $M$ is a connected manifold, it is path-connected.
            Let $q \in M$ be any other point and let $\gamma$ be a path between $p$ and $q$.
            Since $\gamma$ is compact, we can cover it with finitely many charts of both $\mathcal{A}$ and $\mathcal{B}$, respectively, which sequentially have non-empty intersections (otherwise we have a separation for the interval).
            Let those charts be $(U_i, \varphi_i)$ for $i = 1,\dots, k$ and $(V_j, \psi_j)$ for $j = 1,\dots, l$, where each point of $\gamma$ is in at most two charts $U_i$ and at most two charts $V_j$.
            Then we pick intersection points $x_{i, i+1} \in U_i \cap U_{i+1}$.
            Using the argument from step two sequentially on $p, x_{1, 2}$, $x_{2, 3}, \dots, q$ and invariance of the choice of chart from step one,
            we inductively get that the determinant of the derivative must be greater than zero on any chart neighbourhood intersection of $q$.
            In other words, $\mathcal{A} \cup \mathcal{B}$ is an oriented atlas.

            What we have effectively proved is that such compatibility of orientation is an equivalence relation.
            Since the existence condition is binary (such a point $p$ either exists or not), we get that we have at most two equivalence classes, i.e., possible orientations.
            To see that we actually get two orientations, take any oriented atlas $\mathcal{A} = (U_i, \varphi_i = (x^k))_i$
            and denote $\mathcal{A}^-$ the same atlas, with the difference that instead of the $x^1$ coordinate, we have $-x^1$.
            Then $\mathcal{A}$ and $\mathcal{A}^-$ are obviously compatible but give different orientations.

            \underline{\emph{Alternative solution}:}
            We form the following two sets
            \begin{align*}
                A^+ &= \left\{ q \in M ; \; \text{there exist $(U,\varphi) \in \mathcal{A}$ and $(V,\psi) \in \mathcal{B}$ s.t. $\det(d(\psi\circ\varphi^{-1})_{\varphi(q)}) > 0$} \right\} \\
                A^- &= \left\{ q \in M ; \; \text{there exist $(U,\varphi) \in \mathcal{A}$ and $(V,\psi) \in \mathcal{B}$ s.t. $\det(d(\psi\circ\varphi^{-1})_{\varphi(q)}) < 0$} \right\}
            \end{align*}
            Both are clearly open and $M = A^+ \cup A^-$.
            By the first step from the first solution above (invariance of neighbourhood),
            we also have $A^+ \cap A^- = \emptyset$.
            Since $M$ is connected, and these sets define a separation for $M$,
            one of them must be empty.
            By assumption, $p \in A^+$, hence $M = A^+$ and $\mathcal{A} \cup \mathcal{B}$ is by definition an oriented atlas.
        \item We will prove that $M$ is orientable iff it admits a volume form.
            \begin{itemize}
                \item \underline{($\impliedby$):}
                    % Lee - Diff. Geom., p. 328
                    Let $\omega$ be a nowhere vanishing differential $n$-form on $M$.
                    Take a connected open chart $U$ and let $(x^i)$ be local coordinates on $U$.
                    Then $\omega$ has a local coordinate expression
                    \[
                        \omega = f dx^1 \wedge \dots \wedge dx^n
                    \]
                    for some smooth coefficient function $f$.
                    Since $\omega$ is nowhere vanishing, so must be $f$.
                    We have
                    \[
                        \omega(\frac{\partial}{\partial x^1},\dots,\frac{\partial}{\partial x^n})
                        = f \det I
                        = f
                    \]
                    which is nowhere zero for this local frame, and since $U$ is connected, either always positive or always negative on $U$.
                    If it is negative, we have seen in (1i) we can replace $x^1$ for $-x^1$ and obtain an always positive function $f$ on $U$.
                    Hence compatible charts where $\omega$ is a positive multiple of the form $dx^1\wedge\cdots\wedge dx^n$ form an oriented atlas on $M$.
                    %Hence the atlas of charts (so local coordinates) that send $\omega$ to a positive multiple of the Euclidean volume form $f dx^1\wedge\dots\wedge dx^n$ is oriented.
                    %Hence, the pointwise orientation of tangent vector spaces of $M$ determined by $\omega$ (rather, $f$), is continuous,
                    %that is, the entire local frame $\frac{\partial}{\partial x^1},\dots,\frac{\partial}{\partial x^n}$ is consistently oriented.
                    %Equivalently, the atlas consisting of such charts is oriented and $M$ is orientable.
                \item \underline{($\implies$):}
                    % Spivak - Diff Geom. I, pp. 209-210
                    Let $\mathcal{A} = (U_i, \varphi_i = (x^k))_i$ be an oriented atlas on $M$ and let $(\phi_U)_U$ be a partition of unity subordinate to $\mathcal{A}$.
                    Denote the orientation given by the condition in (1i) by $\mu_p$ (recall the point $p$).
                    For each chart $U$ and its local coordinates choose an $n$-form $\omega_U$ such that ``orients'' the local frame $(\frac{\partial}{\partial x^k})$,
                    that is, for each $p \in U$ for a basis $(v_1,\dots,v_n)$ of $T_pM$ we have
                    \[
                        \omega_U(v_1,\dots,v_n) > 0 \iff (v_1,\dots,v_n) \text{ orient $U$ as $\mu_p$}
                    \]
                    (we can just define $\omega_U$ as in the converse using the second step of (1i)).
                    We now define
                    \[
                        \omega = \sum_{U \in \mathcal{A}} \phi_U \omega_U
                    \]
                    which is now a smooth $n$-form.
                    For each $p \in M$, if the basis $(v_1,\dots,v_n) \in T_pM$ give the same orientation as $\mu_p$
                    then
                    \[
                        (\phi_U\omega_U)(p)(v_1,\dots,v_n) \geq 0
                    \]
                    for each chart neighbourhood $U$ of $p$, with a strict inequality for at least one $U$ (otherwise (1i) fails).
                    Hence $\omega$ is nowhere vanishing.
            \end{itemize}

            Now suppose we have two volume forms $\omega$ and $\Omega$ on $M$ that determine the same orientation.
            We then have $\omega = f dx^1 \wedge \dots \wedge dx^n$ and
            $\Omega = g dx^1 \wedge \dots \wedge dx^n$ for two positive smooth functions $f, g$.
            Hence, $\omega = \frac{f}{g}\Omega$, where $\frac{f}{g}$ is well-defined, positive, and smooth.
            The converse trivially follows from the above proof of the right-to-left implication.
        \item Let $(M, g)$ be an oriented Riemannian manifold, i.e., $g$ is a Riemannian metric on $M$.
            % Lee - Diff. Geom., pp. 342-343
            Let us prove that there exists a unique volume form $\omega_g$ determined by the property:
            for any $p \in M$ if $(v_1,\dots,v_n)$ is an ordered oriented orthonormal basis for $T_pM$, then $\omega_g(v_1,\dots,v_n) = 1$.
            
            As usual, we first prove uniqueness.
            Let $\omega$ be such a form (existence later).
            Since $M$ is oriented, for every $p$ in a neighbourhood chart $U$ the basis $(v_1,\dots,v_n)$ forms an ordered oriented orthonormal local frame $(E_1,\dots,E_n)$ for $U$.
            Let $(\varepsilon^1,\dots,\varepsilon^n)$ be its dual frame.
            Then locally $\omega = f \varepsilon^1 \wedge \dots \wedge \varepsilon^n$.
            The defining condition for $\omega$ above then implies $f = 1$, hence $\omega = \varepsilon^1\wedge\dots\wedge\varepsilon^n$ is unique.

            To prove existence, we define $\omega$ as we have determined it to be above.
            Let $(F_1,\dots,F_n)$ be another ordered oriented orthonormal frame (determined by some other continuous choice of bases pointwise),
            and let $(f^1,\dots,f^n)$ be its dual frame.
            Define $\Omega = f^1\wedge\dots\wedge f^n$, so as above, but for the new frame.
            As in the remark, we can write $F_i = A_i^j E_j$ (we mean pointwise, of course) for some matrix $(A_i^j)$ of smooth functions.
            Since both frames are orthonormal, we get that the matrix $(A_i^j(p))$ is orthogonal for each $p$,
            so its determinant is either $1$ or $-1$.
            By choosing a consistently oriented second frame, we get the positive determinant $1$.
            We now apply $\omega$ to the new frame pointwise
            \[
                \omega(F_1,\dots,F_n) = \det(\varepsilon^j(F_i)) = \det(A_i^j) = 1 = \Omega(F_1,\dots,F_n)
            \]
            Hence, $\omega = \Omega$ (independently of frames) so defining $\omega_g$ locally as in the proof of uniqueness gives a global $n$-form which is nowhere vanishing, that satisfies the desired property.
    \end{enumerate}
\end{Exercise}

\begin{Exercise}
    \begin{enumerate}[label=(\roman*)]
        \item Let us show that $\mathcal{L}_V\alpha$ is $C^\infty(M)$-multilinear in its arguments.

            For addition, this is simply a matter of elementary properties of vector and tensor fields as they are both $C^\infty(M)$-additive.
            To check homogeneity it is enough to check the case where we replace $X_1$ by $fX_1$, by symmetry.
            Indeed, we get
            \[
                \mathcal{L}_V\alpha(fX_1,X_2,\dots,X_k) = \underbrace{V(f\alpha(X_1,\dots,X_2))}_{A_1} \underbrace{-\alpha([V,fX_1],X_2,\dots,X_k)}_{A_2} + r
            \]
            where $r$ is the uninteresting remainder, since $\alpha$ is $C^\infty(M)$-multilinear already.
            We will now deal with the Leibniz rule.
            Firstly, at $A_1$ we get
            \[
                V(f\alpha(X_1,\dots,X_k)) = fV(\alpha(X_1,\dots,X_k)) + V(f)\alpha(X_1,\dots,X_k)
            \]
            and at $A_2$ we get
            \[
                -V(f)\alpha(X_1,\dots,X_k) - f\alpha([V,X_1],X_2,\dots,X_k)
            \]
            so the unwanted parts indeed cancel out.
        \item A quick calculation shows that for any $0$-form $f$ we have $\mathcal{L}_V(df) = d(\mathcal{L}_Vf)$.
            
            On the one hand, we have
            \[
                \mathcal{L}_V(df)(X) = V(df(X)) - df([V,X])
                = V(X(f)) - (V(X(f)) - X(V(f))) = X(V(f))
            \]
            and on the other hand
            \[
                (d(\mathcal{L}_Vf))(X) = X(\mathcal{L}_Vf) = X(V(f))
            \]
        \item We will prove that there holds
            \[
                \mathcal{L}_V(\alpha\otimes\beta) = (\mathcal{L}_V\alpha)\otimes\beta + \alpha\otimes(\mathcal{L}_V\beta)
            \]
            for any two $k-$ and $l-$ tensor fields $\alpha$ and $\beta$.

            Since the operator $\mathcal{L}_V$ is clearly linear in its argument,
            it is enough to consider simple tensor fields.
            Since the tensor product of two simple tensor fields is again a simple tensor field,
            it suffices to only consider one such simple tensor field and prove that locally we have
            \[
                \mathcal{L}_V\alpha = (\mathcal{L}_Vf)dx^{i_1}\otimes \cdots \otimes dx^{i_k} + f\mathcal{L}_V(dx^{i_1})\otimes\cdots\otimes dx^{i_k} + \cdots + f dx^{i_1}\otimes\cdots\otimes\mathcal{L}_V(dx^{i_l})
            \]
            where $\alpha = f dx^{i_1} \otimes \cdots \otimes dx^{i_k}$ locally.

            This equality trivially holds if $\alpha$ is a $0$-tensor field.
            Suppose $\alpha = f dx^i$ is a $1$-tensor field.
            Then on the one hand
            \begin{align*}
                \mathcal{L}_V(fdx^i)(X)
                &= V(fdx^i(X)) - fdx^i([V,X])
                = fV(X^i) + V(f)X^i - f(V(X^i) - X(V^i)) \\
                &= V(f)X^i + fX(V^i)
            \end{align*}
            and on the other hand
            \[
                (\mathcal{L}_Vf)dx^i (X) + f(\mathcal{L}_V(dx^i))(X)
                = V(f) X^i + f dV^i (X) = V(f) X^i + fX(V^i)
            \]
            so the equality indeed holds.
            For higher tensor fields we calculate similarly.
            By (2i) it is enough to look at coordinate vector fields,
            and we clearly get a non-zero result only if the vector fields $(X_{i_j})_j$ are a permutation of the coordinate vector fields $(\partial_{i_j})_j$.
            Without loss of generality, $\alpha = fdx^1\otimes\cdots\otimes dx^k$ and $X_i = \partial_i$.
            If we calculate the left-hand side, we get
            \[
                \mathcal{L}_V(fdx^1\otimes\cdots\otimes dx^k)(\partial_1,\dots,\partial_k)
                = V(f\det I) - \alpha([V,\partial_1],\partial_2,\dots,\partial_k) - \cdots - \alpha(\partial_1,\dots,[V,\partial_k])
            \]
            notice now that
            \[
                [V,\partial_i] = (V(\delta_{i,j}) - \partial_i(V^j))\partial_j = - \partial_i(V^j)\partial_j
            \]
            which turns those minuses into pluses we see on the right-hand side!
            If we calculate the right-hand side as well
            (recall, $\mathcal{L}_V(dx^i)(X) = X(V^i)$),
            we get
            \[
                RHS(\partial_1,\dots,\partial_k)
                = V(f)\det I + f dV^1\otimes\cdots\otimes dx^k(\partial_1,\dots,\partial_k) + f dx^1\otimes\cdots\otimes dV^k(\partial_1,\dots,\partial_k)
            \]
            which is the same as above after we calculate the determinants (we get sums at each term, same as for the left-hand side).
        \item We now have
            \[
                \mathcal{L}_V(\omega\wedge\eta) = (\mathcal{L}_V\omega)\wedge\eta + \omega\wedge(\mathcal{L}_V\eta)
            \]
            for any two $k-$ and $l-$ differential forms $\omega$ and $\eta$.

            Indeed, any differential form is just an alternating linear combination of tensor fields,
            and as we have already noticed, $\mathcal{L}_V$ is a linear operator.
            Concretely,
            \[
                \omega\wedge\eta = \frac{(k+1)!}{k!l!}\text{Alt}(\omega\otimes\eta)
            \]
        \item Let us show that there holds $\iota_V\circ\iota_V = 0$ and
            \[
                \iota_V(\omega\wedge\eta) = (\iota_V\omega)\wedge\eta + (-1)^{k}\omega\wedge(\iota_V\eta)
            \]
            for any two $k-$ and $l-$ differential forms $\omega$ and $\eta$.
            
            The first property is straight-forwardly checked.
            If $k \geq 2$, then 
            \[
                \iota_V\iota_V\omega (X_1,\dots,X_{k-2}) = \omega(V, V, X_1,\dots,X_{k-2}) = 0
            \]
            since $\omega$ is alternating.
            Otherwise this follows from the fact that for a $0$-form we have $\iota_Vf = 0$.
            
            The second property is a bit messier.
            Again it is enough to verify the property for simple forms.
            More generally, we will prove that for a collection of $k$ many simple $1$-forms we have
            \[
                \iota_V(\omega^1 \wedge \cdots \wedge \omega^k) = \sum_{i = 1}^{k} (-1)^{i-1} \omega^i(V)\cdot \omega^1\wedge\cdots\hat{\omega^i}\wedge\cdots\wedge\omega^k
            \]
            from which the property immediately follows by linearity.
            Take $k$ vector fields $V, X_2,\dots,X_k$ (consider $V = X_1$).
            We have to prove that
            \begin{align*}
                \iota_V(\omega^1\wedge\cdots\wedge\omega^k)(X_2,\dots,X_k)
                &= (\omega^1\wedge\cdots\wedge\omega^k)(V, X_2,\dots,X_k) \\
                &= \sum_{i=1}^{k}(-1)^{i-1}\omega^i(V)(\omega^1\wedge\cdots\hat{\omega^i}\wedge\cdots\omega^k)(X_2,\dots,X_k)
            \end{align*}
            We know from Lectures that the left-hand side is the determinant of the matrix
            \[
                A_i^j = \omega^i(X_j)
            \]
            while the right-hand side is by the same argument precisely the expansion of $\det A$ by minors along the first column,
            so equal to $\det A$.
        \item We now prove \emph{Cartan's magic formula} for differential forms:
            \[
                \mathcal{L}_V\omega = \iota_V(d\omega) + d(\iota_V\omega)
            \]

            Here we use induction on $k$.
            If $f$ is a $0$-form, then clearly $\mathcal{L}_Vf = V(f)$ and also
            \[
                \iota_V(df) + d(\iota_Vf) = df(V) + 0 = V(f)
            \]
            Let now $k \geq 1$ and suppose the formula holds for all $l$-forms for $l<k$.
            Let $\omega$ be a $k$-form, locally written as
            \[
                \omega = \sum_{I} \omega_I dx^{i_1}\wedge\cdots\wedge dx^{i_k}
            \]
            Denote $u = x^{i_1}$ and $\beta = \omega_I dx^{i_2}\wedge\cdots\wedge dx^{i_k}$,
            so each term of $\omega$ is locally the sum of $du \wedge \beta$, where $u$ is a smooth function.
            By (2ii) we have $\mathcal{L}_V(du) = d(\mathcal{L}_Vu) = d(V(u))$, and by (2iv)
            \begin{align*}
                \mathcal{L}_V(du \wedge \beta)
                &= (\mathcal{L}_Vu)\wedge\beta + du\wedge(\mathcal{L}_V\beta)
                \overset{\text{I.H.}}{=} d(V(u))\wedge\beta + du \wedge (\iota_V(d\beta) + d(\iota_V\beta))
            \end{align*}
            But since $d$ is an antiderivative, and by (2v) so is $\iota_V$, plus $\iota_Vdu = du(V) = V(u)$ and $d\circ d = 0$,
            we compute
            \begin{align*}
                \iota_V(d(du \wedge \beta)) + d(\iota_V(du\wedge\beta))
                &= \iota_V(-du \wedge d\beta) + d(Vu)\beta - du \wedge \iota_V\beta \\
                &= -(Vu)d\beta + du \wedge (\iota_V(d\beta)) + d(Vu) \wedge \beta + (Vu)d\beta + d(\iota_V\beta)
            \end{align*}
            which is precisely $\mathcal{L}_V(du \wedge \beta)$ by the above.
            Now use linearity to extend this beyond simple forms.
        \item We will now use this formula to prove that for any differential form $\omega$ we have
            \[
                \mathcal{L}_V(d\omega) = d(\mathcal{L}_V\omega)
            \]
            and that for any $1$-form $\omega$ and any two vector fields $X,Y$ on $M$ there holds
            \[
                d\omega(X, Y) = X(\omega(Y)) - Y(\omega(X)) - \omega([X, Y])
            \]

            The first equality is a matter of simple calculation where we will merely use the formula and the fact that $d \circ d = 0$.
            On the one hand we have
            \[
                \mathcal{L}_V(d\omega) = \iota_V(dd\omega) + d(\iota_V(d\omega)) = d(\iota_V(d\omega)) 
            \]
            and on the other hand
            \[
                d(\mathcal{L}_V\omega) = d(\iota_V(d\omega)) + dd(\iota_V\omega) = d(\iota_v(d\omega))
            \]
            so we indeed have equality.

            For the second equality first notice that for any $1$-form $\omega$ we have
            \[
                \mathcal{L}_V\omega(X) = V(\omega(X)) - \omega([V,X])
            \]
            This gives us the idea to consider the difference of appropriate Lie derivatives in two different ways.
            Firstly, by definition, we get
            \begin{align*}
                \mathcal{L}_X\omega(Y) - \mathcal{L}_Y\omega(X)
                &= X(\omega(Y)) - \omega([X, Y]) - (Y(\omega(X)) - \omega([Y, X]) \\
                &= X(\omega(Y)) - Y(\omega(X)) - 2 \omega([X, Y])
            \end{align*}
            Secondly, by the formula, we get
            \begin{align*}
                \mathcal{L}_X\omega(Y) - \mathcal{L}_Y\omega(X)
                &= \iota_X(d\omega)(y) + d(\iota_X\omega)(Y) - \iota_Y(d\omega)(X) - d(\iota_Y\omega)(X) \\
                &= 2 d\omega(X, Y) + Y(\omega(X)) - X(\omega(Y))
            \end{align*}
            Equating both approaches, rearranging, and dividing by $2$ yields the desired equality.
    \end{enumerate}
\end{Exercise}

\begin{Exercise}
    Let $M$ be an orientable smooth manifold and $\omega$ a volume form on $M$.
    \begin{enumerate}[label=(\roman*)]
        \item Let $M = \R^n$ and $\omega = dx^1\wedge\cdots\wedge dx^n$.
            We will calculate $\text{div}_\omega(X)$ for a given vector field $X = X^i\partial_i$ in standard coordinates.

            Firstly, we use Cartan's magic formula to get
            \[
                \text{div}_\omega(X)\omega = \mathcal{L}_X\omega = \iota_X(d\omega) + d(\iota_X \omega) = d(\iota_X \omega)
            \]
            since all $(n+1)$-forms, in particular $d\omega$, are zero.
            We know from Lectures that for $\alpha = \sum_{I}a_Idx^I$ we have
            \[
                d\alpha = \sum_{I} da_I\wedge dx^I
            \]
            so we have to find the local expression of the $(n-1)$-form $\iota_X\omega$.
            For any $n-1$ vector fields $Y_1,\dots,Y_{n-1}$ we have
            \begin{align*}
                \iota_X\omega(Y_1,\dots,Y_{n-1})
                &= \omega(X, Y_1,\dots,Y_{n-1})
                = \sum_{i = 1}^{n}X^i\omega(\partial_i,Y_1,\dots,Y_{n-1}) \\
                &= \sum_{i = 1}^{n}X^i \det
                \begin{bmatrix}
                    0 & Y_1^1 & \cdots & Y_{n-1}^1 \\
                    \vdots &  & & \vdots \\
                    1 & Y_1^i & \cdots & Y_{n-1}^i \\
                    \vdots &  & & \vdots \\
                    0 & Y_1^n & \cdots & Y_{n-1}^n \\
                \end{bmatrix}
                = \sum_{i=1}^{n}(-1)^{i+1}X^i \det A_i
            \end{align*}
            where $A_i$ is the minor of the above matrix with the first column and $i$-th row taken out.
            Hence clearly
            \[
                d(\iota_X\omega)
                = \sum_{i=1}^{n}(-1)^{i+1}\frac{\partial X^i}{\partial x^i}dx^i\wedge (dx^1\wedge \cdots \hat{dx^i}\wedge\cdots dx^n)
                = \left( \sum_{i=1}^{n}\frac{\partial X^i}{\partial x^i} \right) \omega
            \]
            where in the last equality we have to transpose $dx^i$ $(i-1)$ times to the right, so the minuses cancel.
            We have finally got
            \[
                \text{div}_\omega(X) = \sum_{i=1}^{n}\frac{\partial X^i}{\partial x^i},
            \]
            the definition we remember from Analysis 2.
        \item Let now $(M, g)$ be an oriented Riemannian manifold.
            We will show that the Riemannian volume form from (1iii) is expressed as
            \[
                \omega_g = \sqrt{\det[g_{ij}]}dx^1 \wedge \cdots \wedge dx^n
            \]
            in any oriented chart $(U, \varphi = (x^i))$, where $g_{ij} = g(\partial_i, \partial_j)$.

            Indeed, let $(U, \varphi)$ be an oriented smooth chart from the instructions.
            Then as in (1iii) we have $\omega_g = f dx^1 \wedge \cdots \wedge dx^n$ for some positive function $f$.
            Recall, that the condition on $\omega_g$ regards any smooth orthonormal oriented frame,
            so pick such a frame $(E_i)$ defined on a neighbourhood of some point $p \in M$.
            Denote by $(\varepsilon^i)$ its dual frame.
            As in (1iii) we again have that $\partial_i = A_i^jE_j$, so we calculate
            \[
                f
                = \omega_g(\partial_1,\dots,\partial_n)
                = \varepsilon^1\wedge\cdots\wedge\varepsilon^n(\partial_1,\dots,\partial_n)
                = \det(\varepsilon^j\partial_i)
                = \det(A_i^j)
            \]
            Consider now $g_{ij}$
            \[
                g_{ij}
                = g(\partial_i,\partial_j)
                = (A_i^kE_k,A_j^lE_l)
                = A_i^kA_j^l g(E_k, E_l)
                = \sum_{k = 1}^{n}A_i^kA_j^k
            \]
            which is precisely the $(i,j)$-th element of the matrix $A^TA$.
            Hence clearly
            \[
                \det[g_{ij}] = \det(A^TA) = \det(A)^2
            \]
            So we finally have
            \[
                f = \det A = \pm\sqrt{\det[g_{ij}]}
            \]
            Since both frames $(\partial_i)_i$ and $(E_j)_j$ are oriented, the sign is positive.
        \item Let $(M, g)$ be an oriented Riemannian manifold and $(U, \varphi = (x^i))$ and oriented chart.
            Let us show that for a vector field $X$ on $M$, expressed locally as $X = X^i\partial_i$,
            we have
            \[
                \text{div}_{\omega_g}(X) = \frac{1}{\sqrt{\det[g_{ij}]}} \partial_k\left( \sqrt{\det[g_{ij}]}X^k \right)
            \]

            Indeed, since $\omega_g = \sqrt{\det[g_{ij}]}dx^1\wedge\cdots\wedge dx^n$,
            we have that
            \begin{align*}
                \iota_X\omega_g(Y_1,\dots,Y_{n-1}
                &= \omega_g(X, Y_1,\dots,Y_{n-1})
                = \sum_{i = 1}^{n}X^i\omega_g(\partial_i,Y_1,\dots,Y_{n-1}) \\
                &= \sum_{i = 1}^{n}\sqrt{\det[g_{ij}]}X^i \det
                \begin{bmatrix}
                    0 & Y_1^1 & \cdots & Y_{n-1}^1 \\
                    \vdots &  & & \vdots \\
                    1 & Y_1^i & \cdots & Y_{n-1}^i \\
                    \vdots &  & & \vdots \\
                    0 & Y_1^n & \cdots & Y_{n-1}^n \\
                \end{bmatrix}
                = \sum_{i=1}^{n}(-1)^{i+1}\sqrt{\det[g_{ij}]}X^i \det A_i
            \end{align*}
            Now, we get
            \begin{align*}
                d(\iota_X\omega_g)
                &= \sum_{i=1}^{n}(-1)^{i+1}\frac{\partial (\sqrt{\det[g_{ij}]}X^i)}{\partial x^i}dx^i\wedge (dx^1\wedge \cdots \hat{dx^i}\wedge\cdots dx^n) \\
                &= \left( \sum_{i=1}^{n}\frac{\partial (\sqrt{\det[g_{ij}]} X^i)}{\partial x^i} \right) \omega
                = \frac{1}{\sqrt{\det[g_{ij}]}}\left( \sum_{i=1}^{n}\frac{\partial (\sqrt{\det[g_{ij}]} X^i)}{\partial x^i} \right) \omega_g
            \end{align*}
            which is precisely what we wanted to prove.
        \item Let $M$ be an orientable smooth manifold, $\omega$ a volume form on $M$,
            and let $\phi_t^X$ denote the flow of a vector field $X$ on $M$.
            Let us prove that there holds
            \[
                \text{div}_\omega(X) = 0 \iff (\phi_t^X)^*\omega = \omega \text{ for any appropriate $t$}
            \]
            % Lee - Smooth, pp.424-426
            \begin{itemize}
                \item \underline{($\implies$):}
                    This direction follows trivially from the hint.
                    If $\text{div}_\omega(X) = 0$ then by definition $\mathcal{L}_V\omega = 0$ at every point.
                    By the hint, $(\phi_t^X)^*\omega$ is a constant function at every point.
                    Since at $t = 0$ we have identity, we get that $(\phi_t^X)^* \omega = \omega$ for all $t$.
                \item \underline{($\impliedby$):}
                    Suppose $\text{div}_\omega(X) \neq 0$ in at least a point.
                    Then, by continuity, there exists a neighbourhood of that point such that $\text{div}_\omega(X)$ is either positive or negative.
                    Without loss of generality, assume $\text{div}_\omega(X) < 0$ in some $U$.
                    Now take a compact coordinate ball $B \subset U$ and small enough $t > 0$ such that still $\phi_t^X(B) \subset U$.
                    Then by the hint
                    \[
                        \frac{d}{dt}\Big|_{t = 0}\left( (\phi_t^X)^*\omega \right)_p < \omega|_p
                    \]
                    for every point $p \in B$, so $(\phi_t^X)^*\omega < \omega$ for every $p \in B$ since the flow is defined here by our choices of $t$.
                    This proves the contrapositive.
            \end{itemize}
    \end{enumerate}
\end{Exercise}

\begin{Exercise}
    Suppose $(q, p)$ are the standard coordinates on $\R^2 \iso T^*\R$.
    Suppose that $\omega$ is a differential $2$-form on $\R^2$ given by $\omega = dq \wedge dp$
    and let $H \in C^\infty(\R^2)$ be a smooth function.
    Define the Hamiltonian vector field $X_H$ on $\R^2$ by the implicit identity $\iota_{X_H}\omega = dH$.
    \begin{enumerate}[label=(\roman*)]
        \item Let us first express the vector field $X_H$ on $\R^2$ in the coordinates $(q,p)$.
            We write
            \[
                X_H = a \partial_q + b \partial_p
            \]
            for some smooth local coefficient functions.
            On the one hand we have
            \[
                \iota_{a\partial_q + b\partial_p}(dq \wedge dp)
                = a dp - bdq
            \]
            and on the other hand we have
            \[
                dH = \frac{\partial H}{\partial q}dq + \frac{\partial H}{\partial p}dp
            \]
            so by equating we immediately get
            \[
                X_H = \frac{\partial H}{\partial p} \partial_q - \frac{\partial H}{\partial q} \partial_p
            \]
            From this it then directly follows that the integral curves of $X_H$ are given by the system of differential equations
            \[
                \dot{q} = \frac{\partial H}{\partial p} \quad \quad \dot{p} = - \frac{\partial H}{\partial q}
            \]
            
            Let now
            \[
                H(q, p) = \frac{p^2}{2m} + \frac{1}{2}kq^2
            \]
            for some positive real numbers $m,k$.
            Let us calculate the flow of $X_H$, and for this we need to actually calculate the integral curves of $X_H$.
            From the above formulae we get
            \[
                \dot{q} = \frac{p}{m} \quad \quad \dot{p} = -kq
            \]
            By differentiating again and inputting equations, we get
            \[
                \ddot{q} = -\frac{k}{m}q \quad \quad \ddot{p} = -\frac{k}{m}p
            \]
            hence
            \[
                q(t) = A\cos\left(\sqrt{\frac{k}{m}}t\right) + B\sin\left(\sqrt{\frac{k}{m}}t\right) \quad\quad
                p(t) = C\cos\left(\sqrt{\frac{k}{m}}t\right) + D\sin\left(\sqrt{\frac{k}{m}}t\right)
            \]
            Picking a starting point $\gamma(0) = (x, y)$ we get $A = x$ and $C = y$,
            and then by inputting this solutions into the original differential equations,
            we also get $B = \frac{y}{\sqrt{km}}$ and $D = -x \sqrt{km}$.
            Therefore
            \[
                \phi_t^{X_H}(x, y) = \gamma_{(x, y)}(t) = \left(
                x\cos\left(\sqrt{\frac{k}{m}}t\right) + \frac{y}{\sqrt{km}}\sin\left(\sqrt{\frac{k}{m}}t\right),
                y\cos\left(\sqrt{\frac{k}{m}}t\right) - x\sqrt{km}\sin\left(\sqrt{\frac{k}{m}}t\right)
                \right)
            \]
        \item We will now show that $X_H(H) = 0$, indeed
            \[
                X_H(H)
                = \left( \frac{\partial H}{\partial p} \partial_q - \frac{\partial H}{\partial q} \partial_p \right) (H)
                = \frac{\partial H}{\partial p}\frac{\partial H}{\partial q} - \frac{\partial H}{\partial q}\frac{\partial H}{\partial p}
                = 0
            \]
        \item We now want to compute $\text{div}_\omega(X_H)$, $\mathcal{L}_{X_H}\omega$,
            and show that there holds $(\phi_t^{X_H})^*\omega = \omega$.
            For the divergence, we simply use (3i) to get
            \[
                \text{div}_\omega(X_H)
                = \frac{\partial X_H^q}{\partial q} + \frac{\partial X_H^p}{\partial p}
                = \frac{\partial^2 H}{\partial p \partial q} - \frac{\partial^2 H}{\partial q \partial p}
                = 0
            \]
            The Lie differentiation follows immediately from the definition of divergence
            \[
                \mathcal{L}_{X_H}\omega = \text{div}_\omega (X_H) \omega = 0
            \]
            and finally, by (3iv), since the divergence is zero, there holds
            \[
                (\phi_t^{X_H})^*\omega = \omega
            \]
    \end{enumerate}
\end{Exercise}

\end{document}
