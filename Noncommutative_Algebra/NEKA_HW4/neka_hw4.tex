\documentclass[a4paper, 12pt]{article}

\usepackage[slovene]{babel}
\usepackage[utf8]{inputenc}
\usepackage[T1]{fontenc}
\usepackage{lmodern}
\usepackage{units}
\usepackage{eurosym}
\usepackage{amsmath}
\usepackage{amssymb}
\usepackage{amsthm}
\usepackage{amsfonts}
\usepackage{mathtools}
\usepackage{graphicx}
\usepackage{color}
%\usepackage{url}
\usepackage{hyperref}
\usepackage{enumerate}
\usepackage{enumitem}
\usepackage{pifont}
\usepackage{tikz-cd}
\usetikzlibrary{babel}
\usepackage{adjustbox}
\usepackage{stmaryrd}

% set margin and layout here
% in case of beamer, comment this out
\usepackage[margin=0.5in]{geometry}

% commonly used math operators
\DeclareMathOperator{\diam}{diam}
\DeclareMathOperator{\rank}{rank}
\DeclareMathOperator{\im}{im}
\DeclareMathOperator{\coker}{coker}
\DeclareMathOperator{\pr}{pr}
\DeclareMathOperator{\rad}{rad}
\DeclareMathOperator{\chrs}{char}
\DeclareMathOperator{\Lin}{Lin}
\DeclareMathOperator{\Ann}{Ann}
\DeclareMathOperator{\Ass}{Ass}
\DeclareMathOperator{\Spec}{Spec}
\DeclareMathOperator{\mSpec}{mSpec}
\DeclareMathOperator{\Quot}{Quot}
\DeclareMathOperator{\Tor}{Tor}
\DeclareMathOperator{\Ext}{Ext}
\DeclareMathOperator{\Hom}{Hom}
\DeclareMathOperator{\End}{End}
\DeclareMathOperator{\Aut}{Aut}
\DeclareMathOperator{\Br}{Br}

% commonly used math objects
\newcommand{\D}{\mathbb{D}}
\renewcommand{\S}{\mathbb{S}}
\newcommand{\B}{\mathbb{B}}
\newcommand{\I}{\mathbb{I}}
\newcommand{\N}{\mathbb{N}}
\newcommand{\Z}{\mathbb{Z}}
\newcommand{\Q}{\mathbb{Q}}
\newcommand{\R}{\mathbb{R}}
\newcommand{\C}{\mathbb{C}}
\renewcommand{\H}{\mathbb{H}}
\renewcommand{\P}{\mathbb{P}}

% commonly used math relations
\newcommand{\iso}{\cong}
\newcommand{\homeo}{\approx}
\newcommand{\htpeq}{\simeq}
\newcommand{\hlgeq}{\sim}
\newcommand{\idtfy}{\longleftrightarrow}

% commonly used math symbols
\newcommand{\closure}[1]{\overline{#1}}
\newcommand{\subideal}{\vartriangleleft}
\newcommand{\supideal}{\vartriangleright}

% cool environment I sometimes use
%\definecolor{bostonuniversityred}{rgb}{0.8, 0.0, 0.0}
%
%\newenvironment{matematika}[1]{
%\textcolor{bostonuniversityred}{\underline{\textsc{#1:}}}
%}{
%}

\newcounter{excounter}[section]
\newenvironment{Exercise}
    {\refstepcounter{excounter}\underline{\textbf{Ex. \theexcounter:}}}
    {\par\vspace{\baselineskip}}

% title data - MODIFY
\title{Noncommutative Algebra, $4^{\text{th}}$ homework}
\author{Benjamin Benčina, 27192018}

\begin{document}

\maketitle

\begin{Exercise}
    Let $Q_1$ and $Q_2$ be quaternion $F$-algebras with $\chrs F \neq 2$.
    We will show the following statements are equivalent:
    \begin{enumerate}[label=(\alph*)]
        \item There exist $a, b, b' \in F^{-1}$ such that $Q_1 \iso \left( \frac{a, b}{F} \right)$ and $Q_2 \iso \left( \frac{a, b'}{F} \right)$.
        \item $Q_1$ and $Q_2$ have a common subfield of dimension $2$ over $F$.
        \item $Q_1$ and $Q_2$ have a common splitting field of dimension $2$ over $F$.
    \end{enumerate}

    Firstly, assume that $F$ does indeed have an extension of degree $2$ (e.g. $F$ must not be algebraically closed).
    Such extensions must necessarily be algebraic, namely extended by a single element. Wherever an extension is given, it is given without loss of generality.
    Secondly, by ``the theorem'' we refer to a theorem from the lectures from Chapter 4.3, consisting of parts (a) and (b) and describing the relationships between splitting fields (and the relative Brauer group) and self-centralizing fields.
    \begin{itemize}
        \item \underline{$(b)\implies (c)$:}
            Let $K$ be such an extension of $F$ with $[K : F] = 2$.
            Since $C(K)$ is a subalgebra in both $Q_1$ and $Q_2$, $K$ must be a self-centralizing field in both $Q_1$ and $Q_2$.
            By part (a) of the theorem, $K$ is a splitting field for both algebras.
        \item \underline{$(a) \implies (b)$:}
            Recall we can always pick $a$ to not be a square (from Homework 2).
            Consider $K = F(\sqrt{a})$.
            Since $a$ is not a square, clearly $[K : F ] = 2$ and $K$ is a subfield of both $Q_1$ and $Q_2$ (we are really looking at $F + iF$).
        \item \underline{$(c) \implies (a)$:}
            Let $K$ be an extension of $F$ with $[K : F] = \deg Q_1 = \deg Q_2 = 2$ that splits both algebras.
            By part (b) of the theorem, we get that $K$ is a self-centralizing subfield in both $Q_1$ and $Q_2$ (incidentally proving (b) on the way).
            This means $C(K)$ is a subalgebra in both quaternion algebras of dimension $2$, generated by ${1, c}$, where $c \in Q_1\setminus F, Q_2\setminus F$.
            But since $[K : F] = 2$, $c^2 \in F\setminus \left\{ 0 \right\}$ ($c$ will be our common basis element $i$).
            Denote $a = c^2$ and write $c = y_1 i_1 + z_1 j_1 + w_1 i_1 j_1 \in Q_1$ (without loss of generality we can omit the pure field term).
            Then clearly $c$ anticommutes with $j_1$. If we write the analogue for $c \in Q_2$, we see also that $c$ anticommutes with $j_2$.
            Hence, we get $Q_1 = \left( \frac{a, b_1}{F} \right)$ and $Q_2 = \left( \frac{a, b_2}{F} \right)$.
    \end{itemize}
\end{Exercise}

\begin{Exercise}
    Let $A$ be a central simple $k$-algebra and $\text{Nrd}\colon A \to k$ its reduced norm.
    For any $a \in A$ we define the left multiplication $L_a \in \End_k(A)$ by $L_a(x) = ax$ and the unreduced norm by $N(a) = \det(L_a)$.
    Let us show that $\text{N}(a) = \text{Nrd}(a)^{\deg(A)}$.

    Since we are calculating the reduced norm in field extensions and the reduced norm does not depend on the extension we take, we might as well take a splitting extension.
    It is therefore enough to prove the case for $A \iso M_n(k)$.
    Now view $A \iso M_n(k)$ as a left $M_n(k)$-module via matrix multiplication.
    Furthermore, $A$ is isomorphic to a direct sum of its columns, each of them isomorphic to $k^n$.
    Then $L_a$ can be viewed as an $n^2 \times n^2$ block-diagonal matrix, each of the $n \times n$ diagonal blocks acting on one of the columns.
    By definition of $L_a$, each of these blocks is the matrix $a$ and has determinant $\det(a)$, hence
    \[
        \text{N}(a) = \det L_a = \det(a)^n = \text{Nrd}(a)^{\deg(A)}.
    \]
\end{Exercise}

\begin{Exercise}
    Let $A$ be a central simple $F$-algebra with $\chrs F \neq 2$.
    Let us show that $[A] = [Q] \in \Br(F)$ for some quaternion algebra $Q$ $\iff$ $A$ has a separable splitting field of degree $2$.
    \begin{itemize}
        \item \underline{$(\implies):$}
            Recall that there are only $2$ possibilities for a quaternion algebra, either $Q \iso M_2(F)$ or $Q$ is a central division algebra.
            In the first case, $[A] = [F] = 1$ in the Brauer group of $F$, so any extension of $F$ is also splitting\footnote{
            Beware that similar considerations as in (1) mush be made, namely, $F$ must allow algebraic extensions.
            If $F$ is, say, algebraically closed, then no such extensions exist, yet $\Br(F) = 1$.}.
            Of course separable extension over $F$ of degree $2$ exist, since $\chrs F \neq 2$ (the problem is that in general they do not split $A$).
            In the second case, $A \iso M_n(Q)$ by the Wedderburn Structure Theorem, but $Q$ is a non-commutative central division algebra, so by the Jacobson-Noether Theorem there exists $c \in Q$ which is separable over $F$.
            Then $F(c)$ is separable and contained in $Q$, so by Koethe's Theorem there exists a separable maximal subfield of $Q$ that contains $F(c)$ and splits $Q$ (and hence $A$).
            By dimension count, it must have degree $2$ over $F$.
        \item \underline{$(\impliedby):$}
            Suppose there exists a separable field extension $[K : F] = 2$ that splits $A$.
            Suppose in addition that $[A] \neq 1$ in the Brauer group of $F$.
            By the Wedderburn Structure Theorem, $A \iso M_m(D)$ for a unique central division $F$-algebra $D$.
            We want to show that $D \iso Q$ for some quaternion division algebra $Q$, or equivalently, that $\dim_F D = 4$.
            Notice that by our additional assumption, we have $\dim_F D \geq 4$.
            Since $K$ splits $A$ and $A \iso M_m(D)$, $K$ splits $D$ as well.
            Without loss of generality we can view $K \subseteq D$.
            By the exercise from Tutorials about Koethe's theorem, $K$ is in fact a maximal subfield in $D$ (here we use the fact that $K$ is separable).
            By the theorem we referenced in (1), since $[K : F] = 2$ and $K$ is maximal in $D$, it must be the case that $\dim_F D = 4$.
    \end{itemize}
\end{Exercise}

\begin{Exercise}
    We will determine the Brauer group of $\C(t)$ and $\R(t)$.
    \begin{itemize}
        \item
            For the first part we will prove that $\C(t)$ is a $C_1$-algebra.
            By an exercise from Tutorials, any central simple algebra over $\C(t)$ will then just be isomorphic to $\C(t)$, so $\Br(\C(t)) = 1$.

            Let $F$ be a homogeneous polynomial of degree $d$ in $\C(t)[f_1,\dots,f_n]$, where $n > d$.
            Since we are considering the equation $F \equiv 0$, we can clearly just rid ourselves of the denominators of all coefficients, so $F \in \C[t][f_1,\dots, f_n]$.
            Take a natural number $N > 0$ and consider the change of variables
            \[
                f_i := \Sigma_{j=0}^{N}a_{ij}t^j
            \]
            for new variables $a_{ij}$. Now substitute this into $F$ and group by powers of $t$ to obtain the following equation
            \[
                0 = F(f_1, \dots, f_n) = \Sigma_{l=0}^{dN+r} F_l(a_{1,0}, \dots, a_{n, N}) t^l
            \]
            where $r$ is the maximal degree of all the coefficients of $F$ and $F_l$ are homogeneous polynomials over $\C$ in the variables $a_{ij}$.
            This equation has a solution precisely when there exist elements $a_{ij} \in \C$ such that $F_l(a_{1,0}, \dots, a_{nN}) = 0$ for all $l = 0, \dots, dN+r$.
            We now have $dN + r + 1$ equations in $n(N+1)$ variables, which need to have a common solution in $\C$.
            Since $r$ is a constant and $d < n$, for a large enough $N$ we have $dN + r + 1 < n(N+1)$.
            Since $\C$ is algebraically closed (in particular, it is also infinite and $C_1$) and we have more variables than (homogeneous) equations in $\C$, we have non-trivial solutions.
            Hence $\C(t)$ is a $C_1$-algebra.
        \item
            We can immediately see how the above approach fails for $\R(t)$.
            Indeed, $\R$ is of course not algebraically closed and, say, $F(x, y, z) = x^2 + y^2 + z^2$ has no non-trivial zero, so $\R$ is not a $C_1$-algebra.
            Then of course $\R(t)$ cannot be a $C_1$-algebra as well, since we can view $\R$ as its subalgebra.

            The problem we are facing is by the system of equations in the previous point analogous to the problem of $x^2 = -1$ not having a solution in $\R$.
            We thus use the proof of Frobenius' Theorem to obtain that $\Br(\R(t)) = \left\{ [\R(t)], [\H(t)] \right\}$, since $\H(t) \otimes \H(t) \iso (\H\otimes\H)(t) \iso M_n(\R)(t) \iso M_n(\R(t))$ by examining coefficients\footnote{
            Note that the solution of both points is somewhat motivated by the observation that any finite-dimensional division algebra over $k(t)$ yields a finite-dimensional division algebra over $k$ through coefficients.
            E.g. any rational function over $\H$ can be seen as a sum of rational functions over $\R$ grouped by the basis of $\H$.
            Likewise, for the $\R(t)$-algebra $\H(t)$ (still grouped by the basis of $\H$).}.
    \end{itemize}
\end{Exercise}

\begin{Exercise}
    Let $A$ be a central simple $k$-algebra. Let $f \colon A \to A$ be an involution of $A$.
    We will do the following:
    \begin{enumerate}[label=(\alph*)]
        \item Describe all involutions of $A$ using $f$.
        \item Show that $M_n(A)$ admits an involution.
    \end{enumerate}
    \begin{enumerate}[label=(\alph*)]
        \item
            By the immediate corollary to the Skolem-Noether Theorem, we have that every $\sigma \in \Aut A$ is inner, so there exists $\alpha \in A$ such that $\sigma (z) = \alpha z \alpha^{-1}$.
            Then
            \[
                \sigma \circ f (xy) = \alpha f(y) f(x) \alpha^{-1} = \alpha f(y) \alpha^{-1} \alpha f(x) \alpha^{-1} = (\sigma \circ f(y)) (\sigma \circ f(x))
            \]
            so $\sigma \circ f$ is again an involution.
            Also note that since $f \colon A \to A^{\text{op}}$ is a homomorphism (indeed, $f$ is an antihomomorphism), by Skolem-Noether Theorem, these are the only involutions.
        \item
            We already know that $M_n(k)$ admits an involution, namely the transposition map $T$.
            Since $M_n(k) \otimes A \iso M_n(A)$, the map $F = T \otimes f$ is a good candidate.
            Indeed, by definition of the tensor product, it is clearly both $k$-linear and $F^2 = id$.
            To verify it is an involution, calculate
            \[
                F(X \otimes y \cdot Z \otimes w) = F(XZ \otimes yw) = Z^{T}X^{T} \otimes f(w) f(y) = Z^{T}\otimes f(w) \cdot X^{T}\otimes f(x) = F(Z \otimes w)F(X \otimes y)
            \]
            Here we relied on the lectures that the product of simple tensors is indeed well-defined.
            Since $F$ is $k$-linear, extend this calculation to the entire tensor product.
    \end{enumerate}
\end{Exercise}
\end{document}

%% TEMPLATES
% lists
%\begin{enumerate}[label=(\alph*)]
% diagram
%\adjustbox{scale=1, center}{
%	\begin{tikzcd}
%		\R_n \arrow[d, "\varphi_n"] \arrow[r, "\Phi"] & \R_m \arrow[d, "\varphi_m"] \\
%		\R \arrow[r, "\widetilde{\Phi}"] & \R
%	\end{tikzcd}
%}
% figure
%\begin{figure}[h]
%	\centering
%	\includegraphics[scale=0.4]{fig}
%	\caption{caption}
%	\label{fig:label}
%\end{figure}
% beamer
%\documentclass[a4paper, 12pt]{beamer}
%\usetheme{CambridgeUS}
%\usecolortheme{beaver}
%\usefonttheme{structuresmallcapsserif}
