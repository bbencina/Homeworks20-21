\documentclass[a4paper, 12pt]{article}

\usepackage[english]{babel}
\usepackage[margin=0.5in]{geometry}

\usepackage[utf8]{inputenc}
\usepackage[T1]{fontenc}
\usepackage{lmodern}
\usepackage{units}
\usepackage{eurosym}
\usepackage{titlesec}
\usepackage{graphicx}
\usepackage{wrapfig}
\usepackage{color}
%\usepackage{url}
\usepackage{hyperref}
\usepackage{enumerate}
\usepackage{enumitem}
\usepackage{pifont}
\usepackage[normalem]{ulem}

% packages
\usepackage{amsmath}
\usepackage{amssymb}
\usepackage{amsthm}
\usepackage{amsfonts}
\usepackage{mathtools}
\usepackage{tikz-cd}
\usetikzlibrary{babel}
\usepackage{adjustbox}
\usepackage{stmaryrd}

% commonly used math operators
\DeclareMathOperator{\diam}{diam}
\DeclareMathOperator{\diag}{diag}
\DeclareMathOperator{\rank}{rank}
\DeclareMathOperator{\tr}{tr}
\DeclareMathOperator{\im}{im}
\DeclareMathOperator{\dom}{dom}
\DeclareMathOperator{\coker}{coker}
\DeclareMathOperator{\codim}{codim}
\DeclareMathOperator{\pr}{pr}
\DeclareMathOperator{\rad}{rad}
\DeclareMathOperator{\chrs}{char}
\DeclareMathOperator{\len}{len}
\DeclareMathOperator{\Lin}{Lin}
\DeclareMathOperator{\Ann}{Ann}
\DeclareMathOperator{\Ass}{Ass}
\DeclareMathOperator{\Spec}{Spec}
\DeclareMathOperator{\mSpec}{mSpec}
\DeclareMathOperator{\Quot}{Quot}
\DeclareMathOperator{\Tor}{Tor}
\DeclareMathOperator{\Ext}{Ext}
\DeclareMathOperator{\Hom}{Hom}
\DeclareMathOperator{\End}{End}
\DeclareMathOperator{\Aut}{Aut}
\DeclareMathOperator{\Br}{Br}
\DeclareMathOperator{\Gal}{Gal}

% commonly used math objects
\newcommand{\F}{\mathbb{F}}
\newcommand{\A}{\mathbb{A}}
\newcommand{\D}{\mathbb{D}}
\renewcommand{\S}{\mathbb{S}}
\newcommand{\T}{\mathbb{T}}
\newcommand{\B}{\mathbb{B}}
\newcommand{\I}{\mathbb{I}}
\newcommand{\N}{\mathbb{N}}
\newcommand{\Z}{\mathbb{Z}}
\newcommand{\Q}{\mathbb{Q}}
\newcommand{\R}{\mathbb{R}}
\newcommand{\C}{\mathbb{C}}
\renewcommand{\H}{\mathbb{H}}
\renewcommand{\P}{\mathbb{P}}

% commonly used math relations
\newcommand{\iso}{\cong}
\newcommand{\homeo}{\approx}
\newcommand{\htpeq}{\simeq}
\newcommand{\hlgeq}{\sim}
\newcommand{\idtfy}{\longleftrightarrow}

% commonly used math symbols
\newcommand{\closure}[1]{\overline{#1}}
\newcommand{\subideal}{\vartriangleleft}
\newcommand{\supideal}{\vartriangleright}

% numbered environments
\theoremstyle{plain}
\newtheorem{theorem}{Theorem}[section]
\newtheorem{corollary}[theorem]{Corollary}
\newtheorem{exercise}[theorem]{Exercise}
\newtheorem{lemma}[theorem]{Lemma}
\newtheorem{proposition}[theorem]{Proposition}

\theoremstyle{definition}
\newtheorem{definition}[theorem]{Definition}

\theoremstyle{remark}
\newtheorem*{claim}{Claim}
\newtheorem*{remark}{Remark}

\newcounter{excounter}[section]
\newenvironment{Exercise}
    {\refstepcounter{excounter}\underline{\textbf{Ex. \theexcounter:}}}
    {\par\vspace{\baselineskip}}


\allowdisplaybreaks

\title{Introduction to Harmonic Analysis - Exam Homework}
\author{Benjamin Benčina}
\date{\today}

\begin{document}

\maketitle

\begin{Exercise}
    Let us prove that for all $f, g \in L^2(\T)$ we have
    \[
        || f * g ||_{L^2(\T)}^2 \leq ||f * f ||_{L^2(\T)} ||g * g||_{L^2(\T)}
    \]
    
    For ease of writing, we denote $|| \cdot ||_{L^2(\T)}$ as $|| \cdot ||_2$.
    From a Corollary from the Lectures we know that
    \begin{equation}
        || f ||_2^2 = \sum_{n \in \Z} | \hat{f}(n) |^2
        \label{eq:L2norm}
    \end{equation}
    holds for all $f \in L^2(\T)$.
    We know from the Measure Theory course (and our Lectures) that since $\T$
    is a space with finite measure, it follows from H\"older's inequality that
    $L^2(\T) \subseteq L^1(\T)$, hence $f, g \in L^1(\T)$.
    This is convenient, as a Theorem from the Lectures, describing the
    properties of convolution, also states that
    \begin{equation}
        \widehat{f * g}(n) = \hat{f}(n) \cdot \hat{g}(n)
        \label{eq:fouriercoefprod}
    \end{equation}
    for all $f, g \in L^1(\T)$.
    We can now finally estimate
    \begin{align*}
        || f * g ||_2^2
        &\overset{\eqref{eq:L2norm}}{=} \sum_{n \in \Z} |\widehat{f * g}(n)|^2
        \overset{\eqref{eq:fouriercoefprod}}{=} \sum_{n \in \Z} | \hat{f}(n) \hat{g}(n) |^2
        = \sum_{n \in \Z} |\hat{f}(n)|^2 \cdot |\hat{g}(n)|^2 \\
        &\overset{\text{H\"older}}{\leq}
        \sqrt{\sum_{n \in \Z} \left( |\hat{f}(n)|^2 \right)^2} \cdot
        \sqrt{\sum_{n \in \Z} \left( |\hat{g}(n)|^2 \right)^2}
        \overset{\eqref{eq:fouriercoefprod}}{=}
        \sqrt{\sum_{n \in \Z} |\widehat{f * f}(n)|^2 } \cdot
        \sqrt{\sum_{n \in \Z} |\widehat{g * g}(n)|^2 } \\
        &\overset{\eqref{eq:L2norm}}{=} || f * f ||_2 \cdot || g * g ||_2
    \end{align*}
    which finishes the proof.
\end{Exercise}

\begin{Exercise}
    Suppose a continuous function $f \colon \T \to \C$ is given by
    \[
        f(x) = \sum_{n=1}^{\infty}\frac{e^{2\pi i n x}}{n^\alpha}
    \]
    where $\alpha > 1$. We will show that $f$ is smooth on $\T\setminus\left\{ 0 \right\}$.

    We denote $z = e^{2\pi i x} \in \T = \S^1 \subset \C$,
    hence the above sum transforms into
    \[
        f(z) = \sum_{n = 1}^{\infty} \frac{z^n}{n^\alpha}
    \]
    Note that the above identification does not spoil smoothness, since $x
    \mapsto t = e^{2\pi i x}$ is a $C^\infty$-diffeomorphism from $(0, 1)$ to
    $\S^1\setminus\left\{ 1 \right\}$.
    It is now enough to prove that the above complex series converges uniformly
    for all $z \neq 1$ on the unit circle, making the above function analytic
    and hence smooth.\footnote{As seen in the Analysis 2 course.} To achieve
    this, we use the complex analytic version of \emph{Abel's test},\footnote{I
    am almost certain that we have proved this in one of the (Complex) Analysis
    courses, but I can not find it in my old notes, so I will prove it here as
    well. One can also find this in a lot of elementary complex analysis books
    such as \cite[p.\ 80]{Macrobert1954} or \cite[pp.\ 91--92]{Moretti1964}.}
    which states that if a sequence $(a_n)_n$ of decreasing (positive) real
    numbers converges to $0$, then the power series
    \[
        \sum_{n=0}^{\infty} a_n z^n
    \]
    converges (uniformly) everywhere on the closed unit circle, except maybe
    for $z = 1$.  The sum can of course be moved to start from $n = 1$, since a
    single term does not impact convergence of a series.
    Before we continue, let us quickly prove that Abel's test indeed holds true.
    Let us take an arbitrary point $z$ from the closed unit circle with $z \neq 1$.
    For each $n \in \N$, we denote partial sums
    \[
        f_n(z) = \sum_{k = 0}^{n} a_k z^k.
    \]
    By multiplying these functions by $(1 - z)$, we obtain
    \begin{align*}
        (1 - z) f_n(z)
        &= \sum_{k = 0}^{n} a_k (1 - z) z^k
        = \sum_{k = 0}^{n} a_k z^k - \sum_{k = 0}^{n} a_k z^{k+1} \\
        &= a_0 + \sum_{k = 1}^{n}a_k z^k - \sum_{k = 1}^{n+1}a_{k-1}z^k
        = a_0 - a_n z^{n+1} + \sum_{k = 1}^{n}(a_k - a_{k-1}) z^k
    \end{align*}
    Since $a_0$ is a constant and by assumption $a_n z^{n+1} \to 0$ uniformly,
    it only remains to show that the last sum converges.
    We prove a stronger statement, it converges absolutely. Indeed, we calculate
    \[
        \sum_{k = 1}^{\infty} |(a_k - a_{k-1})z^k|
        = \sum_{k = 1}^{\infty} |a_k - a_{k-1}| |z|^k
        \overset{|z|\leq 1}{\leq} \sum_{k = 1}^{\infty} (a_{k-1} - a_k)
    \]
    and use that the sequence $(a_n)_n$ converges to $0$.
    Therefore, the sequence $(1 - z)f_n$ converges uniformly on the closed unit disc.
    For $z \neq 1$ we now simply divide by $(1 - z)$ to obtain the desired result.

    In our case $a_n = \frac{1}{n^\alpha}$, which of course converges to $0$.
    Furthermore, since $\alpha > 1$, the above series converges absolutely
    \[
        \sum_{n = 1}^{\infty} \Big\| \frac{z^n}{n^\alpha} \Big\|
        = \sum_{n = 1}^{\infty} \frac{1}{n^\alpha}
        \overset{\alpha > 1}{<} \infty
    \]
    Abel's test completes the proof, since $f$ is now a uniform limit of smooth
    functions.
\end{Exercise}

\stepcounter{excounter}

\stepcounter{excounter}

\begin{Exercise}
    Let $f \in L^1(\T)$ and suppose that
    \begin{equation}
        |f(x) - f(0)| \leq \frac{1}{(\log|x|)^2}
        \label{eq:5-condition}
    \end{equation}
    for all non-zero $x \in [-\frac{1}{2}, \frac{1}{2}] = \T\setminus\left\{ 0 \right\}$.
    Does it follow that
    \[
        \lim_{n \to \infty}S_nf(0) = f(0)?
    \]
    
    \underline{\textsc{yes}}.
    Let us use \emph{Dini's Criterion} for $x = 0$ and $\delta = \frac{1}{2}$.
    Indeed, we have
    \begin{align*}
        \int_{|y|<\delta} \Big|\frac{f(y)-f(0)}{y}\Big|dy
        &\overset{\eqref{eq:5-condition}}{\leq} \int_{|y|<\delta} \Big|\frac{dy}{y(\log|y|)^2}\Big|
        = \int_{0<y<\delta}\frac{dy}{y(\log|y|)^2} + \int_{-\delta<y<0}\frac{dy}{-y(\log(-y))^2} \\
        &\overset{t=\log(\pm y)}{=} 2 \int_{-\infty}^{\log\frac{1}{2}} \frac{dt}{t^2}
        \overset{\text{ANA1}}{=} \frac{1}{\log 2} + \frac{1}{\log 2} = \frac{2}{\log 2} < \infty
    \end{align*}
    where the last equality is justified by a theorem from the Analysis 1
    course, stating the above generalized integral indeed exists since the
    exponent in the denominator is greater than $1$, which finishes the proof.
\end{Exercise}

\begin{Exercise}
    Let us show that
    \begin{equation}
        \label{eq:6-series}
        \sum_{n=2}^{\infty}\frac{\sin(2\pi nx)}{\log n}
    \end{equation}
    converges for all $x \in \T$, and yet is not the Fourier series of any
    function in $L^1(\T)$.

    For the convergence part we recall \emph{Dirichlet's test} from the
    Analysis 1 course.\footnote{Also found in \cite[p.\ 259]{Voxman1981}.} It
    states that for a sequence $(a_n)_n$ of real numbers and a sequence
    $(b_n)_n$ of complex numbers, such that $(a_n)_n$ is monotone, $\lim_{n \to
    \infty}a_n = 0$, and for every $N\geq 1$ we have $|\sum_{n=1}^{N}b_n| \leq
    M$ for some constant $M$, the sequence $\sum_{n = 1}^{\infty}a_nb_n$
    converges.
    We can now use this test on $(a_n)_n = \frac{1}{\log n}$ ($n \geq 2$) and
    $(b_n) = \sin(2\pi xn)$ for $x \in \T$.  The conditions on $(a_n)_n$ are
    clearly satisfied, as the logarithm is an unbounded strictly increasing
    function on $(0, \infty)$.  Now fix $x \in \T\setminus\left\{ 0 \right\}$
    and take $N \geq 2$, and calculate
    \begin{align*}
        \Big| \sum_{n = 2}^{N} \sin(2\pi xn) \Big|
        &= \Big| \sum_{n = 2}^{N} \frac{e^{2\pi i xn} - e^{-2\pi i xn}}{2i} \Big| \\
        &\overset{\text{geom.}}{=} \Big| \frac{e^{4\pi ix}}{2i} \frac{1 - e^{2\pi i x (N-1)}}{1 - e^{2\pi ix}} -
        \frac{e^{-4\pi ix}}{2i} \frac{1 - e^{-2\pi i x (N-1)}}{1 - e^{-2\pi ix}} \Big| \\
        &= \Big| \frac{e^{2\pi i x(N+1)} - e^{2\pi i x 2}}{2i (e^{2\pi ix} - 1)} -
        \frac{e^{-2\pi i x(N+1)} - e^{-2\pi i x 2}}{2i (e^{-2\pi ix} - 1)} \Big| \\
        &= \Big| \frac{(e^{2\pi ix(N+1)} - e^{2\pi ix2})(e^{-2\pi ix} - 1) - (e^{-2\pi ix(N+1)} - e^{-2\pi ix2}) (e^{2\pi ix} - 1)}{2i (e^{2\pi ix} - 1) (e^{-2\pi ix} - 1)} \Big| \\
        &= \Big| \frac{e^{2\pi ixN} - e^{2\pi ix} - e^{2\pi ix(N+1)} + e^{2\pi ix2} - e^{2\pi ixN} + e^{-2\pi ix} + e^{-2\pi ix(N+1)} - e^{-2\pi ix 2}}{2i (1 - e^{2\pi ix} - e^{-2\pi ix} + 1)} \Big| \\
        &\overset{\text{pairs}}{=} \Big| \frac{\sin(2\pi xN) - \sin(2\pi x) - \sin(2\pi x(N+1)) + \sin(4\pi x)}{2 - 2\cos(2\pi x)} \Big| \\
        &= \Big| \frac{\sin(2\pi xN) \cdot (1 - \cos(2\pi x)) -\sin(2\pi x) \cdot (1 - 2\cos(2\pi x) + \cos(2\pi xN))}{2 - 2\cos(2\pi x)} \Big| \\
        &= \Big| \frac{\sin(2\pi x N)}{2} - \frac{\sin(2\pi x)}{2} + \frac{\sin(2\pi x) \cdot (1 - \cos(2\pi xN))}{2(1 - \cos(2\pi x))} \Big| \\
        &\leq \Big| \frac{\sin(2\pi x N)}{2} \Big| + \Big| \frac{\sin(2\pi x)}{2} \Big| + \Big| \frac{\sin(2\pi x) \cdot (1 - \cos(2\pi xN)) \cdot (1 + \cos(2\pi x)}{2(1 - \cos^2(2\pi x))} \Big| \\
        &\leq \Big| \frac{\sin(2\pi x N)}{2} \Big| + \Big| \frac{\sin(2\pi x)}{2} \Big| + \Big| \frac{(1 - \cos(2\pi xN)) \cdot (1 + \cos(2\pi x)}{2\sin(2\pi x)} \Big| \\
        &\leq 1 + \frac{2}{|\sin(2\pi x)|}
    \end{align*}
    which is independent of $N$.
    The only $x$ that could cause problems is $x = 0$, but for $x = 0$ the
    entire sum at the beginning is equal to $0$ for every $N \geq 2$.
    Dirichlet's test hence guarantees convergence for every $x \in \T$.

    Now suppose that \eqref{eq:6-series} is a Fourier series of some Lebesgue-measurable function $f$.
    By the alternative representation of Fourier series as a series of sines and cosines,
    we see that we have
    \[
        a_n = 0 \quad \forall n \in \N_0, \quad b_1 = 0, \quad b_n = \frac{1}{\log n} \quad \forall n \geq 2
    \]
    We know that $b_n = i(\hat{f}(n) - \hat{f}(-n))$.
    Let us now prove that for every function $f \in L^1(\T)$ we have that the series
    \[
        \sum_{n = 1}^{\infty} \frac{\hat{f}(n) - \hat{f}(-n)}{n}
    \]
    converges.
    Indeed, we calculate
    \begin{align*}
        \sum_{n = 1}^{N} \frac{\hat{f}(n) - \hat{f}(-n)}{n}
        &= \sum_{n = 1}^{N} \int_{0}^{1} f(x) \frac{e^{2\pi inx} - e^{-2\pi inx}}{n} dx \\
        &= 2i \int_{0}^{1} f(x) \sum_{n = 1}^{N} \frac{\sin(2\pi inx)}{n} dx \\
        &= 2i \int_{0}^{1} f(x) S_N\Phi(x) dx
    \end{align*}
    where $\Phi(x)$ is the sawtooth function.
    By the Lebesgue Dominated Convergence Theorem, we have
    \begin{align*}
        \Big| \sum_{n = 1}^{\infty} \frac{\hat{f}(n) - \hat{f}(-n)}{n} \Big|
        &= \Big| \lim_{N \to \infty} (2i \int_{0}^{1} f(x) S_N\Phi(x) dx) \Big| \\
        &\overset{\text{LDC}}{=} \Big| 2i \int_{0}^{1} f(x) S\Phi(x) dx \Big| \\
        &\leq 2 \int_{0}^{1} | f(x) S\Phi(x) | dx \\
    \end{align*}
    which converges as $f \in L^1(\T)$ and $\Phi$ is the sawtooth function, proving the statement.

    Let us now apply this to our case, to get that the series
    \[
        \sum_{n = 2}^{\infty} \frac{1}{n \log n}
    \]
    must converge, which is not true, leading to a contradiction.  Indeed, by a
    theorem from the Analysis 1 course\footnote{This is sometimes called the
    \emph{Cauchy condensation test}, at Analysis 1 lectures we called it just
    \emph{Cauchy's test}.}, which states that if $(a_n)_n$ is a positive
    decreasing sequence, then
    \[
        \sum_{n=1}^{\infty} a_n \text{ converges}
        \iff \sum_{n = 1}^{\infty} 2^n a_{2^n} \text{ converges}.
    \]
    We therefore have
    \begin{align*}
        \sum_{n = 2}^{\infty}\frac{1}{n \log n} \text{ converges}
        &\overset{\text{ANA1}}{\iff} \sum_{n = 2}^{\infty} 2^n \frac{1}{2^n \log(2^n)} \text{ converges} \\
        &\iff \log_2(e) \cdot \sum_{n = 2}^{\infty} \frac{1}{n} \text{ converges}
    \end{align*}
    which we know not to be true.
    Hence our initial series cannot be the Fourier series of any $L^1(\T)$-function.
\end{Exercise}

\stepcounter{excounter}

\begin{Exercise}
    Is there a continuous function $f \colon \T \to \R$ such that
    $\lim_{n\to\infty}|S_nf(0)| = \infty$?

    \underline{\textsc{yes}}.
    For the proof we first recall the Banach-Steinhaus Theorem, also known as
    the Uniform Boundedness Principle, which states that for a Banach space
    $X$, a normed space $Y$, and a collection of bounded linear operators
    $\mathcal{A} \subseteq B(X, Y)$, if $\mathcal{A}_x = \left\{ \| Ax \| ; \;
    A \in \mathcal{A} \right\}$ is bounded for every $x \in X$, then $\left\{
        \| A \| ; \; A \in \mathcal{A} \right\}$ is bounded.\footnote{We proved
    this theorem in both the Functional Analysis and the Introduction to
    Functional Analysis course.} In particular, this holds for every Banach
    space $X$ and a collection of bounded linear functionals $\mathcal{A}
    \subseteq X^*$.  Equivalently, taking the contrapositive, we get that for a
    Banach space $X$ and a collection $\mathcal{A} \subseteq X^*$, if $\sup_{A
    \in \mathcal{A}} \| A \| = \infty$, then there must exist $x \in X$ with
    $\sup_{A \in \mathcal{A}} |Ax| = \infty$.  This special case of the theorem
    is what we will use in the proof.

    We define the functionals $\Lambda_n \in C(\T)^*$ by $\Lambda_n f$ being the $n$-th partial sum of the Fourier series of the continuous function $f$ at $0$, so
    \[
        \Lambda_n f
        = S_n f (0)
        = (D_n * f) (0)
        = \int_{0}^{1} f(t)D_n(t) dt
    \]
    where $D_n$ is the usual Dirichlet kernel
    \[
        D_n(t)
        = \sum_{k = -n}^{n} e^{2\pi i k t}
        = \frac{\sin((2n + 1)\pi t)}{\sin(\pi t)}
    \]
    Firstly, it is easy to see that $\lim_{n \to \infty}\| D_n \|_1 = \infty$,
    for example we proved that $\| D_n \|_1 \sim \log n$.
    Secondly, since clearly $\Lambda_n f = \langle f, D_n \rangle_{C(\T)}$,
    by the Riesz Representation Theorem\footnote{Which we also proved at the Introduction to Functional Analysis course.}
    we have that
    \[
        \| \Lambda_n \|_{C(\T)^*} = \| D_n \|_1 \to \infty
    \]
    The above special case of the Banach-Steinhaus Theorem then yields a function $f \in C(\T)$ such the $\Lambda_n f$ is unbounded,
    that is, its Fourier series at $0$ diverges.

    Another possible proof would be to construct such a function, which is done, e.g., in \cite[Vol. I, p.\ 299]{Zygmund2002},
    and we used a similar construction in an exercise from Tutorials.
\end{Exercise}

\begin{Exercise}
    Let $f \in L^1(\T)$ be a functions such that $\hat{f}(j) = 0$ for all $|j| < N$.
    Let us prove that for all $p \in [1, \infty]$ we have
    \[
        \| f'' \|_p \geq C N^2 \| f \|_p
    \]
    where $C > 0$ is independent of $f$, $p$, and $N$.

    Let us try to find a function $\psi \in L^1(\T)$ such that
    \[
        \psi * f'' = f
    \]
    Suppose we already have $\psi$, we try to calculate its Fourier coefficients
    \[
        \widehat{\psi * f''}(m)
        = \hat{\psi}(m) \cdot \hat{f}(m)
        = (2\pi i m)^2 \hat{\psi}(m) \cdot \hat{f}(m)
    \]
    so we want to have
    \[
        \hat{\psi}(j)
        = \frac{1}{(2\pi i j)^2}
        = -\frac{1}{(2\pi j)^2}
    \]
    for every $|j| \geq N$ (the rest we can define to be whatever will fit the
    next lemma). The question is now whether such a $\psi \in L^1(\T)$ exists,
    and for that we employ \cite[Vol. I, Lemma 1.12]{Muscalu2013}, which states
    that if $(a_n)_{n \in \Z}$ is an even sequence of non-negative numbers that
    tend to zero, for which the following holds
    \[
        a_{n+1} + a_{n-1} - 2a_n \geq 0
    \]
    for every $n$, then there exists a (non-negative) function $g \in L^1(\T)$
    with $\hat{g}(n) = a_n$ for all $n \in \Z$.
    Since this fact is new, let us take a quick detour and prove the lemma,
    going along the lines of \cite{Muscalu2013}.

    Notice that for every real $C^2$-function $\varphi$ with $\varphi'(x)x \to 0$
    and $\varphi(x) \to x$ as $x \to \infty$, the following holds
    \begin{equation}
        \label{eq:9-integral}
        \int_{x}^{\infty} \varphi''(y) (y - x) dy
        = \varphi'(y) (y - x) \Big|_x^\infty - \int_{x}^{\infty} \varphi'(y) dy
        = 0 - (0 - \varphi(x)) = \varphi(x)
    \end{equation}
    In the discrete setting, the second derivative corresponds to
    \[
        \Delta^2 x_{i+1}
        = \Delta x_{i+1} - \Delta x_i
        = x_{i+1} - x_i - x_i + x_{i-1} = x_{i+1} + x_{i-1} - 2x_i
    \]
    so \eqref{eq:9-integral} reads as
    \begin{equation}
        \sum_{n > m} (a_{n+1} + a_{n-1} - 2a_n) (n - m) = a_m
        \label{eq:9-discrete}
    \end{equation}
    for every sequence $(a_n)_n$ with $a_n \to 0$ and $n(a_n - a_{n-1}) \to 0$ as $n \to \infty$.
    Recall now that the Fourier coefficients of the Fej\'er kernel are precisely
    \[
        \hat{K}_n(k) = \left( 1 - \frac{|k|}{n} \right)^+
    \]
    so we may write \eqref{eq:9-discrete} as
    \[
        \sum_{n > m} n (a_{n+1} + a_{n-1} - 2a_n) \hat{K}_n(m) = a_m
    \]
    The required function $f$ can thus be defined by
    \[
        \sum_{n = 1}^{\infty} n (a_{n+1} + a_{n-1} - 2a_n) K_n = f
    \]
    This sum of course converges, since $\|K_n\|_1 = 1$ so the partial sums look like
    \[
        \Big\| \sum_{n = 1}^{N} n (a_{n+1} + a_{n-1} - 2a_n) K_n \Big\|_1
        \leq\sum_{n = 1}^{N} n (a_{n+1} + a_{n-1} - 2a_n)
        = a_0 - (N + 1) a_N + N a_{N+1} \xrightarrow{N \to \infty} a_0
    \]
    The lemma is now proven.

    Define now the sequence
    \[
        \hat\psi_N(j) =
        \begin{cases}
            \frac{1}{(2\pi i j)^2} \quad & |j| \geq N \\
            \frac{1}{(2\pi i N)^2} & |j| < N
        \end{cases}
    \]
    for a fix $N \in \N$.
    We consider this lemma for our sequence for $n > N$
    \[
        \frac{1}{(2\pi i (n+1))^2} + \frac{1}{(2\pi i (n-1))^2} - \frac{2}{(2\pi i n)^2}
        = - \frac{n^2 + 1}{2\pi^2(n^2-1)^2} + \frac{1}{2\pi^2 n^2} \leq 0
    \]
    where the last inequality indeed holds, as it is easy to
    verify\footnote{Preferably with a computer, \texttt{Mathematica}'s
    \texttt{Solve} command did the job.} that this expression has zeros in
    $\pm\frac{1}{\sqrt{3}}$, and we verify manually for $n = N$, which is easy,
    as the sequence is negative and increasing.  The case for $n < N$ is of
    course trivial, everything substracts to $0$.  In other words there exists
    the function $-\psi_N \in L^1(\T)$.  Therefore, $\psi_N$ must exist in
    $L^1(\T)$ as well, that has $\hat\psi_N(j) = \frac{1}{(2\pi i j)^2}$ for
    all $|j| \geq N$.  Since $\| K_n \|_1 = 1$, by the above lemma and the
    Lebesgue Monotone Convergence Theorem, we obtain
    \[
        \| \psi_N \|_1 = \sum_{j = 1}^{\infty} j \left( -\hat\psi_N(j+1) - \hat\psi_N(j-1) + 2\hat\psi_N(j) \right)
    \]
    We quickly notice that in the $k$-th partial sum of the above norm, every term but a select few vanishes,
    because the $j$-th term appears $2j$ times with a positive sign and $j + 1 + j - 1 = 2j$ times with a negative sign.
    What remains is
    \[
        \sum_{j = 1}^{k} j \left( -\hat\psi_N(j+1) - \hat\psi_N(j-1) + 2\hat\psi_N(j) \right)
        = \hat\psi_N(1) + (k + 1)\hat\psi_N(k) - k\hat\psi_N(k+1) \to \hat\psi_N(1)
    \]
    and since clearly $|(k+1)\hat\psi_N(k) - k\hat\psi_N(k+1)| \to 0$ (recall the
    coefficients are of the $k^{-2}$ order), we obtain that $\|\psi_N\|_1$ is
    dominated from above by $\frac{1}{N^2}$, that is, there exists a $C' > 0$,
    such that
    \[
        \|\psi_N\|_1 \leq \frac{C'}{N^2}
    \]
    Now, we simply use Young's inequality
    \[
        \| f \|_p
        = \| \psi_N * f'' \|_p
        \leq \| f'' \|_p \cdot \| \psi_N \|_1
        \leq \| f'' \|_p \cdot \frac{1}{N^2} C'
    \]
    and define $C = \frac{1}{C'}$.
\end{Exercise}

\stepcounter{excounter}

\stepcounter{excounter}

\stepcounter{excounter}

\begin{Exercise}
    Suppose a series with complex terms
    \[
        \sum_{n = 0}^{\infty} a_n
    \]
    converges in the sense of Abel, i.e., the following limit exists
    \[
        \lim_{r \nearrow 1} \sum_{n = 0}^{\infty} a_n r^n.
    \]
    Additionally, assume that the sequence $(a_n)_n$ satisfies
    \[
        \lim_{n \to \infty} n a_n = 0.
    \]
    Let us show that the above series must also converge in the usual sense.

    Let $k \in \N$ arbitrary and let us split the sum along $k$.  Firstly, we
    have
    \[
        \Big| \sum_{n = 0}^{k-1} a_n (1 - r^n) \Big|
        \leq (1 - r) \sum_{n = 0}^{k-1} n|a_n|
    \]
    because clearly
    \[
        \frac{1 - r^n}{1 - r} = 1 + r + \cdots + r^{n-1} \overset{r < 1}{\leq} n.
    \]
    Secondly, if $H_k$ denotes the upper bound of $(n|a_n|)_{n\geq k}$, which
    exists as any convergent sequence is bounded, we have
    \[
        \Big| \sum_{n = k}^{\infty} a_n r^n \Big|
        \leq H_k \frac{1}{k} r^k (1 + r + r^2 + \cdots)
        \leq H_k \frac{1}{k} \frac{r^k}{1 - r}
    \]
    Now consider $r = 1 - \frac{1}{k}$ and estimate
    \begin{align*}
        \Big| \sum_{n = 0}^{k-1} a_n - \sum_{n = 0}^{\infty} a_n r^n \Big|
        &= \Big| \sum_{n = 0}^{k-1} a_n - \sum_{n = 0}^{k-1} a_n r^n - \sum_{n=k}^{\infty} a_n r^n \Big| \\
        &\leq \Big| \sum_{n = 0}^{k-1} (1 - r^n) a_n \Big| + \Big| \sum_{n=k}^{\infty} a_n r^n \Big| \\
        &\leq (1 - 1 + \frac{1}{k})\sum_{n = 0}^{k-1} n|a_n| + H_k\frac{1}{k} \frac{(1 - \frac{1}{k})^k}{1 - 1 + \frac{1}{k}} \\
        &= \frac{1}{k} \sum_{n = 0}^{k-1} n |a_n| + (1 - \frac{1}{k})^k H_k
    \end{align*}
    Since each of the terms on the right-hand side goes to $0$ as $k \to
    \infty$ and $r \nearrow 1$ as $k \to \infty$, we have
    \[
        \sum_{n = 0}^{\infty} a_n = \lim_{k \to \infty} \sum_{n = 0}^{k-1} a_n = \lim_{r \nearrow 1} \sum_{n = 0}^{\infty} a_n r^n
    \]
    which finishes the proof.\footnote{This proof is a slight modification of
    the one found in \cite[p.\ 251]{Bromwich1908}, which is due to Alfred
    Tauber and his original $1897$ paper \cite{Tauber1897}.}
\end{Exercise}

\stepcounter{excounter}

\begin{Exercise}
    Let $BV(\T)$ be the set of all functions $f \colon \T \to \C$ of bounded variation.

    We first prove that $BV(\T)$ is a $\C$-vector space.
    Indeed, it is immediate from the definition of a function of bounded variation that for every scalar $\alpha$ and every $f \in BV(\T)$ we have
    \begin{equation}
        \label{eq:15-abshomog}
        V(\alpha f) = |\alpha| V(f) < \infty
    \end{equation}
    To prove the same holds for addition, take $f, g \in BV(\T)$ and an arbitrary partition $P$ of $\T$.
    Then we have
    \begin{align*}
        \sum_{k} |(f + g)(x_k) - (f + g)(x_{k-1})|
        &= \sum_{k} |f(x_k) + g(x_k) - f(x_{k-1}) - g(x_{k-1})| \\
        &\leq \sum_{k} |f(x_k) - f(x_{k-1})| + \sum_{k} |g(x_k) - g(x_{k-1})| \\
    \end{align*}
    After applying the supremum over all partitions, we obtain
    \begin{equation}
        \label{eq:15-triangle}
        V(f + g) \leq V(f) + V(g) < \infty
    \end{equation}
    
    We then define the map
    \begin{equation}
        \label{eq:15-norm}
        \|f\|_{BV(\T)} = \sup_{y \in \T} |f(y)| + V(f)
    \end{equation}
    and prove it is a norm which turns $BV(\T)$ into a Banach space.
    All of the properties of the norm follow rather easily.
    The triangle inequality follows immediately from \eqref{eq:15-triangle} and the fact that the same holds for supremum.
    Likewise, absolute homogeneity follows directly from \eqref{eq:15-abshomog}.
    This map is also obviously non-negative and both of its terms are non-negative as well,
    hence
    \[
        \|f\|_{BV(\T)} = 0
        \iff \sup_{y \in \T}|f(y)| = 0 \land V(f) = 0
        \iff f = 0
    \]
    To prove completeness, consider a Cauchy sequence $(f_n)_{n \in \N} \subset
    BV(\T)$, i.e., for every $\varepsilon > 0$ there exists a $N \in \N$ such
    that for every $n,m > N$ we have $\| f_n - f_m \|_{BV(\T)} < \varepsilon$.
    In particular, $(f_n)$ is also clearly Cauchy in $\|\cdot\|_\infty$ (this
    is really a direct consequence of \eqref{eq:15-norm}), and hence converges
    uniformly to a bounded function $f \colon \T \to \C$.
    Let us show that $f \in BV(\T)$ and that $f_n \to f$ in the $BV$-norm as well.
    Let $P$ be any partition of $\T$ and let $\varepsilon > 0$.
    By assumption, we can choose $N \in \N$ such that $\| f_n - f_m \|_{BV(\T)} < \varepsilon$ for all $n, m > N$.
    By definition of the variation of a function over a partition $P$, if $f_n
    \to f$ (even pointwise), then $V(f_n;\; P) \to V(f; \; P)$ (indeed, the
    limit acts well in finite sums).  But then for any $n > N$ and $y \in \T$
    we have
    \begin{equation}
        \label{eq:15-complete}
        \begin{aligned}
            |f(y) - f_n(y)| + V(f - f_n; \; P)
            &= \lim_{m \to \infty}\left( |f_m(y) - f_n(y)| + V(f_m - f_n; \; P) \right) \\
            &\leq \sup_{m > N,\; y \in \T} \| f_m - f_n \|_{BV(\T)}
            < \varepsilon
        \end{aligned}
    \end{equation}
    Since this estimate holds for all partitions $P$, we have $\| f - f_n \| <
    \varepsilon$ for any $n > N$. Now $f - f_n \in BV(\T)$ and of course $f_n
    \in BV(\T)$ for all $n > N$, so $f \in BV(\T)$ as well.  Naturally, the
    estimate \eqref{eq:15-complete} shows that $\| f - f_n \|_{BV(\T)} \to 0$.

    Lastly, let us prove that
    \[
        |S_nf(x)| \leq \| f \|_{BV(\T)}
    \]
    for all $n \in \N$ and $x \in \T$.

    \underline{\textbf{A note to the examiner:}}
    I will present two solutions to this problem.  The idea for the first is
    based on \cite[p.\ 52, Problem 11]{Butzer1971}.  The second, alternative
    solution is a more intuitive solution with no combinatorial tricks that I
    came up with after reading one too many real analysis book chapters on
    functions of bounded variation in order to finish this exercise.  If the
    second solution fails, please ignore it and please tell me where, I include
    it only to honor all the time I spent on this at first glance short
    exercise.

    \underline{\emph{Trick solution}:}
    We first notice that
    \begin{equation}
        \label{eq:15-cesaro}
        S_n f(x) = \sigma_n f(x) + \frac{1}{n} \sum_{k = -n}^{n} |k| \hat{f}(k) e^{2\pi i k x},
    \end{equation}
    where $\sigma_n f$ denotes the Ces\`aro mean, defined by
    \[
        \sigma_n f = \frac{1}{n} \sum_{k = 0}^{n-1} S_n f.
    \]
    Indeed, let us write out the right-hand side
    \begin{align*}
        \sigma_n f(x) + \frac{1}{n} \sum_{k = -n}^{n} |k| \hat{f}(k) e^{2\pi i k x}
        &= \frac{1}{n} \sum_{m = 0}^{n-1}\sum_{k = -m}^{m} \hat{f}(k) e^{2\pi i k x} + \frac{1}{n} \sum_{k = -n}^{n} |k| \hat{f}(k) e^{2\pi i k x} \\
        &= \hat{f}(0) +  \frac{1}{n} \sum_{k = 1}^{n-1}(n-k)\left( \hat{f}(k) + \hat{f}(-k) \right) e^{2\pi i k x} + \frac{1}{n} \sum_{k = -n}^{n} |k| \hat{f}(k) e^{2\pi i k x} \\
        &= \sum_{k = -n}^{n} \hat{f}(k) e^{2\pi i k x} = S_n f(x)
    \end{align*}
    Then we notice that for all $f \in BV(\T)$ we have
    \begin{equation}
        \label{eq:15-bvcoef}
        |\hat{f}(k)| \leq \frac{V(f)}{2\pi|k|}
    \end{equation}
    for all $k \neq 0$.
    Indeed, integration per partes yields
    \[
        \hat{f}(k)
        = \int_{\T} f(x) e^{-2\pi i k x} dx
        = (-2\pi i k x)f(x) e^{-2\pi i k x} \Big|_\T + \int_{\T} \frac{e^{-2\pi i k x}}{2\pi i k x} df
        = \int_{\T} \frac{e^{-2\pi i k x}}{2\pi i k x} df
    \]
    where the last equality holds because of periodicity and $df$ is the associated measure to $f$,
    and we have
    \begin{align*}
        \Big| \int_{\T} e^{-2\pi i k x} df \Big|
        &\leq \int_{\T} 1 df
        = \lim_{\overset{\Delta x_i \to 0}{(x_i)_0^n \text{ partition of } \T}} \sum_{k = 1}^{n} f(x_k) - f(x_{k-1}) \\
        &\leq \sup_{P \text{ any partition of } \T} \sum_{P} |f(x_k) - f(x_{k-1})|
        = V(f)
    \end{align*}
    Joining \eqref{eq:15-cesaro} and \eqref{eq:15-bvcoef}, we obtain
    \[
        |S_n f(x) |
        \leq |\sigma_n f(x)| + \frac{1}{n}\sum_{k = -n}^{n} |k| |\hat{f}(k)|
        \leq \| f\|_\infty + \frac{2n|k|}{n} \frac{V(f)}{2\pi |k|}
        \leq \| f\|_\infty + V(f)
        = \| f \|_{BV(\T)}
    \]
    completing the exercise.

    \underline{\emph{Alternative solution}:}
    We first notice that if $f \colon \T \to \C$ is of bounded variation, then
    obviously $\Re f \colon \T \to \R$ and $\Im f \colon \T \to \R$ are real
    functions of bounded variation.  The converse is also true, if $f$ and $g$
    are real functions of bounded variations on $\T$, then $F = f + ig$ is a
    complex function of bounded variation. It is therefore enough to focus on
    real functions of bounded variations, which have a somewhat nicer
    theory.\footnote{See e.g.\ Chapter 13 of \cite{Carothers2000},
    Section 3.5 of \cite{Folland1999}, or Section 32 of \cite{Kolmogorov1975}.}
    Indeed, by \cite[Theorem 13.5]{Carothers2000},
    any real function of bounded variation on $\T$ is a difference of two
    increasing functions,\footnote{The converse also holds called Jordan's
    Theorem, found in e.g. \cite[Corollary 13.6]{Carothers2000}.} which we can
    easily see by defining $v(x) = V(f;\; [0, x])$ for $x \in [0, 1] = \T$ and
    $v(0) = 0$. The function $v$ is then clearly increasing on $\T$, as is $v -
    f$, so we can simply write $f = v - (v - f)$ as a difference of two
    increasing function.
    We show $v$ and $v-f$ are increasing by the following calculation
    \[
        v(y) - v(x)
        = V(f; \; [0,y]) - V(f;\; [0,x])
        = V(f; \; [x, y])
        \geq |f(y) - f(x)|
        \geq 0.
    \]
    for all $y > x$ in $\T$.

    This in particular means $f$ is \emph{quasi-continuous}, i.e., a function
    possessing finite left- and right-hand limits at each point (or just
    countable number of jumps).  Now, since on a compact interval (like $\T$
    can be thought of) any quasi-continuous function is a uniform limit of step
    functions, all step functions are in $L^2(\T)$ (as piecewise continuous
    (even constant) functions), and $L^2(\T)$ is complete, all real functions
    of bounded variation on $\T$ are also in $L^2(\T)$.  This of course also
    means that all complex functions of bounded variation on $\T$ are in
    $L^2(\T)$. Now recall from lectures that $S_n$ projects orthogonally on the
    appropriate finite dimensional supspace of $L^2(\T)$, so by a Theorem from
    the Introduction to Functional Analysis class, its operator norm is bounded
    by $\| S_n \|_{L^2(\T)} \leq 1$.  The operator norms can be compared as
    \begin{equation}
        \| S_n \|_{BV(\T)} \leq \| S_n \|_{L^2(\T)}
        \label{eq:15-opnormcomp}
    \end{equation}
    because the supremum is taken over a smaller set.  The following estimation
    then goes through
    \[
        | S_nf(x) |
        \leq \| S_nf \|_{BV(\T)}
        \leq \| S_n \|_{BV(\T)} \| f \|_{BV(\T)}
        \overset{\eqref{eq:15-opnormcomp}}{\leq} \| S_n \|_{L^2(\T)} \| f \|_{BV(\T)}
        \leq \| f \|_{BV(\T)}
    \]
\end{Exercise}

\begin{Exercise}
    To a continuous function $f \colon \T \to \C$ for every $N \in \N$ we associate its \emph{discrete Fourier coefficients}
    \[
        a_N(n) = \frac{1}{N} \sum_{k = 1}^{N} f(e^{\frac{2\pi ik}{N}}) e^{\frac{-2\pi ikn}{N}}
    \]
    as well as its usual Fourier coefficients
    \[
        a(n) = \int_{\T} f(x) e^{-2\pi inx} dx
    \]
    Let us prove that
    \[
        \lim_{N\to\infty} a_N(n) = a(n)
    \]
    
    Denote $g_n(x) = f(x)e^{-2\pi ixn}$.
    We make a partition of the interval $I = [0, 1]$ into $N$ equal parts,
    i.e., $x_k = \frac{k}{N}$ for $k = 0,\dots,N$.
    Notice that this partition maps to $\T = \S^1$ as $y_k = e^{2\pi i x_k} =
    e^{\frac{2\pi i k}{N}}$.  We now observe that $a_N(n)$ is exactly the right
    Riemann sum of the function $g_n$ on $\T$ for this partition.  Indeed,
    whereas in the usual Riemann sum we take function values on arbitrary
    points of the partition intervals, we always take right-most points for the
    right Riemann sum. Concretely, for an arbitrary partition
    $P = \left\{ x_0<\cdots<x_n \right\}$ of $[0,1]$, we define
    \[
        R_Pf = \sum_{i = 1}^{n} f(x_i)(x_i - x_{i-1})
    \]
    The right Riemann sum is thus a special case of the usual Riemann sum and
    all the convergence theorems hold for it as well.  Hence
    \[
        \lim_{N\to\infty} a_N(n) = \lim_{N\to\infty} R_Ng_n = \int_{\T} g_n(x) dx = a(n)
    \]
    where $R_N$ yields the right Riemann sum given by the above partitioning of
    $I$ into $N$ equal parts.  Note that in this case, the factor of
    $\frac{1}{N}$ plays the role of the interval lengths, which at first glance
    appear to be omitted.
\end{Exercise}

\nocite{Grafakos2014Classical}
\bibliographystyle{ieeetr}
\bibliography{uni.bib}

\end{document}
