\documentclass[a4paper, 12pt]{article}

\usepackage[slovene]{babel}
\usepackage[utf8]{inputenc}
\usepackage[T1]{fontenc}
\usepackage{lmodern}
\usepackage{units}
\usepackage{eurosym}
\usepackage{amsmath}
\usepackage{amssymb}
\usepackage{amsthm}
\usepackage{amsfonts}
\usepackage{mathtools}
\usepackage{graphicx}
\usepackage{wrapfig}
\usepackage{color}
%\usepackage{url}
\usepackage{hyperref}
\usepackage{enumerate}
\usepackage{enumitem}
\usepackage{pifont}
\usepackage{tikz-cd}
\usetikzlibrary{babel}
\usepackage{adjustbox}
\usepackage{stmaryrd}

% set margin and layout here
% in case of beamer, comment this out
\usepackage[margin=0.5in]{geometry}

% commonly used math operators
\DeclareMathOperator{\diam}{diam}
\DeclareMathOperator{\rank}{rank}
\DeclareMathOperator{\im}{im}
\DeclareMathOperator{\coker}{coker}
\DeclareMathOperator{\pr}{pr}
\DeclareMathOperator{\rad}{rad}
\DeclareMathOperator{\chrs}{char}
\DeclareMathOperator{\len}{len}
\DeclareMathOperator{\Lin}{Lin}
\DeclareMathOperator{\Ann}{Ann}
\DeclareMathOperator{\Ass}{Ass}
\DeclareMathOperator{\Spec}{Spec}
\DeclareMathOperator{\mSpec}{mSpec}
\DeclareMathOperator{\Quot}{Quot}
\DeclareMathOperator{\Tor}{Tor}
\DeclareMathOperator{\Ext}{Ext}
\DeclareMathOperator{\Hom}{Hom}
\DeclareMathOperator{\End}{End}
\DeclareMathOperator{\Aut}{Aut}
\DeclareMathOperator{\Br}{Br}
\DeclareMathOperator{\Gal}{Gal}

% commonly used math objects
\newcommand{\D}{\mathbb{D}}
\newcommand{\T}{\mathbb{T}}
\renewcommand{\S}{\mathbb{S}}
\newcommand{\B}{\mathbb{B}}
\newcommand{\I}{\mathbb{I}}
\newcommand{\N}{\mathbb{N}}
\newcommand{\Z}{\mathbb{Z}}
\newcommand{\Q}{\mathbb{Q}}
\newcommand{\R}{\mathbb{R}}
\newcommand{\C}{\mathbb{C}}
\renewcommand{\H}{\mathbb{H}}
\renewcommand{\P}{\mathbb{P}}

% commonly used math relations
\newcommand{\iso}{\cong}
\newcommand{\homeo}{\approx}
\newcommand{\htpeq}{\simeq}
\newcommand{\hlgeq}{\sim}
\newcommand{\idtfy}{\longleftrightarrow}

% commonly used math symbols
\newcommand{\closure}[1]{\overline{#1}}
\newcommand{\subideal}{\vartriangleleft}
\newcommand{\supideal}{\vartriangleright}

% cool environment I sometimes use
%\definecolor{bostonuniversityred}{rgb}{0.8, 0.0, 0.0}
%
%\newenvironment{matematika}[1]{
%\textcolor{bostonuniversityred}{\underline{\textsc{#1:}}}
%}{
%}

\newcounter{excounter}[section]
\newenvironment{Exercise}
    {\refstepcounter{excounter}\underline{\textbf{Nal. \theexcounter:}}}
    {\par\vspace{\baselineskip}}

% start sections with 0
%\setcounter{section}{-1}


% title data - MODIFY
\title{Uvod v funkcionalno analizo, domača naloga}
\author{Benjamin Benčina, 27192018}

\begin{document}

\maketitle

\begin{Exercise}
    Dokažimo, da obstaja tak $\varphi \in (l^\infty)^*\setminus\lbrace 0 \rbrace$,
    da velja $\varphi(x) = 0$ za vsako periodično zaporedje $x \in l^\infty$.

    Naj bo $P$ množica vseh periodičnih zaporedij, torej
    \[
        P = \left\{ x \in l^\infty ; \; \exists N \in \N \quad \forall n \in \N \colon x_{n+N} = x_n \right\}
    \]
    Takoj opazimo, da je $P \leq l^\infty$,
    saj je vsota $N$ in $M$-periodičnega zaporedja $LCM(N,M)$-periodična.
    Po izreku s predavanj velja
    \[
        \closure{P} = \bigcap \left\{ \ker f ; \; f \in (l^\infty)^*, \quad P \subseteq \ker f \right\}
    \]
    Želimo torej dokazati, da $P$ ni gost podprostor.
    Vzemimo $a \neq 0$ in definirajmo
    \[
        \alpha = \left( a,-a,a,a,-a,-a,a,a,a,-a,-a,-a,\dots \right) \in l^\infty
    \]
    torej zaporedje, kjer na $n$-tem koraku dodamo $n$ $a$-jev in $n$ $(-a)$-jev.
    Za vsak $0 < \varepsilon < |a|$ potem krogla $\B(\alpha, \varepsilon)$ ne vsebuje nobenega periodičnega zaporedja.
    Res, če je $x \in l^\infty$ $N$-periodično, ki sledi zaporedju $\alpha$ na razdalji največ $\varepsilon$,
    najkasneje med $2N$ in $3N$ korakov globoko v zaporedje $\alpha$ (koraki definirani kot zgoraj) pade ven iz krogle.
    Prostor $P$ torej ne more biti gost.
\end{Exercise}

\begin{Exercise}
    Naj bosta $X$ in $Y$ Banachova prostora.
    \begin{enumerate}[label=(\alph*)]
        \item
            Denimo, da za $Z \leq X$ obstaja omejen linearen operator $P \colon X \to X$,
            ki zadošča $\im P = Z$ in $P^2 = P$.
            Dokažimo, da je $Z$ zaprt podprostor.

            Naj bo $(z_n)_n$ zaporedje elementov iz $Z$, ki konvergira proti $x \in X$.
            Radi bi pokazali, da $x \in Z$.
            Ker je $\im P = Z$, obstaja zaporedje $(x_n)_n \subset X$,
            da je $z_n = P(x_n)$ za vsak $n \in \N$.
            Sedaj imamo
            \[
                P^2(x_n) = P(z_n) \to P(x) \in Z
            \]
            hkrati pa po idempotentnosti
            \[
                P^2(x_n) = P(x_n) = z_n \to x
            \]
            torej $x \in Z$.
        \item
            Naj bo $T \in \mathcal{B}(X, Y)$ operator z zaprto sliko.
            Dokažimo ekvivalenco spodnjih trditev.
            \begin{enumerate}[label=(\roman*)]
                \item
                    Obstajata omejena linearna operatorja $P\colon X \to X$ in $Q\colon Y \to Y$,
                    ki zadoščata $P^2 = P$, $Q^2 = Q$, $\im P = \ker T$ in $\im Q = \im T$.
                \item
                    Obstaja operator $S \in \mathcal{B}(Y, X)$,
                    ki zadošča $STS = S$ in $TST = T$.
            \end{enumerate}
            \begin{itemize}
                \item \underline{(ii) $\implies$ (i):}
                    Opazimo, da
                    \[
                        (TS)^2 = (TS)(TS) = T(STS) = TS,
                    \]
                    hkrati pa
                    \[
                        \im T = \im TST \subseteq \im TS \subseteq \im T,
                    \]
                    torej je vmes enakost in lahko vzamemo $Q = TS$.
                    Podobno je
                    \[
                        (I - ST)^2 = I - ST - ST + (ST)^2 = I - ST - ST + ST = I - ST,
                    \]
                    od tod pa sledi
                    \[
                        \im (I - ST) = \ker ST \subseteq \ker TST = \ker T.
                    \]
                    Ker seveda $\ker T \subseteq \ker ST$,
                    so vmes enakosti in lahko vzamemo $P = I - ST$.
                \item \underline{(i) $\implies$ (ii):}
                    Vzemimo $x \in \ker T$.
                    Potem obstaja $z \in X$, da je $Pz = x$.
                    Računamo
                    \[
                        Px = PPz = Pz = x,
                    \]
                    torej je $P$ identična preslikava na $\ker T$ in ničelna preslikava drugod.
                    Analogno dobimo, da je $Q$ identična preslikava na $\im T$ in ničelna preslikava drugod.
                    Preslikavi $P$ in $Q$ lahko obravnavamo kot projekciji na primerna podprostora,
                    ki pa sta oba zaprta po točki (a) (v bistvu že kar po predpostavki) in zato Banachova.
                    Oglejmo si preslikavo $\widetilde{T}$ v naslednjem diagramu

                    \adjustbox{scale=1, center}{
                        \begin{tikzcd}
                            X \arrow[r, "T"] \arrow[d, "I - P"] & Y \arrow[d, "Q"] \\
                            Z \iso X/\ker T \arrow[r, "\widetilde{T}"] & \im T
                        \end{tikzcd}
                    }
                    Preslikava $\widetilde{T}$ je po prvem izreku o izomorfizmu izomorfizem,
                    po diagramu pa je tudi zvezna.
                    Zato obstaja zvezen inverz $\widetilde{S}$,
                    ki pa ga po Hahn-Banachovem izreku razširimo do preslikave $S'\colon Y \to X$.
                    Nazadnje definiramo $S = (I - P)S'Q$.
                    Preslikava $S$ ustreza želenim lastnostim po konstrukciji.
                    Res,
                    \[
                        TSTx =
                        \begin{cases}
                            0 ; \quad  x \in \ker T \\
                            Tx ; \quad  \text{sicer}
                        \end{cases}
                    \]
                    in
                    \[
                        STSy =
                        \begin{cases}
                            0 ; \quad  y \notin \im T \\
                            Sy ; \quad \text{sicer}
                        \end{cases}
                    \]
                    kar zaključi dokaz.
            \end{itemize}
    \end{enumerate}
\end{Exercise}

\begin{Exercise}
    Naj bo $X$ Banachov prostor in $Y,Z \leq X$ zaprta in disjunktna podprostora.
    Označimo
    \[
        k = \inf\left\{ ||y - z|| ; \; y \in Y, z \in Z, ||y|| = ||z|| = 1 \right\}.
    \]
    Dokažimo, da je $Y + Z$ zaprt podprostor natanko tedaj, ko je $k > 0$.

    Najprej se spomnimo naloge $2$ iz poskusne domače naloge,\footnote{
        Za popravke glej Dodatek, točka (a).}
    ki pravi, da je $Y + Z$ zaprt natanko tedaj, ko obstaja $C > 0$,
    da je $ || y || \leq C || y + z ||$ za vse $y \in Y$ in $z \in Z$.
    Od tod takoj sledi implikacija v desno, saj lahko zamenjamo $z \idtfy -z$
    in v posebnem primeru $|| y || = || z || = 1$ dobimo
    \[
        \exists C \colon \quad || y - z || \geq \frac{||y||}{C} = \frac{1}{C} > 0,
    \]
    torej velja to tudi za infimum.

    Obratno, naj bo $k > 0$.
    Radi bi dokazali, da je $Y + Z$ zaprt prostor,
    za kar pa je po poskusni domači nalogi dovolj pokazati,
    da je projekcija $\pr_y \colon Y+Z \to Y$ zvezna preslikava.
    Opazimo, da $k$ zaznamuje razdaljo med sferama $\S_Y$ in $\S_Z$.
    Ker je $k > 0$, ga lahko zapišemo kot $k = \frac{1}{\alpha - 1}$ za neki $\alpha > 1$.
    Računamo
    \begin{align*}
        k &= \frac{1}{ \alpha - 1}
        = \left\| \frac{y}{||y||} - \frac{z}{||z||} \right\|  
        \stackrel{\triangle}{\leq} \left\| \frac{y}{||y||} - \frac{z}{||y||} \right\| + \left\| \frac{z}{||y||} - \frac{z}{||z||} \right\| \\
        &= \frac{||y-z||}{||y||} + \left\| \frac{(||z|| - ||y||)z}{||z||\cdot||y||} \right\|
        = \frac{||y-z||}{||y||} + \frac{1}{||y||} \left| ||z|| - ||y|| \right|
    \end{align*}
    Sedaj lahko zamenjamo $z \idtfy -z$ in premečemo enačbo, da dobimo
    \[
        ||y|| \leq (\alpha - 1) ||y + z|| + | ||y|| - ||z|| |
        \leq (\alpha - 1) ||y+z|| + ||y+z||
        = \alpha ||y+z||
    \]
    iz česar sledi, da je $ \pr_y$ zvezna preslikava.
\end{Exercise}

\begin{Exercise}
    Na Hilbertovem prostoru $L^2[-1, 1]$ definiramo funkcijo $\varphi\colon L^2[-1,1] \to \R$ s predpisom
    \[
        \varphi(f) = \int_{-1}^{1} |f(x)|^2 dx - 2 \int_{-1}^{1} x^2f(x)dx.
    \]
    Naj bo $\mathcal{M} = \left\{ f \in L^2[-1, 1] ; \; \int_{-1}^{1}f(x)dx = 0 \right\} \leq L^2[-1, 1]$.
    Določimo vrednost $\inf_{f \in \mathcal{M}} \varphi(f)$.

    %Najprej omenimo, da je po nekem limitnem izreku iz teorije mere prostor $\mathcal{M}$ zaprt.
    Opazimo, da lahko izraz ``dopolnimo do kvadrata'' in računamo
    \begin{align*}
        \varphi(f) &= \int_{-1}^{1} f(x)^2 dx - \int_{-1}^{1} 2x^2f(x)dx
        = \int_{-1}^{1} \left( f(x)^2 - 2x^2f(x) \right)dx \\
        &= \int_{-1}^{1} \left( \left( f(x) - x^2 \right)^2 - x^4 \right)dx
        = \int_{-1}^{1} \left( f(x) - x^2 \right)^2 dx - \int_{-1}^{1} x^4 dx
    \end{align*}
    Drugi integral je sedaj neodvisen od funkcije $f$,
    prvi pa je vedno nenegativen.
    Torej je potrebno za izračun infimuma minimizirati le prvi integral,
    ki pa ima očitno ničlo v $f(x) = x^2$.
    Težava je seveda, da $x^2 \notin \mathcal{M}$,
    zato si oglejmo $g(x)$ pravokotno projekcijo funkcije $x^2$ na $\mathcal{M}$.

    Konkretno, opazimo, da je $\mathcal{M} = \left\{ f \in L^2[-1, 1] ; \; \langle f, 1 \rangle = 0 \right\}$.
    Od tod sledi, da je $\mathcal{M}^\perp = \Lin\lbrace1\rbrace = \Lin\lbrace\frac{1}{2}\rbrace$,
    kjer je ta množica že sama ONS.
    Računamo
    \[
        P_{\mathcal{M}^\perp} x^2 = \langle x^2, \frac{1}{2} \rangle \frac{1}{2}
        = \frac{1}{2}\int_{-1}^{1}x^2dx \cdot \frac{1}{2}
        = \frac{1}{2}\frac{x^3}{3}\Big|_{-1}^1 \cdot \frac{1}{2}
        = \frac{1}{3} \cdot \frac{1}{2}
    \]
    Potem je $g(x) = x^2 - \frac{1}{3}$.
    Izračunajmo še vrednost funkcionala
    \begin{align*}
        \varphi(g) &= \int_{-1}^{1}\left( x^2 - \frac{1}{3} - x^2 \right)^2 dx - \int_{-1}^{1}x^4dx
        = \int_{-1}^{1}\frac{1}{9} dx - \frac{2}{5}
        = \frac{2}{9} - \frac{2}{5}
        = \frac{10}{45} - \frac{18}{45} = -\frac{8}{45}
        %\varphi(g) &= \int_{-1}^{1}| x^2 - \frac{1}{3} |^2 dx - 2\int_{-1}^{1} \left( x^4 - \frac{1}{3}x^2 \right)dx
        %= 2 \int_{0}^{1} \left( x^4 - \frac{2}{3}x^2 + \frac{1}{9} \right)dx - 4 \int_{0}^{1} \left( x^4 - \frac{2}{3}x^2 \right) \\
        %&= 2 \left( \frac{1}{5} - \frac{2}{9} + \frac{1}{9} \right) - 4 \left( \frac{1}{4} - \frac{2}{9} \right)
        %= \frac{2}{5} - \frac{2}{9} - 1 + \frac{8}{9}
        %= \frac{2}{5} + \frac{2}{3} - 1
        %= \frac{1}{15}
    \end{align*}
\end{Exercise}

\begin{Exercise}
    Definirajmo operator $A\colon l^2 \to l^2$ s predpisom
    \[
        A(x_1,x_2,\dots) = \left( x_1+\frac{1}{2}x_2,-\frac{1}{2}x_1 + x_2,\dots \right)
    \]
    kjer imamo na mestih $2n-1$ in $2n$ elementa $x_{2n-1} + \frac{1}{2n}x_{2n}$ in $-\frac{1}{2n}x_{2n-1} + x_{2n}$.
    \begin{enumerate}[label=(\alph*)]
        \item
            Najprej dokažimo, da je $A$ zvezen in določimo $A^*$.

            Opazimo, da je $A$ vsota dveh operatorjev, in sicer $A = I + D$,
            kjer je
            \[
                Dx = \left( \frac{1}{2}x_2, -\frac{1}{2}x_1,\dots \right)
            \]
            Dovolj je pokazati, da je zvezen operator $D$.
            Računamo
            \[
                \| Dx \|^2 = \sum_{n = 1}^\infty \frac{|x_{2n}|^2 + |x_{2n-1}|^2}{4n^2}
                \leq \frac{1}{4} \sum_{n = 1}^\infty |x_{n}|^2 = \frac{1}{4} \| x \|^2
            \]
            iz česar sledi, da je $D$ zvezen operator.
            Operator $A$ je potem zvezen kot vsota zveznih operatorjev.

            Vemo, da je $A^* = (I + D)^* = I + D^*$,
            zato je zopet dovolj najti $D^*$.
            Razpišemo
            \[
                \langle Dx,y \rangle = \sum_{n = 1}^\infty \frac{x_{2n}\overline{y_{2n-1}}}{2n}
                - \sum_{n = 1}^\infty \frac{x_{2n-1}\overline{y_{2n}}}{2n}
            \]
            Takoj postane jasno, da je
            \[
                D(y_1, y_2, \dots) = \left( -\frac{1}{2}y_2, \frac{1}{2}y_1,\dots \right)
            \]
            torej kot $D$, le da so povsod zamenjani minusi.
            Vstavimo v enačbo in dobimo
            \[
                A^*(y_1,y_2,\dots) = (y_1 - \frac{1}{2}y_2, \frac{1}{2}y_1 + y_2,\dots)
            \]
        \item
            Ali je $A$ surjektiven?
            
            Glede na to, da so koordinate po parih odvisne, sumimo,
            da lahko iz sistema enačb izračunamo kakšno prasliko.
            Na vsakem zaporednem paru nam $Ax = y$ porodi naslednji sistem enačb
            \begin{align*}
                x_{2n-1} + \frac{1}{2n}x_{2n} &= y_{2n-1} \\
                -\frac{1}{2n}x_{2n-1} + x_{2n} &= y_{2n}
            \end{align*}
            Z nekaj preobračanja enačb dobimo enolično rešitev
            \begin{align*}
                x_{2n-1} &= \frac{4n^2}{4n^2+1} \left(y_{2n-1} - \frac{1}{2n}y_{2n} \right) \\
                x_{2n} &= \frac{4n^2}{4n^2+1} \left( \frac{1}{2n}y_{2n-1} + y_{2n} \right)
            \end{align*}
            Vprašati se moramo še, ali je tak $x \in l^2$?
            Odgovor je \underline{da}, saj $\frac{4n^2}{4n^2+1} < 1$, $\frac{1}{2n} < 1$ in $y \in l^2$.
            Operator $A$ je torej surjektiven.
        \item
            Ali je $A$ kompakten?

            Najprej opazimo, da je kompakten operator $D$.
            Res, $Dx = \lim_{n \to \infty} P_{2n}D$, saj velja
            \begin{align*}
                \| (D - P_{2n}D)x \|
                &= \| (\underbrace{0, \dots, 0}_{2n}, \frac{x_{2n+2}}{4n+4}, \frac{x_{2n+1}}{4n+2}, \dots)\|\\
                &\leq \frac{1}{4n+2}\| (\underbrace{0, \dots, 0}_{2n}, x_{2n+2}, x_{2n+1}, \dots)\| \\
                &\leq \frac{1}{4n+2} \|x\|
            \end{align*}
            in zato $\| D - P_{2n}D \| \to 0$.
            Ker je množica kompaktnih operatorjev podprostor, $A$ ne sme biti kompakten operator,
            saj bi sicer bil kompakten tudi $I = A - D$, od tod pa bi sledilo, da $\dim l^2 < \infty$,
            kar je protislovje.
    \end{enumerate}
\end{Exercise}

\begin{Exercise}
    Naj bo $H$ Hardyjev prostor na enotskem disku $\D \subset \C$
    \[
        H = \left\{ f \colon \D \to \C \text{ analitična } ; \;
        \sup_{r < 1} \frac{1}{2\pi} \int_{0}^{2\pi} |f(re^{i\theta})|^2 d\theta < \infty\right\},
    \]
    opremljen s skalarnim produktom
    \[
        \langle f, g \rangle_H = \sup_{r < 1} \frac{1}{2\pi} \int_{0}^{2\pi} f(re^{i\theta})\overline{g(re^{i\theta})} d\theta
    \]
    in pripadajočo normo $\|.\|_H$.

    \begin{enumerate}[label=(\alph*)]
        \item
            Najprej dokažimo, da je za vsak $z \in \D$ operator $\Phi_z\colon H \to \C$,
            podan s predpisom $f \mapsto f(z)$, zvezen.

            Spomnimo se zopet poskusne domače naloge, konkretno naloge (4c).\footnote{
                Za popravke glej Dodatek, točka (b).}
            Če vzamemo Hardyjevo integralsko jedro\footnote{
                Načeloma je integralsko jedro funkcija $k(z, \omega)$,
                kjer pa je seveda $f_z(\omega) = k(z,\omega)$ za vsak $z \in \D$.
                Zato je razlika zanemarljiva.}
            \[
                f_z(\omega) = \frac{1}{1 - \overline{z}\omega}
            \]
            potem je po Cauchyjevi formuli $\Phi_z(f) = \langle f, f_z \rangle_H$ za vse $f \in H$.
            Ker je v Hilbertovih prostorih skalarni produkt zvezen,
            je posledično $\Phi_z$ zvezna preslikava za vsak $z \in \D$.
        \item
            Naj bo še $\|.\|$ norma na $H$, s katero je $H$ Banachov prostor
            in glede na katero so operatorji $\Phi_{\frac{1}{n+1}}$ zvezni za vsak $ n \in \N$.
            Pokažimo, da sta normi $\|.\|$ in $\|.\|_H$ ekvivalentni na $H$.

            Imamo torej, da za vsak $n \in \N$ obstaja $C_n > 0$,
            da je $| \Phi_{\frac{1}{n+1}}(f) | \leq C_n \|f\|$.
            Zaradi harmoničnosti holomorfnih funkcij na kompaktnih diskih $\frac{1}{n+1}\D$
            (uporabimo princip maksimuma)
            je nujno $\lim_{n \to \infty} C_n = 0$, torej obstaja $C = \max_n C_n$.

            Po posledici izreka o odprti preslikavi je dovolj pokazati,
            za obstaja konstanta $D$, da je $\|f\|_H \leq D \|f\|$ za vsak $f \in \D$
            (vzamemo identično preslikavo $(H, \|.\|) \to (H, \|.\|_H)$).
            Računamo
            \begin{align*}
                \|f\|_H^2 &= \sup_{r<1}\frac{1}{2\pi}\int_{0}^{2\pi}\Big|f(re^{i\theta})\Big|^2d\theta
                = \sup_{r<1}\frac{1}{2\pi}\int_{0}^{2\pi}\Big| \Phi_{re^{i\theta}}(f)\Big|^2 d\theta \\
                &= \lim_{n \to 0} \int_{0}^{2\pi} \Big| \Phi_{\frac{1}{n+1}e^{i\theta}}(f) \Big|^2 d\theta
                \leq \lim_{n\to 0} \frac{1}{2\pi} \int_{0}^{2\pi}C^2\|f\|^2 d\theta \\
                &= C^2 \|f\|^2
            \end{align*}
            torej res $\|f\|_H \leq C \|f\|$ za vsak $f \in H$.
    \end{enumerate}
\end{Exercise}

\begin{Exercise}
    Naj bo $H$ Hilbertov prostor in $A \in \mathcal{B}(H)$.
    Dokažimo naslednjo zaporedno verigo implikacij.
    \begin{enumerate}[label=(\alph*)]
        \item
            Obstaja Hilbertov prostor $K$, ki vsebuje prostor $H$,
            in tak normalen operator $B \in \mathcal{B}(K)$,
            da je $B(H) \subseteq H$ in $B|_H = A$.
        \item
            Za vsak $n \in \N$ in izbor vektorjev $x_0,\dots,x_n \in H$ velja
            \begin{equation}
                \sum_{i,j = 0}^n \langle A^ix_j, A^jx_i \rangle \geq 0
                \label{eq:nic}
            \end{equation}
            Poleg tega obstaja tudi taka konstanta $c > 0$,
            da za poljuben izbor vektorjev $x_0,\dots,x_n \in H$ velja
            \begin{equation}
                \sum_{i,j=0}^n \langle A^{i+1}x_j, A^{j+1}x_i \rangle
                \leq c \sum_{i,j=0}^n \langle A^ix_j,A^jx_i\rangle
                \label{eq:konst}
            \end{equation}
        \item
            Obstaja konstanta $c > \|A\|^2$, da je za vsak $n \in \N$
            in poljuben izbor vektorjev $x_0,\dots,x_n \in H$
            izpolnjen pogoj \eqref{eq:konst}.
        \item
            Pogoj \eqref{eq:nic} velja za vsak $n \in \N$ in
            vsak izbor vektorjev $x_0,\dots,x_n \in H$.
        \item
            Za vsak $n \in \N$ in vektorje $x_0,\dots,x_n \in H$ velja
            \[
                \sum_{i,j=0}^n \langle A^{i+j}x_i,A^{i+j}x_j\rangle \geq 0.
            \]
    \end{enumerate}
    \begin{itemize}
        \item \underline{(a) $\implies$ (b):}
            Za $n = 0$ je rezultat trivialen.
            Za $n = 1$ dobimo
            \begin{align*}
                \sum_{i,j=0}^1 \langle A^jx_i, A^ix_j \rangle &=
                \langle x_0,x_0 \rangle + \langle Ax_0,x_1 \rangle
                + \langle x_1, Ax_0 \rangle + \langle Ax_1,Ax_1\rangle \\
                &= \langle x_0,x_0 \rangle + \langle Bx_0,x_1 \rangle
                + \langle x_1, Bx_0 \rangle + \langle Bx_1,Bx_1\rangle \\
                &= \|x_0\|^2 + 2\Re\langle Bx_0, x_1\rangle + \|Bx_1\|^2 \\
                &= \|x_0\|^2 + 2\Re\langle x_0, B^*x_1\rangle + \|B^*x_1\|^2 \\
                &= \| x_0 + B^*x_1\|^2 = \| x_0 + A^*x_1 \|^2
            \end{align*}
            po polarizacijski identiteti.
            Sumimo, da za vsak $n \in \N$ velja
            \[
                \sum_{i,j=0}^n \langle A^jx_i, A^ix_j\rangle = \|x_0 + A^*x_1 + \cdots + A^{*n}x_n \|^2
            \]
            Dokažimo indukcijski korak
            \begin{align*}
                \sum_{i,j=0}^{n+1} \langle A^jx_i, A^ix_j \rangle
                &= \sum_{i,j=0}^{n+1} \langle B^jx_i, B^ix_j \rangle \\
                &= \|x_0 + \cdots + B^{*n}x_n\|^2 + \sum_{i = 0}^n 2\Re\langle B^{*i}x_i, B^{*(n+1)}x_{n+1} \rangle + \|B^{*(n+1)}x_{n+1}\|^2 \\
                &= \|x_0 + \cdots + B^{*n}x_n\|^2 +2\Re\langle  \sum_{i = 0}^n B^{*i}x_i,  B^{*(n+1)}x_{n+1} \rangle + \|B^{*(n+1)}x_{n+1}\|^2 \\
                &= \|x_0 + \cdots + B^{*(n+1)}x_{n+1}\|^2
                = \|x_0 + \cdots + A^{*(n+1)}x_{n+1}\|^2
            \end{align*}
            zopet po polarizacijski identiteti,
            kjer na vsakem koraku uporabljamo normalnost $B$,
            da lahko svobodno menjamo $B$ in $B*$ in dobimo želeno obliko.
            S tem je \eqref{eq:nic} očitno dokazana.
            Po zgornjem računu je \eqref{eq:konst} ekvivalentna temu, da je
            \[
                \| B^*(x_0 + \cdots + B^{*n}x_n) \|^2 \leq c \| x_0 + \cdots + B^{*n}x_n \|^2,
            \]
            kar pa je očitno, saj je $B^*$ zvezen operator.
        \item \underline{(b) $\implies$ (c):}
            Po \eqref{eq:nic} je desna stran enačbe \eqref{eq:konst} (lahko tudi brez $c$) nenegativna.
            Iz \eqref{eq:nic} za izbiro $(Ax_0,\dots,Ax_n)$ prav tako sledi,
            da je leva stran enačbe \eqref{eq:konst} nenegativna.
            Torej imamo enačbo oblike $X = cY$, kjer sta oba $X$ in $Y$ pozitivna.
            Enačba bo torej res tudi za vsako konstanto $d \geq c$,
            zato vzamemo $d = \max\left\{ c, \|A\|^2 + 1 \right\} > \|A\|^2$.
        \item \underline{(c) $\implies$ (d):}
            Z rekurzivnim upoštevanje točke (c) dobimo, da je
            \[
                \sum_{i,j=0}^n \langle A^{j+k}x_i,A^{i+k}x_j\rangle
                \leq c \sum_{i,j=0}^n \langle A^{j+k-1}x_i, A^{i+k-1}x_j \rangle
            \]
            za vsak $k \in \N$ (predpostavka je natanko $k = 1$).
            Od tod dobimo padajoče zaporedje
            \[
                \cdots \leq \frac{\sum_{i,j=0}^n \langle A^{j+k}x_i,A^{i+k}x_j\rangle}{c^k}
                \leq \cdots
                \leq \frac{\sum_{i,j=0}^n \langle A^{j+1}x_i,A^{i+1}x_j\rangle}{c}
                \leq \sum_{i,j=0}^n \langle A^{j}x_i,A^{i}x_j\rangle
            \]
            Dovolj je torej pokazati, da to zaporedje limitira k pozitivni vrednosti,
            ko $k \to \infty$, oziroma kar k $0$.
            V ta namen je dovolj pokazati, da so zgornje vsote primerno omejene.
            Ocenimo
            \begin{align*}
                \Big| \sum_{i,j=0}^n \langle A^{j+k}x_i,A^{i+k}x_j\rangle\Big|
                &\stackrel{\triangle}{\leq} \sum_{i,j=0}^n \Big|\langle A^{j+k}x_i,A^{i+k}x_j\rangle\Big|\\
                &\stackrel{\text{CSB}}{\leq} \sum_{i,j=0}^n \|A^{j+k}x_i\|\cdot \|A^{i+k}x_j\|\\
                &\leq \sum_{i,j=0}^n \|A\|^k\cdot\|A^jx_i\|\cdot\|A\|^k\cdot\|A^ix_j\|\\
                &\leq \|A\|^{2k}\sum_{i,j=0}^n \|A^jx_i\|\cdot\|A^ix_j\|
            \end{align*}
            in opazimo, da je zadnja vsota neodvisna od $k$ in jo označimo z $D$.
            Ker je po (c) $\|A\|^{2k} < c^k$,
            računamo
            \[
                \lim_{k \to \infty}\Big\| \frac{\sum_{i,j=0}^n \langle A^{j+k}x_i,A^{i+k}x_j\rangle}{c^k}\Big\|
                \leq \lim_{k\to\infty}\underbrace{\frac{\|A\|^{2k}}{c^k}}_{<1} D = 0
            \]
            S tem je implikacija dokazana, saj je zgornje zaporedje padajoče.
        \item \underline{(d) $\implies$ (e):}
            Definiramo vektorje $y_i = A^ix_i$ in vstavimo v enačbo
            \[
                \sum_{i,j=0}^n \langle A^{i+j}x_i, A^{i+j}x_j \rangle
                = \sum_{i,j=0}^n \langle A^jy_i, A^iy_j \rangle
                \stackrel{\eqref{eq:nic}}{\geq} 0
            \]
    \end{itemize}
\end{Exercise}

\underline{\textbf{Dodatek - poprava poskusne domače naloge:}}
V tem razdelku bom dopolnil/popravil tiste dele poskusne domače naloge,
ki se navezujejo na rešitve te (prave) domače naloge.
\begin{enumerate}[label=(\alph*)]
    \item \underline{Naloga 2:}
        Za dokaz, da iz tega, da je $Y \otimes Z \leq X$ zaprt podprostor,
        sledi, da je $\pr_y$ zvezna preslikava, uporabimo izrek o zaprtem grafu.
        Naj bo $(y_n, z_n)_n \to x$ konvergentno zaporedje v $Y \oplus Z$
        in naj bo $(\pr_y(y_n,z_n))_n = (y_n)_n \to y$ zaporedje slik v $Y$.
        Radi bi pokazali, da je $\pr_y(x) = y$, kar pa je očitno,
        saj je po predpostavki $x \in Y \oplus Z$, torej $x = (y_0, z)$ za neka $y_0 \in Y$ in $z \in Z$.
        Ker je $y_0 = \lim_{n\to\infty}y_n$, je po enoličnosti limite $y_0 = y$,
        torej $\pr_y(x) = \pr_y(y, z) = y$.
    \item \underline{Naloga 4:}
        S Cauchyjevo integralsko formulo dokažimo, da je
        \[
            f_z(\omega) = \frac{1}{1 - \overline{z}\omega}
        \]
        res t.i. Hardyjevo integralsko jedro,
        (torej Rieszov vektor za evaluacijski funkcional):
        \begin{align*}
            \langle f, f_z\rangle_H
            &= \lim_{r \to 1} \frac{1}{2\pi} \int_{0}^{2\pi} f(re^{i\theta}) \overline{\left(\frac{1}{1 - \overline{z}\omega}\right)} d\theta
            = \lim_{r \to 1} \frac{1}{2\pi} \int_{0}^{2\pi} \frac{f(re^{i\theta})}{1 - zre^{-i\theta}} d\theta \\
            &= \lim_{r \to 1} \frac{1}{2i\pi} \int_{0}^{2\pi} \frac{f(re^{i\theta})}{re^{i\theta} - z} ire^{i\theta}d\theta
            = \lim_{r \to 1} \frac{1}{2i\pi} \int_{r\T} \frac{f(r\omega)}{r\omega - z} d\omega \\
            &= \lim_{r \to 1} f_r(z) = f(z)
        \end{align*}
        kjer $f_r(z) = f(rz)$.
\end{enumerate}

\end{document}

%% TEMPLATES
% lists
%\begin{enumerate}[label=(\alph*)]
% diagram
%\adjustbox{scale=1, center}{
%	\begin{tikzcd}
%		\R_n \arrow[d, "\varphi_n"] \arrow[r, "\Phi"] & \R_m \arrow[d, "\varphi_m"] \\
%		\R \arrow[r, "\widetilde{\Phi}"] & \R
%	\end{tikzcd}
%}
% figure
%\begin{figure}[h]
%	\centering
%	\includegraphics[scale=0.4]{fig}
%	\caption{caption}
%	\label{fig:label}
%\end{figure}
% wrapped figure
%\begin{wrapfigure}{r}{3in}
%	\includegraphics[scale=0.4]{fig}
%	\caption{caption}
%	\label{fig:label}
%\end{wrapfigure}
% beamer
%\documentclass[a4paper, 12pt]{beamer}
%\usetheme{CambridgeUS}
%\usecolortheme{beaver}
%\usefonttheme{structuresmallcapsserif}
% sth above equality
%\stackrel{?}{=}
